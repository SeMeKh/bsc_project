\زیرقسمت{\واژه{متدولوژی}‌ها و \واژه{تکنیک}‌های توسعه‌ی نرم‌افزار}

\واژه{اس.دی.ام.} چارچوبی را برای اعمال \واژه{پرکتیس}‌های مهندسی
نرم‌افزار با هدف «فراهم نمودن ابزارهای لازم برای
توسعه‌ی سیستم‌های \واژه{سافتویر.اینتنسیو}» فراهم می‌آورد. \واژه{متدولوژی} شامل دو عنصر اصلی زیر است
\مرجع{ramsin2006engineering}:

\شروع{شمارش}

\فقره{ مجموعه‌ای از \واژه{کانونشن}‌ها که شامل \واژه{زبان.مدلینگ}
(\واژه{سینتکس} و \واژه{سمنتیکس}) است. }

\فقره{ یک فرآیند، که ترتیب \واژه{فعالیت}‌ها را مشخص می‌سازد،
\واژه{آرتیفکت}‌هایی که با \واژه{زبان.مدلینگ} توسعه می‌یابند را شرح
می‌دهد، وظیفه‌مندی افراد توسعه‌دهنده و تیم را روشن ساخته و شرایطی برای
پایش و اندازه‌گیری محصول‌ها و فعالیت‌ها فراهم می‌آورد. }

\پایان{شمارش}

هر متدولوژی می‌تواند از چند \واژه{تکنیک}، که \واژه{پرکتیس} هم نامیده
می‌شود، در فرآیندها و \واژه{کانونشن}‌های خود بهره گیرد. برای مثال،
\واژه{متدولوژی} \واژه{ایکس.پی.}، از \واژه{تکنیک}‌هایی مانند
\واژه{تی.دی.دی.}، \واژه{پی.پی.} و \واژه{سی.آی.} بهره می‌گیرد
\مرجع{andres2004extreme}.

هر \واژه{تکنیک} نیز جنبه‌ای از فرآیندها و/یا \واژه{کانونشن}‌ها را تعیین
می‌نماید. \واژه{تکنیک}‌ها مختص یک \واژه{متدولوژی} خاص نیستند و ممکن است
توسط چندین \واژه{متدولوژی} استفاده شوند.

%%% Local Variables:
%%% mode: latex
%%% TeX-master: "../main"
%%% End:
