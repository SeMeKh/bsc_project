\قسمت{تاریخچه}\برچسب{sec:back:history}

روند تکامل رویکرد رایج در \واژه{تست.ن.ا.}، از سال ۱۹۵۷ تا سال ۲۰۰۰ به
پنج دوره‌ی \نام{رفع‌اشکال-محور}{Debugging-oriented}،
\نام{اثبات-محور}{Demonstration-oriented}،
\نام{تخریب-محور}{Destruction-oriented}،
\نام{ارزیابی-محور}{Evaluation-oriented} و
\نام{پیشگیری-محور}{Prevention-oriented} تقسیم‌بندی
‌می‌شود \مرجع{laycock1993theory}. برخی فعالیت‌های بارز این دوره‌ها به شرح
زیر است: \شروع{شمارش}

\فقره{ تا سال ۱۹۵۷، برنامه نویس‌ها پس از نوشتن برنامه، آن را اجرا نموده
تا از کارکرد صحیح آن اطمینان یابند و اگر \واژه{باگ}ی در اجرا پیدا
می‌شد، آن را پیدا کرده و رفع می‌نمودند. این فرآیند ادامه پیدا می‌کرد تا
زمانی که احساس کنند \واژه{باگ}ی باقی نمانده است.  }

\فقره{ در سال ۱۹۵۷، \نام{بیکر}{Charles L. Baker} برای اولین بار
\واژه{تست.ن.ا.} را، به‌عنوان روشی برای «اثبات عملکرد صحیح» برنامه، از
\واژه{دیباگ} تمیز داد \مرجع{baker1957review}.  }

\فقره{ \نام{دایسترا}{Edsger Dijkstra} در سال ۱۹۶۹ متذکر شد کاربرد \واژه{تست}،
«یافتن \واژه{باگ}‌ها» است، نه اثبات عملکرد صحیح
برنامه \مرجع{testinghistory}.  در سال ۱۹۸۳، راهنمایی برای
\واژه{اعتبار.سنجی}، \واژه{وارسی} و \واژه{تست.ن.ا.} منتشر شد
\مرجع{adrion1982validation} که «تشخیص \واژه{باگ}‌های تحلیل و طراحی» را،
علاوه بر تشخیص \واژه{باگ}‌های پیاده‌سازی، با \واژه{تست.ن.ا.} میسر می‌سازد
\مرجع{luo2001software}.  }

\فقره{ در سال ۱۹۸۷، معیار \واژه{اتکا.پذیری} به‌عنوان عاملی کلیدی در
«اندازه‌گیری \واژه{کیفیت.ن.ا.}» معرفی شد \مرجع{musa1987software}.  }

\فقره{ در سال ۱۹۹۰، \واژه{تست.ن.ا.} به عنوان یکی از موثرترین عوامل
«پیشگیری از \واژه{باگ}»‌ معرفی شد \مرجع{beizer20672software}.  }

\پایان{شمارش}

پس از رویکردهای مذکور در اواخر دهه ۱۹۹۰، با ظهور \واژه{متدولوژی}‌های
جدید مانند \واژه{ایکس.پی.}، که در دسته‌بندی \واژه{متدولوژی}‌های
\واژه{ای.اس.دی.} قرار می‌گیرد، رویکردهای جدیدی در \واژه{تست.ن.ا.} ایجاد
شد.

در \واژه{متدولوژی}‌های سنتی با \واژه{مدل.آبشاری}، \واژه{تست} یکی از
فاز‌های \واژه{اس.دی.پی.} بود که فقط یک بار انجام می‌شد و پیش‌نیاز آن،
اتمام فازهای \واژه{تحلیل}، \واژه{طراحی} و \واژه{پیاده.سازی} بود. در
\واژه{متدولوژی}‌های آبشاری، هر کدام از فازهای \واژه{تحلیل}،
\واژه{طراحی}، \واژه{پیاده.سازی} و \واژه{تست} به‌صورت مجزا اجرا می‌شد
\مرجع{agileVsWaterfall} و \واژه{تست} نقش موثری در فازهای تحلیل، طراحی
و پیاده‌سازی نداشت.

اما در \واژه{متدولوژی}‌های چابک، مراحل تحلیل، طراحی، پیاده سازی و
\واژه{تست} به‌صورت \واژه{تکراری} اجرا می‌شود. در هر \واژه{تکرار}،
نیازمندی کامل نرم‌افزار مشخص نیست و در \واژه{تکرار} بعد، ممکن است تغییر
کند. نقش \واژه{تست} در «پاسخگویی به تغییرات»، که یکی از چهار ارزش
\واژه{بیانیه.اجایل} است \مرجع{manifesto}، بسیار کلیدی است. در
\واژه{تست} چابک، رویکرد «\نام{دستیاری در کیفیت}{Quality Assistance}»
بر رویکرد «\نام{اطمینان از کیفیت}{Quality Assurance}» غلبه می‌کند \مرجع{ghahrai}.

\نام{بک}{Kent Beck} در سال ۲۰۰۲، با معرفی \واژه{تکنیک}
\واژه{تی.دی.دی.}، \واژه{تست} را به‌عنوان «محرک توسعه‌ی نرم‌افزار» معرفی
می‌کند و نوشتن \واژه{تست} را پیش‌نیاز \واژه{پیاده.سازی} می‌داند. وی
«بهبود طراحی نرم‌افزار» را از نتایج بکارگیری این \واژه{تکنیک} می‌داند
\مرجع{beck2003test}. همچنین با بهره‌گیری از این نگرش در \واژه{متدولوژی}
\واژه{ای.ام.دی.دی.}، از \واژه{تست.ن.ا.} به‌عنوان «\واژه{توصیف}
نرم‌افزار» بهره گرفته می‌شود \مرجع{agileModeling:specByExample}.


%%% Local Variables:
%%% mode: latex
%%% TeX-master: "../main"
%%% End:
