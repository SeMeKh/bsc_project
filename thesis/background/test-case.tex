\زیرقسمت{\واژه{تست}، \واژه{تست.کیس} و \واژه{تست.سویت}}

یک \واژه{تست.کیس}، از ورودی و خروجی مورد انتظار قسمتی از برنامه برای
آن ورودی تشکیل می‌شود. برای معتبر بودن یک \واژه{تست.کیس}، ممکن‌است لازم
باشد ورودی دارای \واژه{پری.کاندیشن}‌هایی و خروجی دارای
\واژه{پست.کاندیشن}‌هایی نیز باشد. از واژه‌ی «\واژه{تست}» نیز می‌توان به
جای واژه‌ی «\واژه{تست.کیس}» استفاده کرد \مرجع{singh2011software}.

وضعیت اجرای یک \واژه{تست.کیس} دو حالت دارد؛ \واژه{سبز} (موفقیت‌آمیز) به
معنی سازگاری خروجی برنامه با خروجی مورد انتظار و \واژه{قرمز} (شکست) به
معنی عدم سازگاری خروجی برنامه با خروجی مورد انتظار می‌باشد
\مرجع{alience:unit}.

یک «\واژه{تست.سویت}» شامل مجموعه‌ای از \واژه{تست}‌ها می‌باشد که وضعیت
اجرای آن \واژه{تست.سویت}، در صورتی \واژه{سبز} است که تک‌تک
\واژه{تست.کیس}‌های آن در وضعیت \واژه{سبز} باشند \مرجع{singh2011software}.


%%% Local Variables:
%%% mode: latex
%%% TeX-master: "../main"
%%% End:
