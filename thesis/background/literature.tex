\قسمت{ادبیات}\برچسب{sec:back:vocabulary}

اصطلاح‌های مرتبط با توسعه‌ی نرم‌افزار، ابهام‌ها و برداشت‌های گوناگون دارند
برای مثال در \مرجع{gartner2012atdd}، هر چهار اصطلاح
«\واژه{ای.تی.دی.دی.}»، «\واژه{بی.دی.دی.}»، «\نام{توصیف با
  مثال}{Specification by Example}»، «\نام{آزمون پذیرش چابک}{Agile
  Acceptance Testing}» و «\نام{تست داستان کاربر}{User Story Testing}»،
هم‌معنی تلقی شده‌اند. برای پیشگیری از ابهام و شفاف‌سازی اصطلاح‌های استفاده‌شده
در این پایان‌نامه، در این قسمت توضیحی اجمالی و بدون ارزیابی جنبه‌های
مختلف، برای هر اصطلاح ارائه شده است.

\زیرقسمت{\واژه{تست}، \واژه{تست.کیس} و \واژه{تست.سویت}}
\برچسب{sec:testcase}

یک \واژه{تست.کیس}، از ورودی و خروجی مورد انتظار قسمتی از برنامه برای
آن ورودی تشکیل می‌شود. هر عملکرد \واژه{تست.کیس} قسمت مشخصی از سامانه را
که «\واژه{sut}» نامیده می‌شود، مورد آزمون قرار می‌دهد. برای معتبر بودن
یک \واژه{تست.کیس}، ممکن‌است لازم باشد ورودی \واژه{sut} دارای
\واژه{پری.کاندیشن}‌هایی و خروجی دارای \واژه{پست.کاندیشن}‌هایی نیز
باشد. از واژه‌ی «\واژه{تست}» نیز می‌توان به جای واژه‌ی «\واژه{تست.کیس}»
استفاده کرد \مرجع{singh2011software}.

وضعیت اجرای یک \واژه{تست.کیس} دو حالت دارد؛ \واژه{سبز} (موفقیت‌آمیز) به
معنی سازگاری خروجی برنامه با خروجی مورد انتظار و \واژه{قرمز} (شکست) به
معنی عدم سازگاری خروجی برنامه با خروجی مورد انتظار می‌باشد
\مرجع{alience:unit}.

یک «\واژه{تست.سویت}» شامل مجموعه‌ای از \واژه{تست}‌ها می‌باشد که وضعیت
اجرای آن \واژه{تست.سویت}، در صورتی \واژه{سبز} است که تک‌تک
\واژه{تست.کیس}‌های آن در وضعیت \واژه{سبز} باشند \مرجع{singh2011software}.


%%% Local Variables:
%%% mode: latex
%%% TeX-master: "../main"
%%% End:


\زیرقسمت{دسته‌بندی مدل \چر{V}} یکی از دسته‌بندی‌های مشهور برای سطح‌های
\واژه{تست}، دسته‌بندی مدل \چر{V} می‌باشد. این دسته‌بندی، پیش از پیدایش
\واژه{متدولوژی}‌های چابک، در \واژه{متدولوژی} «\واژه{وی.مدل}»، که با کمی
تغییر در \واژه{متدولوژی} \واژه{مدل.آبشاری} و تمرکز بیشتر بر \واژه{تست}
بوجود آمده، تعریف شده است \مرجع{singh2011software}.

وجه تمایز سطح‌ها در این دسته‌بندی، مراحل تولید نرم‌افزار و تقدم زمانی
فازهای تحلیل، طراحی و پیاده‌سازی می‌باشد. همانطور که در شکل
\رجوع{fig:v-model} نشان داده شده، پس از هر فاز، \واژه{تست}ی برای
\واژه{آرتیفکت} آن فاز نوشته می‌شود \مرجع{singh2011software}.

\شروع{شکل}[tbp]
\تنظیم‌ازوسط
\درج‌تصویر[پهنا=0.7\پهنای‌سطر]{v-model}
\تر\موقتم{
  نمودار سطح‌های آزمون در \واژه{وی.مدل}
}
\شرح[\موقتم]{\موقتم~\مرجع{singh2011software}}
\برچسب{fig:v-model}
\پایان{شکل}

در این دسته‌بندی چهار سطح «\واژه{اکسپتنس.تست}»، «\واژه{سیستم.تست}»،
«\واژه{اینتگریشن.تست}» و «\واژه{یونیت.تست}» به ترتیب برای مراحل تحلیل،
طراحی سطح بالا، طراحی سطح پایین و پیاده‌سازی معرفی شده اند.

%%% Local Variables:
%%% mode: latex
%%% TeX-master: "../main"
%%% End:


\زیرقسمت{\واژه{هرم.تست}}

\واژه{هرم.تست} اصطلاحی است که اولین بار توسط \نام{کوهن}{Mike Cohn}
برای دسته‌بندی سطح‌های مختلف \واژه{تست} در \واژه{متدولوژی}‌های چابک
معرفی شد. وجه تمایز سطح‌ها در این دسته بندی، هزینه، زمان اجرا و تعداد
\واژه{تست}‌های مورد نیاز می‌باشد.\مرجع{fowler:pyramid}.

\شروع{شکل}[tbp]
\تنظیم‌ازوسط
\درج‌تصویر[پهنا=0.7\پهنای‌سطر]{test-pyramid}
\تر\موقتم{
  نمودار سطح‌های \واژه{تست} در هرم \واژه{تست}
}
\شرح[\موقتم]{\موقتم~\مرجع{fowler:pyramid}}
\برچسب{fig:test-pyramid}
\پایان{شکل}

همانطور که در نمودار \رجوع{fig:test-pyramid} نشان داده شده‌است، در این
دسته‌بندی هرچه سطح \واژه{تست} بالاتر باشد، هزینه و زمان اجرای
\واژه{تست}‌ در آن سطح بیشتر، و تعداد \واژه{تست}‌های مورد نیاز آن سطح،
کمتر می‌شود. با این معیار، سه سطح زیر در این دسته‌بندی تعریف شده است
\مرجع{fowler:pyramid}:

\شروع{شمارش}

\فقره{\متن‌سیاه{سطح واحد}: پایین‌ترین سطح \واژه{هرم.تست}،
  \واژه{یونیت.تست} می‌باشد. هدف اصلی \واژه{یونیت.تست}، این است که واحد
  کوچکی از نرم‌افزار را در نظر گرفته، آن‌را از سایر واحد‌ها مجزا ساخته
  (مانند شکل \رجوع{fig:unit-test}) و بررسی نماییم که این واحد وظیفه‌ی
  خود را مطابق انتظار انجام‌دهد. هر \واژه{یونیت.تست}، به‌صورت مستقل، قبل
  از \واژه{اینتگریشن} با سایر واحدهای \واژه{ماژول}، اجرا می‌شود
  \مرجع{msdn:unit}.

  \واژه{یونیت.تست}‌ تا حد ممکن در سطوح پایین معماری نرم‌افزار مورد
  استفاده قرار می‌گیرد و به قسمت کوچکی از متن برنامه متمرکز است، عموما
  توسط خود برنامه‌نویس‌‌ها به تعداد زیاد نوشته می‌شود و باید بسیار سریع
  اجرا شود \مرجع{fowler:unit}.

  این‌که منظور از یک «واحد» چه باشد، به نیازمندی‌ها و شرایط پروژه بستگی
  دارد. معمولا در برنامه‌نویسی شیء‌گرا، \واژه{کلاس} به‌عنوان واحد انتخاب
  می‌شود \مرجع{fowler:unit}. }

\فقره {\متن‌سیاه{سطح سرویس}: هر نرم‌افزار، از چند سرویس تشکیل شده است که
  پس از دریافت ورودی از لایه‌ی \واژه{یو.آی.}، عملیات مورد انتظار سامانه
  را انجام داده و نتیجه را در اختیار \واژه{یو.آی.} قرار می‌دهند. منطق
  برنامه توسط سرویس‌ها پیاده‌سازی می‌شود. این سطح \واژه{تست}، عملکرد صحیح
  سرویس‌های \واژه{سیستم} را می‌آزماید. \مرجع{cohn2010succeeding}.

  \واژه{تست} سرویس در لایه‌ی میانی \واژه{هرم.تست} قرار
  می‌گیرد. \واژه{تست} در این سطح، هزینه‌های \واژه{تست} سطح \واژه{یو.آی.}
  را ندارد، اما تیم را از \واژه{اینتگریشن} صحیح واحدها در کنار هم
  مطمئن می‌سازد \مرجع{cohn2010succeeding}.  ریز‌دانگی آزمون‌ها در این
  سطح تقریبا معادل سطح‌ \واژه{اینتگریشن.تست} از دسته‌بندی مدل \چر{V}
  می‌باشد. }

\فقره { \متن‌سیاه{سطح \واژه{یو.آی.}}:بالا ترین سطح \واژه{هرم.تست}، سطح
  \واژه{یو.آی.} است. در این سطح، مطابقت عملکرد نهایی \واژه{سیستم} با
  نیازمندی‌هایی که مشتری مشخص می‌نماید، در محصول نهایی بررسی می‌شود. این
  سطح از \واژه{تست} در مقابل تغییرات بسیار شکننده است و هزینه‌ی زیادی
  صرف نوشتن و اجرای آن می‌شود \مرجع{cohn2010succeeding}. ریزدانگی
  \واژه{تست}‌ها در این سطح تقریبا معادل سطح‌های \واژه{سیستم.تست} و
  \واژه{اکسپتنس.تست} از دسته‌بندی مدل \چر{V} می‌باشد.  }

\پایان{شمارش}

\شروع{شکل}[tbp]
\تنظیم‌ازوسط
\درج‌تصویر[پهنا=0.7\پهنای‌سطر]{unit-test}
\تر\موقتم{
  رابطه‌ی میان \واژه{یونیت.تست} و واحد‌های برنامه
}
\شرح[\موقتم]{\موقتم~\مرجع{fowler:unit}}
\برچسب{fig:unit-test}
\پایان{شکل}


%%% Local Variables:
%%% mode: latex
%%% TeX-master: "../main"
%%% End:


\زیرقسمت{\واژه{متدولوژی}‌ها و \واژه{تکنیک}‌های توسعه‌ی نرم‌افزار}

\واژه{اس.دی.ام.} چارچوبی را برای اعمال \واژه{پرکتیس}‌های مهندسی
نرم‌افزار با هدف «فراهم نمودن ابزارهای لازم برای
توسعه‌ی \واژه{سیستم}‌های \واژه{سافتویر.اینتنسیو}» فراهم می‌آورد. \واژه{متدولوژی} شامل دو عنصر اصلی زیر است
\مرجع{ramsin2006engineering}:

\شروع{شمارش}

\فقره{ مجموعه‌ای از \واژه{کانونشن}‌ها که شامل \واژه{زبان.مدلینگ}
(\واژه{سینتکس} و \واژه{سمنتیکس}) است. }

\فقره{ یک فرآیند، که ترتیب \واژه{فعالیت}‌ها را مشخص می‌سازد،
\واژه{آرتیفکت}‌هایی که با \واژه{زبان.مدلینگ} توسعه می‌یابند را شرح
می‌دهد، وظیفه‌مندی افراد توسعه‌دهنده و تیم را روشن ساخته و شرایطی برای
پایش و اندازه‌گیری محصول‌ها و فعالیت‌ها فراهم می‌آورد. }

\پایان{شمارش}

هر متدولوژی می‌تواند از چند \واژه{تکنیک}، که \واژه{پرکتیس} هم نامیده
می‌شود، در فرآیندها و \واژه{کانونشن}‌های خود بهره گیرد. برای مثال،
\واژه{متدولوژی} \واژه{ایکس.پی.}، از \واژه{تکنیک}‌هایی مانند
\واژه{تی.دی.دی.}، \واژه{پی.پی.} و \واژه{سی.آی.} بهره می‌گیرد
\مرجع{andres2004extreme}.

هر \واژه{تکنیک} نیز جنبه‌ای از فرآیندها و/یا \واژه{کانونشن}‌ها را تعیین
می‌نماید. \واژه{تکنیک}‌ها مختص یک \واژه{متدولوژی} خاص نیستند و ممکن است
توسط چندین \واژه{متدولوژی} استفاده شوند.

%%% Local Variables:
%%% mode: latex
%%% TeX-master: "../main"
%%% End:


\زیرقسمت{\واژه{تی.اف.دی.}}

\واژه{تی.اف.دی.} یک \واژه{تکنیک} است که ترتیب فعالیت‌های توسعه‌ی سیستم
توسط توسعه‌دهنده‌ی فنی را مشخص می‌سازد.

همانطور که در نمودار فعالیت شکل \رجوع{fig:tfd} نشان داده شده، برای
اعمال هر تغییر در متن برنامه، ابتدا \واژه{تست} نوشته می‌شود، تنها به
مقداری که آزمون در وضعیت \واژه{قرمز} قرار گیرد. سپس ترجیحا کل
\واژه{تست.سویت} اجرا می‌شود تا مطمئن شویم بقیه‌ی \واژه{تست}‌ها در وضعیت
\واژه{سبز} قرار دارند و فقط آخرین \واژه{تست} در وضعیت \واژه{قرمز} قرار
دارد. سپس تغییر لازم در متن برنامه اعمال می‌شود تا کل \واژه{تست.سویت}
به وضعیت \واژه{سبز} برسد \مرجع{agiledata:tdd}.

\شروع{شکل}[tbp]
  \تنظیم‌ازوسط
  \درج‌تصویر[پهنا=0.5\پهنای‌سطر]{tfd}
  \تر\موقتم{
    نمودار فعالیت مراحل \واژه{تی.اف.دی.}
  }
  \شرح[\موقتم]{\موقتم~\مرجع{agiledata:tdd}}
  \برچسب{fig:tfd}
\پایان{شکل}

%%% Local Variables:
%%% mode: latex
%%% TeX-master: "../main"
%%% End:


\زیرقسمت{\واژه{تی.دی.دی.}}\برچسب{sec:back:tdd}

\واژه{تی.دی.دی.} \واژه{تکنیک}ی است که برنامه‌نویس را به استفاده از دو
\واژه{تکنیک} \واژه{تی.اف.دی.} و \واژه{ریفکتورینگ} با دو هدف اصلی زیر
ملزم می‌کند \مرجع{agiledata:tdd}:

\شروع{شمارش}

\فقره { با بهره‌گیری از \واژه{تی.اف.دی.}، برنامه‌نویس موظف است قبل از
  پیاده‌سازی هر واحد از کد، به نیازمندی آن واحد و طراحی آن واحد فکر
  کرده و در حقیقت با نوشتن \واژه{تست}، نیازمندی آن واحد را نیز توصیف
  می‌کند. }

\فقره { با توجه به اینکه برنامه‌نویس موظف است پس از افزودن هر واحد،
  \واژه{ریفکتورینگ} را به‌عنوان مرحله‌ای اجباری از فرآیند توسعه‌ی
  نرم‌افزار لحاظ کند، تمیزی کد کیفیت طراحی نرم‌افزار همواره در بهترین
  حالت حفظ می‌شود. }

\پایان{شمارش}

همانطور که در نمودار حالت شکل \رجوع{fig:tdd-states} نمایش داده شده‌است،
در حالت \واژه{سبز}، فقط امکان \واژه{ریفکتورینگ} هست و برای تغییر متن
برنامه، باید ابتدا با نوشتن یک \واژه{تست} برای آن، به حالت
\واژه{قرمز} رفته و سپس امکان تغییر متن برنامه را داریم.


\شروع{شکل}[tbp]
\تنظیم‌ازوسط
\درج‌تصویر[پهنا=0.7\پهنای‌سطر]{tdd-states}
\شرح
[نمودار حالت وضعیت‌های \واژه{تست.سویت} و نحوه‌ی گذار بین آنها]
{نمودار حالت وضعیت‌های \واژه{تست.سویت} و نحوه‌ی گذار بین آنها با
  استفاده افزودن \واژه{تست}، \واژه{ریفکتورینگ} و تغییر متن برنامه \مرجع{agiledata:tdd}}
\برچسب{fig:tdd-states}
\پایان{شکل}


%%% Local Variables:
%%% mode: latex
%%% TeX-master: "../main"
%%% End:


\زیرقسمت{\واژه{ای.تی.دی.دی.}}



%%% Local Variables:
%%% mode: latex
%%% TeX-master: "../main.tex"
%%% End:


%%% Local Variables:
%%% mode: latex
%%% TeX-master: "../main"
%%% End:
