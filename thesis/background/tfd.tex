\زیرقسمت{\واژه{تی.اف.دی.}}

\واژه{تی.اف.دی.} یک \واژه{تکنیک} است که ترتیب فعالیت‌های توسعه‌ی \واژه{سیستم}
توسط توسعه‌دهنده‌ی فنی را مشخص می‌سازد.

همانطور که در نمودار فعالیت شکل \رجوع{fig:tfd} نشان داده شده، برای
اعمال هر تغییر در متن برنامه، ابتدا \واژه{تست} نوشته می‌شود، تنها به
مقداری که آزمون در وضعیت \واژه{قرمز} قرار گیرد. سپس ترجیحا کل
\واژه{تست.سویت} اجرا می‌شود تا مطمئن شویم بقیه‌ی \واژه{تست}‌ها در وضعیت
\واژه{سبز} قرار دارند و فقط آخرین \واژه{تست} در وضعیت \واژه{قرمز} قرار
دارد. سپس تغییر لازم در متن برنامه اعمال می‌شود تا کل \واژه{تست.سویت}
به وضعیت \واژه{سبز} برسد \مرجع{agiledata:tdd}.

\شروع{شکل}[tbp]
  \تنظیم‌ازوسط
  \درج‌تصویر[پهنا=0.5\پهنای‌سطر]{tfd}
  \تر\موقتم{
    نمودار فعالیت مراحل \واژه{تی.اف.دی.}
  }
  \شرح[\موقتم]{\موقتم~\مرجع{agiledata:tdd}}
  \برچسب{fig:tfd}
\پایان{شکل}

%%% Local Variables:
%%% mode: latex
%%% TeX-master: "../main"
%%% End:
