\قسمت{\واژه{تکنیک} \واژه{بی.دی.دی.}}\برچسب{sec:back:bdd}

\واژه{بی.دی.دی.} یکی از تکنیک‌های توسعه‌ی چابک نرم‌افزار است که در سال
۲۰۰۶ توسط \نام{نورث}{Dan North} برای پاسخ‌گویی به ابهام‌ها و مشکل‌هایی که
در \واژه{تی.دی.دی.} بوجود آمده‌بود، معرفی
شد. \مرجع{north2006introducing}.  تاکید \واژه{بی.دی.دی.} بر بهینه‌سازی
ارتباط میان نقش‌های برنامه‌نویس، آزمون‌گر و \واژه{دامین.اکسپرت}‌ می‌باشد و
برای تیمی که تمام اعضای آن برنامه‌نویس باشند، \واژه{بی.دی.دی.} سودی
نسبت به \واژه{تی.دی.دی.} ندارد \مرجع{north2012}.

این تکنیک، که اخیرا هم در عمل و هم در پژوهش بسیار متداول شده است، حاصل
ترکیب و بهبود \واژه{پرکتیس}‌های معرفی شده در تکنیک‌های \واژه{تی.دی.دی.}
و \واژه{ای.تی.دی.دی.} و علاوه‌بر آن در نظر گرفتن اصول زیر است
\مرجع{alience:bdd}:

\شروع{بارهها}

\باره{ لازم است هدف هر \واژه{یوزر.استوری} و سودی که برای مشتری دارد،
  مشخص باشد.}

\باره{ با نگرش «\نام{از بیرون به درون}{From the outside in}»، فقط
  رفتارهایی از سامانه پیاده‌سازی می‌شوند که بیشترین سود را به مشتری
  می‌رسانند. با داشتن این نگرش، اتلاف هزینه کمینه می‌شود. }

\باره{ رفتارهای مورد انتظار از سامانه، بین \واژه{دامین.اکسپرت}،
  توسعه‌دهنده‌ی \واژه{سیستم} و آزمون‌گر به یک زبان مشترک توصیف می‌شوند. در نتیجه
  ارتباط ضروری میان این نقش‌ها بهبود می‌یابد. }

\باره { این اصول، در تمام سطوح انتزاعی توصیف برنامه، تا پایین‌ترین
  سطح که پیاده‌سازی یک واحد است، رعایت می‌شوند. }

\پایان{بارهها}

\شروع{definition} به گزاره‌ای که در صورت صحت پیاده‌سازی برقرار باشد یک
\واژه{سناریو} می‌گوییم.  \پایان{definition}

\شروع{definition}
\برچسب{gwt}
در \واژه{بی.دی.دی.} یک \واژه{سناریو} به صورت زیر نمایش داده می‌شود:

{\نقل
\متن‌سیاه{\واژه{given}} شرط $g$ برقرار باشد، \\
\متن‌سیاه{\واژه{when}} اتفاق $w$ رخ بدهد، \\
\متن‌سیاه{\واژه{then}} شرط $t$ باید برقرار باشد.
}

و یا به طور خلاصه:

{\نقل
\متن‌سیاه{\واژه{given}} $g$ \\
\متن‌سیاه{\واژه{when}} $w$ \\
\متن‌سیاه{\واژه{then}} $t$
}

این شیوه‌ی بیان سناریو \واژه{given-when-then} نام دارد.
\پایان{definition}

%%% Local Variables:
%%% mode: latex
%%% TeX-master: "../main"
%%% End:

