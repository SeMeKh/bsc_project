\قسمت{\واژه{تکنیک} \واژه{بی.دی.دی.}}\برچسب{sec:back:bdd}

\واژه{بی.دی.دی.} یکی از تکنیک‌های توسعه‌ی چابک نرم‌افزار است که در سال
۲۰۰۶ توسط \نام{نورث}{Dan North} برای پاسخ‌گویی به ابهام‌ها و مشکل‌هایی که
در \واژه{تی.دی.دی.} بوجود آمده‌بود، معرفی
شد. \مرجع{north2006introducing}.  تاکید \واژه{بی.دی.دی.} بر بهینه‌سازی
ارتباط میان نقش‌های برنامه‌نویس، آزمون‌گر و \واژه{دامین.اکسپرت}‌ می‌باشد و
برای تیمی که تمام اعضای آن برنامه‌نویس باشند، \واژه{بی.دی.دی.} سودی
نسبت به \واژه{تی.دی.دی.} ندارد \مرجع{north2012}.

این تکنیک، که اخیرا هم در عمل و هم در پژوهش بسیار متداول شده است، حاصل
ترکیب و بهبود \واژه{پرکتیس}‌های معرفی شده در تکنیک‌های \واژه{تی.دی.دی.}
و \واژه{ای.تی.دی.دی.} و علاوه‌بر آن در نظر گرفتن اصول زیر است
\مرجع{alience:bdd}:

\شروع{بارهها}

\باره{ لازم است هدف هر \واژه{یوزر.استوری} و سودی که برای مشتری دارد،
  مشخص باشد.}

\باره{ با نگرش «\نام{از بیرون به درون}{From the outside in}»، فقط
  رفتارهایی از \واژه{سیستم} پیاده‌سازی می‌شوند که بیشترین سود را به
  مشتری می‌رسانند. با داشتن این نگرش، اتلاف هزینه کمینه می‌شود. }

\باره{ رفتارهای مورد انتظار از سامانه، بین \واژه{دامین.اکسپرت}،
  توسعه‌دهنده‌ی \واژه{سیستم} و آزمون‌گر به یک زبان مشترک توصیف می‌شوند. در نتیجه
  ارتباط ضروری میان این نقش‌ها بهبود می‌یابد. }

\باره { این اصول، در تمام سطوح انتزاعی توصیف برنامه، تا پایین‌ترین
  سطح که پیاده‌سازی یک واحد است، رعایت می‌شوند. }

\پایان{بارهها}

\زیرقسمت{مشخصات \واژه{بی.دی.دی.}}  در حال حاضر، \واژه{تکنیک}
\واژه{بی.دی.دی.}، هنوز در حال توسعه است و تعریف روشنی از
\واژه{بی.دی.دی.} که بر آن اجماع باشد، وجود ندارد. با توجه به اینکه
فرآیند \واژه{بی.دی.دی.}، از ابتدا به‌صورت انتزاعی معرفی شده و جزییاتی
برای آن ارائه نشده، مشخصات \واژه{بی.دی.دی.} مبهم و پراکنده
هستند. همچنین چارچوب‌ها و ابزارهایی که برای \واژه{بی.دی.دی.} ارائه
شده‌اند، بیشتر به بخش «پیاده‌سازی آزمون» از فرآیند \واژه{بی.دی.دی.}
تمرکز دارند؛ در صورتی که \واژه{بی.دی.دی.} بر حوزه‌ی گسترده‌تری از
\واژه{اس.دی.پی.} تاثیر دارد \مرجع{solis2011study}.

در مقاله‌ی \مرجع{solis2011study}، ۶ مورد از مشخصات اصلی
\واژه{بی.دی.دی.} که بر کل \واژه{اس.دی.پی.}، نه فقط قسمت «پیاده‌سازی
آزمون»، تاثیر گذار اند، ارائه شده است. در اینجا به صورت خلاصه به آن‌ها
اشاره می‌کنیم:

\شروع{شمارش}

\فقره{ \متن‌سیاه{\واژه{یوبی}:}

  «\واژه{یوبی}»، هسته‌ی \واژه{بی.دی.دی.} را تشکیل می‌دهد. \واژه{یوبی}،
  زبانی است که برآمده از \واژه{بیزنس.دامین} می‌باشد و ابهام را از
  مکالمه‌ی بین مشتری و تیم توسعه کاهش می‌دهد. همچنین یک واژه‌نامه در
  ابتدای پروژه ایجاد شده، اکثر واژه‌های آن در مرحله‌ی تحلیل افزوده
  می‌شوند و در مراحل بعد امکان گسترش دارد.

  هر \واژه{بیزنس.دامین}، \واژه{یوبی} خاص خود را لازم دارد.
  \واژه{بی.دی.دی.} یک \واژه{قالب} ساده برای مرحله‌ی تحلیل ارائه نموده
  است که مستقل از \واژه{بیزنس.دامین} است و از آن در \واژه{یوبی}
  بهره‌گیری می‌شود. این \واژه{قالب} در بخش \رجوع{gwt} شرح داده شده است.
}

\فقره{ \متن‌سیاه{فرآیند \واژه{تفکیک} \واژه{تکراری}:}

  توقع مشتری از یک پروژه‌ی نرم‌افزاری، کسب \واژه{بیزنس.ولیو}
  می‌باشد. معمولاً تشخیص و روشن‌سازی \واژه{بیزنس.ولیو}، دشوار است. به
  همین دلیل، \واژه{بیزنس.ولیو} به اجزای ملموس‌تر تفکیک می‌شود. در شکل
  \رجوع{fig:bdd-conceptual} رابطه‌ی این اجزا نسبت به هم نمایش داده شده
  است.

  \شروع{شکل}[tbp]
  \تنظیم‌ازوسط
  \درج‌تصویر[پهنا=1\پهنای‌سطر]{bdd-conceptual}
  \تر\موقتم{
    مدل مفهومی در تحلیل \واژه{بی.دی.دی.}
  }
  \شرح[\موقتم]{\موقتم~\مرجع{solis2011study}}
  \برچسب{fig:bdd-conceptual}
  \پایان{شکل}
 
  مرحله‌ی تحلیل در \واژه{بی.دی.دی.}، با شناسایی \واژه{رفتار}های مورد
  انتظار از سامانه آغاز می‌شود. در مراحل ابتدایی، شناسایی رفتارهای مورد
  انتظار آسان‌تر از شناسایی ارزش‌های تجاری سامانه است. هر رفتار تجاری،
  تعدادی \واژه{نتیجه.تجاری} محسوس را محقق می‌سازد.

  هر \واژه{نتیجه.تجاری} با یک \واژه{فیچر.ست} اعمال می‌شود. یک
  \واژه{فیچر.ست}‌ با گفتگو بین تیم توسعه و مشتری است برای تحقق
  \واژه{نتیجه.تجاری} در سامانه تدوین گشته و اولویت‌بندی و تبیین صریح
  ارتباظ آن با \واژه{نتیجه.تجاری}، ضروری است.

  در نهایت \واژه{اسکوپ} هر \واژه{فیچر.ست}، با استفاده از چند
  \واژه{یوزر.استوری} معین می‌شود. هر \واژه{یوزر.استوری}، یک تعامل بین
  کاربر و سامانه را توصیف می‌کند. هر \واژه{یوزر.استوری}، به سه
  سوال زیر پاسخ می‌دهد:

  \شروع{بارهها}

  \باره{نقش کاربر در \واژه{یوزر.استوری} چیست؟}
  
  \باره{کاربر از این \واژه{فیچر} چه می‌خواهد؟}

  \باره{اگر سامانه این \واژه{فیچر} را داشته باشد، چه سودی به کاربر
    می‌رسد؟}

  \پایان{بارهها}

  همچنین برای هر \واژه{یوزر.استوری}، چند \واژه{اکسپتنس.کریتریا} به
  زبان مشتری تعیین می‌گردد که اگر برآورده شوند، آن \واژه{یوزر.استوری}
  به درستی محقق شده است. در \واژه{بی.دی.دی.}، \واژه{اکسپتنس.کریتریا}
  در \واژه{قالب} «\واژه{سناریو}» توصیف می‌شود (ر.ک. به \رجوع{gwt}).

  اجرای تکراری فرآیند در \واژه{بی.دی.دی.} ضروری است. در هر مرحله از
  توصیف نیازمندی‌های، تا حدی پیش می‌رویم که بتوانیم \واژه{فیچر} جدیدی را
  پیاده‌سازی نماییم.}

\فقره{ \متن‌سیاه{قالب مشخص برای توصیف \واژه{یوزر.استوری} و
    \واژه{سناریو}:}
    
  در \واژه{بی.دی.دی.}، توصیف \واژه{فیچر}‌ها و \واژه{یوزر.استوری.ها} در
  قالبی مشخص صورت می‌گیرد و دلخواه نیست. این قالب‌ها، قسمتی از
  \واژه{یوبی} را تشکیل می‌دهند و ساختار \واژه{پلین.تکست} دارند
  (ر.ک. به \رجوع{gwt}). ساختار \واژه{پلین.تکست} باعث می‌شود هم محدودیتی
  برای تعریف \واژه{یوبی} بوجود نیاید، هم زبان برای تمام اعضای پروژه
  قابل فهم باشد، در عین اینکه قالبی دارد که توسعه‌دهنده‌ها بتوانند آن
  به‌راحتی آن را به \واژه{تست} تبدیل نمایند. }

  \فقره{ \متن‌سیاه{فرآیند \واژه{ای.تی.دی.دی.}:}
  
    \واژه{بی.دی.دی.} \واژه{تکنیک}‌های استفاده شده در فرآیند
    \واژه{ای.تی.دی.دی.} (ر.ک. به \رجوع{sec:atdd}) را به ارث
    می‌برد. همانطور که در شکل \رجوع{fig:bdd-flow} نمایش داده شده، مشابه
    \واژه{تکنیک} \واژه{ای.تی.دی.دی.}، پیش از آغاز به توسعه‌ی هر
    \واژه{فیچر} برنامه، \واژه{اکسپتنس.تست} برای آن نوشته می‌شود. این
    \واژه{اکسپتنس.تست}، مستقیماً از روی \واژه{اکسپتنس.کریتریا} به زبان
    مشترک با مشتری نوشته شده، تولید می‌شود. سپس فرآیند \واژه{تی.دی.دی.}
    (ر.ک. به \رجوع{sec:back:tdd}) ادامه پیدا می‌کند تا آنکه
    \واژه{اکسپتنس.تست} با موفقیت انجام شود. }
  
  \شروع{شکل}[tbp]
  \تنظیم‌ازوسط
  \درج‌تصویر[پهنا=0.7\پهنای‌سطر]{bdd-flow}
  \تر\موقتم{
    فرآیند توسعه‌ی \واژه{فیچر}‌ها در \واژه{بی.دی.دی.}
  }
  \شرح[\موقتم]{\موقتم~\مرجع{bddoverview}}
  \برچسب{fig:bdd-flow}
  \پایان{شکل}

  \فقره{ \متن‌سیاه{توصیف خوانا و \واژه{اجرایی}:}

    در \واژه{بی.دی.دی.}، با توجه به اینکه \واژه{اکسپتنس.تست} مستقیماً
    از روی \واژه{اکسپتنس.کریتریا} که شرایط پذیرش \واژه{یوزر.استوری} به
    زبان کاربر است نوشته می‌شود، توصیفی خوانا برای برنامه است که قابلیت
    اجرایی نیز دارد.

    در \واژه{بی.دی.دی.}، توصیه می‌شود متن برنامه نیز به‌صورت رفتار-محور
    نوشته شود؛ متن برنامه باید توصیف رفتار \واژه{کلاس}‌های باشد و نام
    \واژه{متد}‌ها و \واژه{کلاس}‌ها، از \واژه{یوبی} گرفته شده باشند. گرچه
    رعایت این موارد در متن برنامه در بعضی موارد ممکن نیست، توصیه شده
    تا جای ممکن، نام \واژه{کلاس}‌ها نمایان‌گر عنوان \واژه{یوزر.استوری}
    باشد و نام \واژه{متد}ها، نمایانگر \واژه{سناریو}‌ها باشند. }

  \فقره{ \متن‌سیاه{رعایت مشخصات رفتار-رانه در مرحله‌های مختلف:}

    مشخصات ذکر شده درباره‌ی \واژه{بی.دی.دی.}، درباره‌ی مرحله‌های مختلف از
    توسعه‌ی نرم‌افزار صادق‌اند. در مرحله‌ی برنامه‌ریزی اولیه، موارد
    \واژه{نتیجه.تجاری} مشخص می‌شوند. در مرحله‌ی تحلیل، \واژه{فیچر.ست}
    حاصل می‌شود و در مرحله‌ی پیاده‌سازی و \واژه{تست}، قواعد
    \واژه{تی.دی.دی.} و \واژه{اکسپتنس.تست} رعایت می‌شوند.

    با وجود اینکه \واژه{بی.دی.دی.} برای تمام فازها توصیه‌ی رعایت توسعه‌ی
    رفتار-رانه را دارد، ابزار موجود برای \واژه{بی.دی.دی.}، در تمام
    مراحل، برای آن قابلیت ندارند. برای مثال، در حال حاضر، هیچ کدام از
    ابزاری که به‌عنوان ابزار \واژه{بی.دی.دی.}  یا \واژه{فریم.ورک}
    \واژه{بی.دی.دی.}  موجود هستند، قابلیتی برای رعایت \واژه{بی.دی.دی.}
    در مرحله‌ی برنامه‌ریزی اولیه ارائه نمی‌کنند. }

\پایان{شمارش}
 
\زیرقسمت{\واژه{قالب} توصیف \واژه{یوزر.استوری.ها}‌ و \واژه{سناریو}‌ها}
\برچسب{gwt}

سلام

\شروع{definition} به گزاره‌ای که در صورت صحّت پیاده‌سازی برقرار باشد یک
\واژه{سناریو} می‌گوییم.  \پایان{definition} \شروع{definition}

در \واژه{بی.دی.دی.} یک \واژه{سناریو} به صورت زیر نمایش
داده می‌شود:

{\نقل
\متن‌سیاه{\واژه{given}} شرط $g$ برقرار باشد، \\
\متن‌سیاه{\واژه{when}} اتفاق $w$ رخ بدهد، \\
\متن‌سیاه{\واژه{then}} شرط $t$ باید برقرار باشد.
}

و یا به طور خلاصه:

{\نقل
\متن‌سیاه{\واژه{given}} $g$ \\
\متن‌سیاه{\واژه{when}} $w$ \\
\متن‌سیاه{\واژه{then}} $t$
}

این شیوه‌ی بیان سناریو \واژه{given-when-then} نام دارد.
\پایان{definition}

%%% Local Variables:
%%% mode: latex
%%% TeX-master: "../main"
%%% End:

