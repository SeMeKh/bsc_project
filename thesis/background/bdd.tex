\قسمت{تکنیک \واژه{بی.دی.دی.}}\برچسب{sec:back:bdd}

%%% Local Variables:
%%% mode: latex
%%% TeX-master: "../main"
%%% End:

\شروع{definition}
به گزاره‌ای که در صورت صحّت پیاده‌سازی برقرار باشد یک \واژه{سناریو} می‌گوییم.
\پایان{definition}

\شروع{definition}
\برچسب{gwt}
در \واژه{بی.دی.دی.} یک \واژه{سناریو} به صورت زیر نمایش داده می‌شود:

{\نقل
\متن‌سیاه{\واژه{given}} شرط $g$ برقرار باشد، \\
\متن‌سیاه{\واژه{when}} اتّفاق $w$ رخ بدهد، \\
\متن‌سیاه{\واژه{then}} شرط $t$ باید برقرار باشد.
}

و یا به طور خلاصه:

{\نقل
\متن‌سیاه{\واژه{given}} $g$ \\
\متن‌سیاه{\واژه{when}} $w$ \\
\متن‌سیاه{\واژه{then}} $t$
}

این شیوه‌ی بیان سناریو \واژه{given-when-then} نام دارد.
\پایان{definition}
