\زیرقسمت{\واژه{ای.تی.دی.دی.}}

\واژه{ای.تی.دی.دی.} \واژه{تکنیک}ی است که بر دو \واژه{پرکتیس} زیر
استوار است \مرجع{gartner2012atdd}:

\شروع{شمارش}

\فقره{ پیش از پیاده‌سازی هر قابلیت، اعضای تیم با مشتری درباره مثال‌های
  واقعی استفاده از آن قابلیت در عمل گفتگو و همکاری می‌کنند. سپس تیم
  توسعه، این مثال‌ها را به \واژه{اکسپتنس.تست} ترجمه می‌کند. }

\فقره{ این \واژه{اکسپتنس.تست}‌ها بخشی مهمی از توصیف دقیق آن قابلیت
  محسوب می‌شوند و پیاده‌سازی یک قابلیت زمانی انجام می‌شود که این
  \واژه{تست}‌ها در وضعیت \واژه{سبز} قرار گیرند. }

\پایان{شمارش}

همانطور که در نمودار فعالیت \رجوع{fig:atdd} نمایش داده شده، در این تکنیک
مشابه با روش \واژه{تی.دی.دی.}، قبل از پیاده‌سازی هر قابلیت،
\واژه{اکسپتنس.تست} برای آن نوشته می‌شود. سپس از اینکه فقط \واژه{تست}
اخیر در وضعیت \واژه{قرمز} قرار دارد اطمینان یافته و چرخه‌ی
\واژه{تی.دی.دی.} ادامه پیدا می‌کند تا اینکه \واژه{اکسپتنس.تست} مربوط به
قابلیت مورد نظر در وضعیت \واژه{سبز} قرار گیرد.


\شروع{شکل}[tbp]
\تنظیم‌ازوسط
\درج‌تصویر[پهنا=\پهنای‌سطر]{atdd}
\تر\موقتم{
  نمودار فعالیت مراحل \واژه{ای.تی.دی.دی.}
}
\شرح[\موقتم]{\موقتم~\مرجع{gartner2012atdd}}
\برچسب{fig:atdd}
\پایان{شکل}

%%% Local Variables:
%%% mode: latex
%%% TeX-master: "../main.tex"
%%% End:
