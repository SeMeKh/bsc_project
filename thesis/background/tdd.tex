\زیرقسمت{\واژه{تی.دی.دی.}}\برچسب{sec:back:tdd}

\واژه{تی.دی.دی.} \واژه{تکنیک}ی است که برنامه‌نویس را به استفاده از دو
\واژه{تکنیک} \واژه{تی.اف.دی.} و \واژه{ریفکتورینگ} با دو هدف اصلی زیر
ملزم می‌کند \مرجع{agiledata:tdd}:

\شروع{شمارش}

\فقره { با بهره‌گیری از \واژه{تی.اف.دی.}، برنامه‌نویس موظف است قبل از
  پیاده‌سازی هر واحد از کد، به نیازمندی آن واحد و طراحی آن واحد فکر
  کرده و در حقیقت با نوشتن \واژه{تست}، نیازمندی آن واحد را نیز توصیف
  می‌کند. }

\فقره { با توجه به اینکه برنامه‌نویس موظف است پس از افزودن هر واحد،
  \واژه{ریفکتورینگ} را به‌عنوان مرحله‌ای اجباری از فرآیند توسعه‌ی
  نرم‌افزار لحاظ کند، کیفیت طراحی نرم‌افزار همواره در بهترین حالت حفظ
  می‌شود. }

\پایان{شمارش}

همانطور که در نمودار حالت شکل \رجوع{fig:tdd-states} نمایش داده شده‌است،
در حالت \واژه{سبز}، فقط امکان \واژه{ریفکتورینگ} هست و برای تغییر متن
برنامه، باید ابتدا با نوشتن یک \واژه{تست} برای آن، به حالت
\واژه{قرمز} رفته و سپس امکان تغییر متن برنامه را داریم.


\شروع{شکل}[tbp]
\تنظیم‌ازوسط
\درج‌تصویر[پهنا=0.7\پهنای‌سطر]{tdd-states}
\شرح
[نمودار حالت وضعیت‌های \واژه{تست.سویت} و نحوه‌ی گذار بین آنها]
{نمودار حالت وضعیت‌های \واژه{تست.سویت} و نحوه‌ی گذار بین آنها با
  استفاده افزودن \واژه{تست}، \واژه{ریفکتورینگ} و تغییر متن برنامه \مرجع{agiledata:tdd}}
\برچسب{fig:tdd-states}
\پایان{شکل}


%%% Local Variables:
%%% mode: latex
%%% TeX-master: "../main"
%%% End:
