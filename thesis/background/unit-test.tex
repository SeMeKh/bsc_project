\زیرقسمت{\واژه{یونیت.تست}}

پایین‌ترین سطح از \واژه{هرم.تست}، \واژه{یونیت.تست} می‌باشد. هدف اصلی
\واژه{یونیت.تست}، این است که واحد کوچکی از نرم‌افزار را در نظر گرفته،
آن‌را از سایر واحد‌ها مجزا ساخته (مانند شکل \رجوع{fig:unit-test}) و
بررسی نماییم که این واحد وظیفه‌ی خود را مطابق انتظار انجام می‌دهد. هر
\واژه{یونیت.تست}، به‌صورت جداگانه قبل از \واژه{اینتگریشن} با سایر
واحدهای \واژه{ماژول} انجام می‌شود \مرجع{msdn:unit}.

با وجود هدف مشخص، حداقل ۲۴ تعریف گوناگون برای «\واژه{یونیت.تست}» وجود
دارد، زیرا تعریف‌ها بیش از حد خاص هستند. با وجود اختلاف‌ها، سه عنصر
مشترک میان تعریف‌ها وجود دارد \مرجع{fowler:unit}:

\شروع{شمارش}

\فقره{ \واژه{یونیت.تست}‌ تا حد ممکن در سطوح پایین معماری نرم‌افزار مورد
  استفاده قرار می‌گیرد و به قسمت کوچکی از متن برنامه متمرکز است. }

\فقره{ عموما توسط خود برنامه‌نویس‌ نوشته می‌شود و تیم جداگانه‌ای برای
  \واژه{یونیت.تست} تشکیل نمی‌شود. }

\فقره { \واژه{یونیت.تست} باید بسیار سریع‌تر از سایر سطح‌های \واژه{تست}
  و به‌صورت خودکار اجرا شود. }

\پایان{شمارش}

\شروع{شکل}[tbp]
\تنظیم‌ازوسط
\درج‌تصویر[پهنا=0.7\پهنای‌سطر]{unit-test}
\تر\موقتم{
  رابطه‌ی میان \واژه{یونیت.تست} و واحد‌های برنامه
}
\شرح[\موقتم]{\موقتم~\مرجع{fowler:unit}}
\برچسب{fig:unit-test}
\پایان{شکل}

این‌که منظور از یک «واحد» چه باشد، به نیازمندی و شرایط بستگی دارد و تیم
برای هر هدف، واحد را تعریف می‌کند. معمولا در برنامه‌نویسی شیء‌گرا، واحد
\واژه{کلاس} است \مرجع{fowler:unit}.

وضعیت اجرای یک \واژه{یونیت.تست} دو حالت \واژه{سبز} به معنی سازگاری
عملکرد واحد با انتظار و \واژه{قرمز} به معنی عدم سازگاری عملکرد با
انتظار دارد. \واژه{تست.سویت} شامل تمام \واژه{تست}‌هایی که برای واحدهای
مختلف نوشته‌شده است می‌باشد. وضعیت اجرای کل \واژه{تست.سویت}، در صورتی
\واژه{سبز} است که تک‌تک \واژه{یونیت.تست}‌ها در وضعیت \واژه{سبز} باشند
\مرجع{alience:unit}.

%%% Local Variables:
%%% mode: latex
%%% TeX-master: "../main"
%%% End:
