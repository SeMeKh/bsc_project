\زیرقسمت{\واژه{هرم.تست}}

\واژه{هرم.تست} اصطلاحی است که اولین بار توسط \نام{کوهن}{Mike Cohn}
برای دسته‌بندی سطوح مختلف \واژه{تست} در \واژه{متدولوژی}‌های چابک
معرفی شد. وجه تمایز سطوح در این دسته بندی، هزینه، زمان اجرا و تعداد
\واژه{تست}‌های مورد نیاز می‌باشد \مرجع{fowler:pyramid}.

\شروع{شکل}[tbp]
\تنظیم‌ازوسط
\درج‌تصویر[پهنا=0.7\پهنای‌سطر]{test-pyramid}
\تر\موقتم{
  نمودار سطوح \واژه{تست} در هرم \واژه{تست}
}
\شرح[\موقتم]{\موقتم~\مرجع{fowler:pyramid}}
\برچسب{fig:test-pyramid}
\پایان{شکل}

همانطور که در نمودار \رجوع{fig:test-pyramid} نشان داده شده‌است، در این
دسته‌بندی هرچه سطح \واژه{تست} بالاتر باشد، هزینه و زمان اجرای
\واژه{تست}‌ در آن سطح بیشتر، و تعداد \واژه{تست}‌های مورد نیاز آن سطح،
کمتر می‌شود. با این معیار، سه سطح زیر در این دسته‌بندی تعریف شده است
\مرجع{fowler:pyramid}:

\شروع{شمارش}

\فقره{\متن‌سیاه{سطح واحد}: پایین‌ترین سطح \واژه{هرم.تست}،
  \واژه{یونیت.تست} می‌باشد. هدف اصلی \واژه{یونیت.تست}، این است که واحد
  کوچکی از نرم‌افزار را در نظر گرفته، آن‌را از سایر واحد‌ها مجزا ساخته
  (مانند شکل \رجوع{fig:unit-test}) و بررسی نماییم که این واحد وظیفه‌ی
  خود را مطابق انتظار انجام دهد. هر \واژه{یونیت.تست}، به طور مستقل، و قبل
  از \واژه{اینتگریشن} با سایر واحدهای \واژه{ماژول}، اجرا می‌شود
  \مرجع{msdn:unit}.

  \واژه{یونیت.تست}‌ تا حد ممکن در سطوح پایین معماری نرم‌افزار مورد
  استفاده قرار می‌گیرد و به قسمت کوچکی از متن برنامه متمرکز است، عموما
  توسط خود برنامه‌نویس‌‌ها به تعداد زیاد نوشته می‌شود و باید بسیار سریع
  اجرا شود \مرجع{fowler:unit}.

  این‌که منظور از یک «واحد» چه باشد، به نیازمندی‌ها و شرایط پروژه بستگی
  دارد. معمولا در برنامه‌نویسی شیء‌گرا، \واژه{کلاس} به‌عنوان واحد انتخاب
  می‌شود \مرجع{fowler:unit}. }

\فقره {\متن‌سیاه{سطح سرویس}: هر نرم‌افزار، از چند سرویس تشکیل شده است که
  پس از دریافت ورودی از لایه‌ی \واژه{یو.آی.}، عملیات مورد انتظار سامانه
  را انجام داده و نتیجه را در اختیار \واژه{یو.آی.} قرار می‌دهند. منطق
  برنامه توسط سرویس‌ها پیاده‌سازی می‌شود. این سطح \واژه{تست}، عملکرد صحیح
  سرویس‌های \واژه{سیستم} را می‌آزماید \مرجع{cohn2010succeeding}.

  \واژه{تست} سرویس در لایه‌ی میانی \واژه{هرم.تست} قرار
  می‌گیرد. \واژه{تست} در این سطح، هزینه‌های \واژه{تست} سطح \واژه{یو.آی.}
  را ندارد، اما تیم را از \واژه{اینتگریشن} صحیح واحدها در کنار هم
  مطمئن می‌سازد \مرجع{cohn2010succeeding}.  ریز‌دانگی آزمون‌ها در این
  سطح تقریبا معادل سطح‌ \واژه{اینتگریشن.تست} از دسته‌بندی مدل \چر{V}
  می‌باشد. }

\فقره { \متن‌سیاه{سطح \واژه{یو.آی.}}:بالاترین سطح \واژه{هرم.تست}، سطح
  \واژه{یو.آی.} است. در این سطح، مطابقت عملکرد نهایی \واژه{سیستم} با
  نیازمندی‌هایی که مشتری مشخص می‌نماید، در محصول نهایی بررسی می‌شود. این
  سطح از \واژه{تست} در مقابل تغییرات بسیار شکننده است و هزینه‌ی زیادی
  صرف نوشتن و اجرای آن می‌شود \مرجع{cohn2010succeeding}. ریزدانگی
  \واژه{تست}‌ها در این سطح تقریبا معادل سطوح \واژه{سیستم.تست} و
  \واژه{اکسپتنس.تست} از دسته‌بندی مدل \چر{V} می‌باشد.  }

\پایان{شمارش}

\شروع{شکل}[tbp]
\تنظیم‌ازوسط
\درج‌تصویر[پهنا=0.7\پهنای‌سطر]{unit-test}
\تر\موقتم{
  رابطه‌ی میان \واژه{یونیت.تست} و واحد‌های برنامه
}
\شرح[\موقتم]{\موقتم~\مرجع{fowler:unit}}
\برچسب{fig:unit-test}
\پایان{شکل}


%%% Local Variables:
%%% mode: latex
%%% TeX-master: "../main"
%%% End:
