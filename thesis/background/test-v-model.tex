\زیرقسمت{دسته‌بندی مدل \چر{V}} یکی از دسته‌بندی‌های مشهور برای سطوحی
\واژه{تست}، دسته‌بندی مدل \چر{V} می‌باشد. این دسته‌بندی، پیش از پیدایش
\واژه{متدولوژی}‌های چابک، در \واژه{متدولوژی} «\واژه{وی.مدل}»، که با کمی
تغییر در \واژه{متدولوژی} \واژه{مدل.آبشاری} و تمرکز بیشتر بر \واژه{تست}
بوجود آمده، تعریف شده است \مرجع{singh2011software}.

وجه تمایز سطوح در این دسته‌بندی، مراحل تولید نرم‌افزار و تقدم زمانی
فازهای تحلیل، طراحی و پیاده‌سازی می‌باشد. همانطور که در شکل
\رجوع{fig:v-model} نشان داده شده، پس از هر فاز، \واژه{تست}ی برای
\واژه{آرتیفکت} آن فاز نوشته می‌شود \مرجع{singh2011software}.

\شروع{شکل}[tbp]
\تنظیم‌ازوسط
\درج‌تصویر[پهنا=0.7\پهنای‌سطر]{v-model}
\تر\موقتم{
  نمودار سطوحی آزمون در \واژه{وی.مدل}
}
\شرح[\موقتم]{\موقتم~\مرجع{singh2011software}}
\برچسب{fig:v-model}
\پایان{شکل}

در این دسته‌بندی چهار سطح «\واژه{اکسپتنس.تست}»، «\واژه{سیستم.تست}»،
«\واژه{اینتگریشن.تست}» و «\واژه{یونیت.تست}» به ترتیب برای مراحل تحلیل،
طراحی سطح بالا، طراحی سطح پایین و پیاده‌سازی معرفی شده‌اند.

%%% Local Variables:
%%% mode: latex
%%% TeX-master: "../main"
%%% End:
