\فصل{نتیجه‌گیری و کارهای آتی}
\برچسب{chap:future}

\قسمت{جمع‌بندی}
در طول این پروژه مجموعاً فعّالیت‌های زیر صورت گرفت:
\شروع{enumerate}
\فقره بررسی کارهای پیشین در زمینه‌ی توسعه‌ی \واژه{تست}
\فقره ارائه‌ی معماری پیشنهادی با ایده گرفتن از نقاط قوّت و ضعف مشهود در روش‌های پیشین
\فقره مقایسه‌ی این معماری با روش‌های قبلی
\فقره پیاده‌سازی یک \واژه{فریم.ورک} \واژه{تست} مبتنی بر این معماری
\فقره پیاده‌سازی یک نرم‌افزار نمونه و تألیف \واژه{تست} برای آن توسّط \واژه{فریم.ورک} پیشنهادی
\پایان{enumerate}

نهایتاً برآورد مؤلّفان این بود که استفاده از معماری پیشنهادی، همان‌طور که انتظار می‌رفت، کمک شایانی به ساده‌سازی انجام \واژه{تست} نرم‌افزار نمونه کرد. انتظار می‌رود که استفاده از این معماری در \واژه{تست} نرم‌افزارهای بزرگتر، مؤثّرتر نیز واقع شود.

\قسمت{کارهای آتی}
در طول انجام پروژه، ایده‌های فراوانی مبتنی بر معماری پیشنهادی به ذهنمان رسید که در راستای خارج نشدن از حوزه‌ی این پروژه وارد آن‌ها نشدیم، امّا بررسی آن‌ها خالی از لطف نخواهد بود:

\شروع{itemize}
\فقره حذف و هرس \واژه{سناریو}ها پس از بررسی صحّت آن‌ها به دفعات کافی
\فقره تعریف \واژه{متریک}های جدید سنجش کیفیّت نرم‌افزار بر حسب \واژه{سناریو}ها
\فقره استفاده از \واژه{فریم.ورک} \واژه{تست} پیاده‌شده در یک پروژه‌ی متن‌باز و بزرگ‌تر، جهت ارزیابی دقیق‌تر این روش
\فقره پیاده‌سازی \واژه{اکتور} به شیوه‌های مختلف و بررسی میزان اثرگذاری آن‌ها
\پایان{itemize}
