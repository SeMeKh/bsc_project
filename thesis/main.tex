\documentclass[BScThesis,twoside]{sharifthesis}

\usepackage{amssymb}
\usepackage{mathrsfs}
\usepackage{algorithmicx}
\usepackage{graphicx}
\usepackage{multirow}
\usepackage{comment}
\usepackage[style=ieee,backend=biber]{biblatex}
\newcommand{\bibliographytitle}{\rl{کتاب‌نامه}}
\usepackage{paralist}
\usepackage{textcomp}
\usepackage{ctable}% for tables (it provides table-notes and imports booktabs package too)

\newcounter{tablerow}
\renewcommand{\arraystretch}{1.5}
\usepackage{cleveref}
\newcommand{\crefrangeconjunction}{--}
\crefformat{tablerow}{#2#1#3}
\crefmultiformat{tablerow}{#2#1#3}%
{ و~#2#1#3}%
{, #2#1#3}%
{ و~#2#1#3}
\crefformat{section}{#2#1#3}
\crefmultiformat{section}{#2#1#3}%
{ و~#2#1#3}%
{, #2#1#3}%
{ و~#2#1#3}



\newcommand{\URL}%
{یو.آر.ال.}
\newcommand{\postgresql}%
{پُستگِرِس.کیو.اِل.}
\newcommand{\booktabs}%
{بوک‌تَبز}
\نوواژه[تست.ن.ا.]{آزمون نرم‌افزار}{Software Test}
\نوواژه[تست]{آزمون}{Test}
\نوواژه[یونیت.تست]{آزمون واحد}{Unit Test}
\نوواژه[اکسپتنس.تست]{آزمون پذیرش}{Acceptance Test}
\نوواژه[اکسپتنس.کریتریا]{شرط پذیرش}{Acceptance Criteria}
\نوواژه[هرم.تست]{هرم آزمون}{Test Pyramid}
\نوواژه[تست.سویت]{مجموعه‌ی آزمون‌ها}{Test Suite}
\نوواژه[سیستم.تست]{آزمون سامانه}{System Test}
\نوواژه[سیستم]{سامانه}{System}
\نوواژه[ماژول]{مدول}{Module}
\نوواژه{کلاس}{Class}
\نوواژه{قالب}{Template}
\نوواژه[تست.کیس]{مورد آزمون}{Test Case}
\نوواژه[وی.مدل]{مدل چرخه عمر \چر{V}-شکل}{V-Shaped Software Lifecycle Model}
\نوواژه[مدل.وی]{مدل وی}{V Model}
\نوواژه[یو.آی.]{رابط کاربری}{User Interface}
\نوواژه[اینترفیس]{واسط}{Interface}
\نوواژه[بیزنس.دامین]{حوزه تجاری}{Business Domain}
\نوواژه[بیزنس.ولیو]{ارزش تجاری}{Business Value}
\نوواژه[ریفکتورینگ]{فاکتوربندی مجدد}{Refactoring}
\نوواژه[بیانیه.اجایل]{بیانیه ی توسعه ی چابک نرم افزار}{Manifesto for
Agile Software Development}
\نوواژه[اس.دی.ام.]{متدولوژی توسعه ی نرم افزار}{Software Development Methodology}
\نوواژه{ذینفع}{Stakeholder}
\نوواژه{فعالیت}{Activity}
\نوواژه{قرمز}{Fail}
\نوواژه{سبز}{Success}
\نوواژه[پی.پی.]{برنامه‌نویسی دونفره}{Pair Programming}
\نوواژه[سی.آی.]{یکپارچه‌سازی مداوم}{Continuous Integration}
\نوواژه[اینتگریشن]{یکپارچه‌سازی}{Integration}
\نوواژه[اینتگریشن.تست]{آزمون یکپارچه‌سازی}{Integration Test}
\نوواژه[فانکشنال]{کارکردی}{Functional}
\نوواژه[زبان.مدلینگ]{زبان مدل‌سازی}{Modeling Language}
\نوواژه[سافتویر.اینتنسیو]{نرم‌افزار-متمرکز}{Software-intensive}
\نوواژه[دیباگ]{رفع‌باگ}{Debug}
\نوواژه[آرتیفکت]{ساخته}{Artifact}
\نوواژه[سینتکس]{نحوی}{Syntax}
\نوواژه[سمنتیکس]{معنایی}{Semantics}
\نوواژه[وفق.پذیری]{وفق‌پذیری}{Adaptability}
\نوواژه[باگ]{باگ}{Bug}
\نوواژه[کانونشن]{عرف}{Convention}
\نوواژه[متدولوژی]{متدولوژی}{Methodology}
\نوواژه[تکنیک]{تکنیک}{Technique}
\نوواژه[پرکتیس]{تجربه}{Practice}
\نوواژه[اتکا.پذیری]{اتکاپذیری}{Reliability}
\نوواژه[ایکس.پی.]{اکس.پی.}{Extreme Programming}
\نوواژه[کیفیت.ن.ا.]{کیفیت نرم‌افزار}{Software Quality}
\نوواژه[اعتبار.سنجی]{اعتبار سنجی}{Validation}
\نوواژه{وارسی}{Verification}
\نوواژه{تحلیل}{Analysis}
\نوواژه{تکراری}{Iterative}
\نوواژه{تکرار}{Iteration}
\نوواژه{طراحی}{Design}
\نوواژه{توصیف}{Specification}
\نوواژه[اس.دی.پی.]{فرآیند توسعه‌ی نرم‌افزار}{Software Development
Process}
\نوواژه{تفکیک}{Decomposition}
\نوواژه[اسکوپ]{محدوده}{Scope}
\نوواژه[ای.اس.دی.]{توسعه‌ی چابک نرم‌افزار}{Agile Software Development}
\نوواژه{نگهداری}{Maintenance}
\نوواژه[پیاده.سازی]{پیاده‌سازی}{Implementation}
\نوواژه[مدل.آبشاری]{مدل آبشاری}{Waterfall Model}
\نوواژه[ای.ام.دی.دی.]{توسعه‌ی چابک مدل-رانه}{Agile Model Driven Development}
\نوواژه[بی.دی.دی.]{توسعه‌ی رفتار-رانه}{Behavior Driven Development}
\نوواژه{رفتار}{Behavior}
\نوواژه[فیچر]{ویژگی}{Feature}
\نوواژه[فیچر.ست]{مجموعه‌ی ویژگی}{Feature Set}
\نوواژه[نتیجه.تجاری]{نتیجه تجاری}{Business Outcome}
\نوواژه[تی.دی.دی.]{توسعه‌ی آزمون-رانه}{Test Driven Development}
\نوواژه[تی.اف.دی.]{توسعه‌ی اول-آزمون}{Test First Development}
\نوواژه[ای.تی.دی.دی.]{توسعه‌ی آزمون پذیرش-رانه}{Acceptance Test Driven
Development}
\نوواژه[پری.کاندیشن]{پیش-شرط}{Pre-condition}
\نوواژه[پست.کاندیشن]{پسا-شرط}{Post-condition}
\نوواژه[دان]{انجام‌شده}{Done}
\نوواژه[یوزر.استوری]{داستان کاربر}{User Story}
\نوواژه[یوزر.استوری.ها]{داستان‌های کاربر}{User Stories}
\نوواژه[دامین.اکسپرت]{خبره حوزه}{Domain Expert}
\نوواژه[پلین.تکست]{متن ساده}{Plain Text}

\نوواژه{سناریو}{Scenario}
\نوواژه[given]{اگر}{Given}
\نوواژه[when]{وقتی}{When}
\نوواژه[then]{آن‌گاه}{Then}
\نوواژه[given-when-then]{\واژه{given}-\واژه{when}-\واژه{then}}{Given-When-Then}

\نوواژه{متریک}{Metric}
\نوواژه[کلاسیک]{کلاسیک}{Classic}
\نوواژه[اکتور]{تعامل‌گر}{Actor}
\نوواژه[coupling]{وابستگی}{Coupling}
\نوواژه[component]{مؤلفه}{Component}
\نوواژه[reuse]{استفاده‌ی مجدد از کد}{Code Reuse}
\نوواژه[پایتون]{زبان برنامه‌نویسی پایتون}{Python Programming Language}
\نوواژه[فریم.ورک]{چارچوب}{Framework}
\نوواژه[decorator]{دکوراتور}{Decorator}
\نوواژه[یوبی]{زبان مشترک}{Ubiquitous Language}

\نوواژه[inconsistency]{ناهم‌خوانی}{Inconsistency}
\نوواژه[defect]{نقص}{Defect}
\نوواژه[trigger]{محرک }{Trigger}

\نوواژه[insanity]{اینسنیتی}{Insanity}
\نوواژه[انتیتی]{موجودیت}{Entity}
\نوواژه{جنگو}{Django}
\نوواژه[کراد]{سی.آر.یو.دی.}{Create Read Update Delete}
\نوواژه[سرور]{کارگزار}{Server}
\نوواژه[ادد]{افزودن}{Add}



\newcommand{\myrotate}[3][]{\rotatebox{90}{\parbox[c][#1]{#2}{\centering\arraybackslash\rl{#3}}}}
\newcommand{\multilinescell}[2][c]{\begin{tabular}[#1]{@{}c@{}}#2\end{tabular}}
\newcommand{\twolinescell}[3][c]{\multilinescell[#1]{#2\\#3}}
\newcommand{\itemrl}[1][]{\item[\rl{#1}]}
\eqcommand{چرخش}{myrotate}
\eqcommand{سلولچندخطی}{multilinescell}
\eqcommand{سلول‌دوخطی}{twolinescell}
\eqcommand{فقره‌راست}{itemrl}



\newcommand{\Eqn}[1]%
{فرمول~(#1) }
\newcommand{\Eqns}[1]%
{فرمول‌های~(#1) }

\eqcommand{فرمول}{Eqn}
\eqcommand{فرمولهای}{Eqns}

\فرمان‌نو{\نگا}{ن.بـ.}
\فرمان‌نو{\نگاص}[1]{\نگا{} صفحه‌ی~\رجوع‌صفحه{#1}}
\فرمان‌نو{\آیم}[1][$i$]%
{#1اُم}

\newcolumntype{C}[1]{>{\centering\arraybackslash}m{#1}}

\فرمان‌نو{\نامک}[1]{%
\fcolorbox{red}{yellow}{\begin{minipage}{\textwidth}#1\end{minipage}}}

\newcommand{\tobewritten}[1]{\نامک{%
این جعبه، باید با متن متناظر با \چر{#1} جای‌گزین گردد.

برای دریافتن معنی \چر{#1} به پرونده‌ی \چر{TODO} نگاه کنید.
}}

\graphicspath{{img/}}% tell tex engine address of folder containing your pictures

\addbibresource{resources/resources.bib}


\newcommand{\faKeywords}{تست، رفتار رانه، سناریو، مستقل‌سازی
}
\eqcommand{واژه‌های‌کلیدی}{faKeywords}
\آرم{\درج‌تصویر[scale=.7]{logo}}
\تاریخ{شهریور ۱۳۹۴}
%در دستور زیر، از \\ استفاده نکنید
\عنوان{مستقل‌سازی سناریو از تست رفتار رانه}
%اگر عنوان طولانی بوده (و در عنوان انگلیسی از دو خظ استفاده شده) باید دو خط زیر از کامنت خارج و دو خط عنوان توسط آن‌ها تعریف گردد.
%\عنوانخطیک{خط نخست عنوان طولانی}
%\عنوانخطدو{و ادامه‌ی آن در خط دوم}
\نویسنده{سید مهران خلدی، محمد حسین سخاوت}
\دانشگاه{{\نستعلیق\درشت‌تر دانشگاه صنعتی شریف %
\\[0.6cm]}
دانشکده‌ی مهندسی کامپیوتر}
\دانشگاه‌عادی{دانشگاه صنعتی شریف\\
دانشکده‌ی مهندسی کامپیوتر}
\موضوع{گرایش نرم‌افزار}
\استادراهنما{دکتر سید حسن میریان}
%اگر استاد مشاور ندارید، خط زیر را comment کنید
%همچنین، فراموش نکنید در آخر این پرونده، اطلاعات انگلیسی معادل این دستورها را هم پر کنید
%\استادمشاور{دکتر <نام استاد مشاور>} 

\newcommand{\efootnote}[1]{\footnote{\lr{#1}}}
\newcommand{\ecfootnote}[1]{}


% ================ Correct hyphenations ================
\hyphenation{test}


\makeglossaries
%\includeonly{related_works/related_works}
%\includeonly{evaluation/evaluation}

% ===== DEPRACATED AREA =====
% Following commands are provided to make older documents compilable.
% These commands are depracated and should not be used in new documents.
\newcommand{\ترجمه‌ج}[2]
{\ترجمه[#1‌ها]{#1}{#2}}
\newcommand{\ترجمه‌جمع}[3]
{\ترجمه[#3]{#1}{#2}}
\newcommand{\برگردان}[3]
{\ترجمه{#1}{#3}\زیرنویس{#2}}
\eqcommand{اسم}
{نام}
% ===== END OF DEPRACATED AREA =====

\شروع{نوشتار}


\newcommand{\StartDocument}{\frontmatter \baselineskip1.2\baselineskip \pagestyle{empty} \null \vfill
\شروع{وسط‌چین}
{\نستعلیق‌درشت بسم اللّه الرّحمن الرّحیم}
\پایان{وسط‌چین}
\vfill}

%the initial title is supposed to be printed on the cover.
%for non final version, you can leave following commands as is to create only one title page (printed on paper)
%for final version you need to swap folowing commented/uncommented makethesistitle commands to achieve this order: title on the cover THEN in the name of god page THEN another title page but this time printed on paper
\makethesistitle
\StartDocument
%\makethesistitle

\setlength{\baselineskip}{0.9cm}
\begin{comment}
\فصل*{پیش‌گفتار}
\thispagestyle{pagenumberonlyPagestyle}
پیشگفتار اختیاری است. در صورت تمایل به نگارش پیش‌گفتار، محیط کامنت که آن را دربرگرفته باید حذف شود.
\end{comment}

\شروع{قدردانی}
از جناب آقای دکتر مصطفی مهدیه به خاطر کمک‌ها، مشورت‌ها و پیگیری‌های
عالمانه و دلسوزانه‌ی ایشان قدردانی می‌نماییم.

\پایان{قدردانی}
% END OF COMMENT FOR PhD Proposal.

\شروع{چکیده}{\واژه‌های‌کلیدی}
\واژه{بی.دی.دی.} یکی از \واژه{تکنیک}‌های جدید برای \واژه{ای.اس.دی.}
است که استفاده از آن در صنعت نرم‌افزار رواج دارد. در این پایان‌نامه ابتدا
به بررسی دقیق \واژه{بی.دی.دی.} و مقایسه‌ی آن با \واژه{تکنیک}‌های مشابه
می‌پردازیم.

در ادامه یک \واژه{فریم.ورک} \واژه{تست} جهت استفاده در \واژه{تکنیک} \واژه{بی.دی.دی.} ارائه می‌کنیم، که با اتکا بر \واژه{decoupling} ساختار \واژه{تست}، توسعه‌دهنده را در افزایش کیفیت \واژه{تست}‌های تألیفی و کاهش هزینه‌ی نگهداری آن‌ها یاری می‌کند.

نهایتاً و به عنوان نمونه، با استفاده از این \واژه{فریم.ورک} \واژه{تست} برای یک سامانه‌ی کوچک نرم‌افزاری، \واژه{تست}‌هایی تألیف می‌کنیم.

\پایان{چکیده}


\setlength{\baselineskip}{0.9cm}
\pagenumbering{tartibi}\tableofcontents\listoftables\listoffigures
%list of abbreviations may be added here...


\PrepareForMainContent
%	In the name of GOD
\doublespace

\فصل{مقدمه}
\برچسب{chap:intro}

فرآیند \واژه{تست.ن.ا.} از اولین پیدایش تا کنون با رویکرد‌ها و کاربردهای
متنوعی اجرا شده‌است. در این پایان‌نامه بر رویکرد بهره‌گیری از توصیف
نرم‌افزار به زبان مشتری در \واژه{تست} تمرکز شده و معماری‌ جدیدی برای
\واژه{تست} با \واژه{تکنیک} \واژه{بی.دی.دی.}، که بر اساس این رویکرد
ارائه شده، پیشنهاد می‌شود.

گرچه ایده‌ی مطرح شده در معماری پیشنهادی منحصر به \واژه{تکنیک}
\واژه{بی.دی.دی.} نیست و با سایر \واژه{تکنیک}‌ها و \واژه{متدولوژی}‌ها نیز
قابل انطباق است، حوزه‌ی بررسی‌این پایان‌نامه به \واژه{تکنیک}
\واژه{بی.دی.دی.} محدود شده است.

\قسمت{تعریف موضوع پروژه}

\قسمت{اهمیت موضوع پروژه}

\قسمت{انگیزه انجام پروژه}
ایده‌ی این پروژه از همکاری مؤلّفان در یک پروژه‌ی شرکتی شکل گرفت. به عنوان توسعه‌دهندگان سیستم مذکور، همیشه حس می‌کردیم که بخش قابل توجّهی از کدهایی که برای \واژه{تست} نرم‌افزار می‌نویسیم شباهت‌های زیادی به یک‌دیگر دارند.
از سوی دیگر، به دلیل تغییر نیازمندی‌های سیستم، همیشه لازم بود تا تست‌ها به‌روزرسانی شوند و این تغییرات . تمام این مشکلات منجر می‌شد تا کیفیّت تست‌ها در طول زمان کاهش یافته و اثرگذاری خود را از دست دهند.

بنابراین سعی کردیم الگوهای مشترکی بین تست‌ها پیدا کنیم و با \واژه{ریفکتورینگ} مکرّر تست‌هایی که داشتیم به ساختاری رسیدیم که بسیاری از مشکلات فوق را کم‌رنگ کرده بود. تلاش کردیم تا ساختار نهایی را تعمیم دهیم تا قابل استفاده در پروژه‌های دیگر نیز باشد. نتیجه معماری‌ای شد که در این پروژه ارائه می‌کنیم.

طبق بررسی‌هایی که کردیم اگر چه ایده‌های مشابهی وجود داشت، ...


\قسمت{ساختار پایان‌نامه}


%%% Local Variables:
%%% mode: latex
%%% TeX-master: "../main"
%%% End:


\فصل{زمینه}
\برچسب{chap:back}


\قسمت{تاریخچه}\برچسب{sec:back:history}

روند تکامل رویکرد رایج در \واژه{تست.ن.ا.}، از سال ۱۹۵۷ تا سال ۲۰۰۰ به
پنج دوره‌ی \نام{رفع‌اشکال-محور}{Debugging-oriented}،
\نام{اثبات-محور}{Demonstration-oriented}،
\نام{تخریب-محور}{Destruction-oriented}،
\نام{ارزیابی-محور}{Evaluation-oriented} و
\نام{پیشگیری-محور}{Prevention-oriented} تقسیم‌بندی
‌می‌شود\مرجع{laycock1993theory}. برخی فعالیت‌های بارز این دوره‌ها به شرح
زیر است: \شروع{شمارش}

\فقره{ تا سال ۱۹۵۷، برنامه نویس‌ها پس از نوشتن برنامه، آن را اجرا نموده
تا از کارکرد صحیح آن اطمینان یابند و اگر \واژه{باگ}ی در اجرا پیدا
می‌شد، آن را پیدا کرده و رفع می‌نمودند. این فرآیند ادامه پیدا می‌کرد تا
زمانی که احساس کنند \واژه{باگ}ی باقی نمانده است.  }

\فقره{ در سال ۱۹۵۷، \نام{بیکر}{Charles L. Baker} برای اولین بار
\واژه{تست.ن.ا.} را، به‌عنوان روشی برای «اثبات عملکرد صحیح» برنامه، از
\واژه{دیباگ} تمیز داد\مرجع{baker1957review}.  }

\فقره{ \نام{دایسترا}{Edsger Dijkstra} در سال ۱۹۶۹ متذکر شد کاربرد تست،
«یافتن \واژه{باگ}ها» است، نه اثبات عملکرد صحیح
برنامه\مرجع{testinghistory}.  در سال ۱۹۸۳، راهنمایی برای
\واژه{اعتبار.سنجی}، \واژه{وارسی} و \واژه{تست.ن.ا.} منتشر شد
\مرجع{adrion1982validation} که «تشخیص \واژه{باگ}‌های تحلیل و طراحی» را،
علاوه بر تشخیص \واژه{باگ}‌های پیاده‌سازی، با \واژه{تست.ن.ا.} میسر می‌سازد
\مرجع{luo2001software}.  }

\فقره{ در سال ۱۹۸۷، معیار \واژه{اتکا.پذیری} به‌عنوان عاملی کلیدی در
«اندازه‌گیری \واژه{کیفیت.ن.ا.}» معرفی شد \مرجع{musa1987software}.  }

\فقره{ در سال ۱۹۹۰، \واژه{تست.ن.ا.} به عنوان یکی از موثرترین عوامل
«پیشگیری از \واژه{باگ}»‌ معرفی شد \مرجع{beizer20672software}.  }

\پایان{شمارش}

پس از رویکردهای مذکور در اواخر دهه ۱۹۹۰، با ظهور \واژه{متدولوژی}‌های
جدید مانند \واژه{ایکس.پی.}، که در دسته‌بندی \واژه{متدولوژی}‌های
\واژه{ای.اس.دی.} قرار می‌گیرد، رویکردهای جدیدی در \واژه{تست.ن.ا.} ایجاد
شد.

در \واژه{متدولوژی}‌های سنتی با \واژه{مدل.آبشاری}، \واژه{تست} یکی از
فاز‌های \واژه{اس.دی.پی.} بود که فقط یک بار انجام می‌شد و پیش‌نیاز آن،
اتمام فازهای \واژه{تحلیل}، \واژه{طراحی} و \واژه{پیاده.سازی} بود. در
\واژه{متدولوژی}‌های آبشاری، هر کدام از فازهای \واژه{تحلیل}،
\واژه{طراحی}، \واژه{پیاده.سازی} و \واژه{تست} به‌صورت مجزا اجرا می‌شد
\مرجع{agileVsWaterfall} و \واژه{تست} نقش موثری در فازهای تحلیل، طراحی
و پیاده‌سازی نداشت.

اما در \واژه{متدولوژی}‌های چابک، مراحل تحلیل، طراحی، پیاده سازی و
\واژه{تست} به‌صورت \واژه{تکراری} اجرا می‌شود. در هر \واژه{تکرار}،
نیازمندی کامل نرم‌افزار مشخص نیست و در \واژه{تکرار} بعد، ممکن است تغییر
کند. نقش \واژه{تست} در «پاسخگویی به تغییرات»، که یکی از چهار ارزش
\واژه{بیانیه.اجایل} است \مرجع{manifesto}، بسیار کلیدی است. در
\واژه{تست} چابک، رویکرد «\نام{دستیاری در کیفیت}{Quality Assistance}»
بر رویکرد «\نام{اطمینان از کیفیت}{Quality Assurance}» غلبه می‌کند\مرجع{ghahrai}.

\نام{بک}{Kent Beck} در سال ۲۰۰۲، با معرفی \واژه{تکنیک}
\واژه{تی.دی.دی.}، \واژه{تست} را به‌عنوان «محرک توسعه‌ی نرم‌افزار» معرفی
می‌کند و نوشتن \واژه{تست} را پیش‌نیاز \واژه{پیاده.سازی} می‌داند. وی
«بهبود طراحی نرم‌افزار» را از نتایج بکارگیری این \واژه{تکنیک} می‌داند
\مرجع{beck2003test}. همچنین با بهره‌گیری از این نگرش در \واژه{متدولوژی}
\واژه{ای.ام.دی.دی.}، از \واژه{تست.ن.ا.} به‌عنوان «\واژه{توصیف}
نرم‌افزار» بهره گرفته می‌شود \مرجع{agileModeling:specByExample}.


%%% Local Variables:
%%% mode: latex
%%% TeX-master: "../main"
%%% End:


\قسمت{ادبیات}\برچسب{sec:back:vocabulary}

اصطلاح‌های مرتبط با توسعه‌ی نرم‌افزار، ابهام‌ها و برداشت‌های گوناگون دارند
برای مثال در \مرجع{gartner2012atdd}، هر چهار اصطلاح
«\واژه{ای.تی.دی.دی.}»، «\واژه{بی.دی.دی.}»، «\نام{توصیف با
  مثال}{Specification by Example}»، «\نام{آزمون پذیرش چابک}{Agile
  Acceptance Testing}» و «\نام{تست داستان کاربر}{User Story Testing}»،
هم‌معنی تلقی شده‌اند. برای پیشگیری از ابهام و شفاف‌سازی اصطلاح‌های استفاده
شده در این پایان‌نامه، در این قسمت توضیحی اجمالی و بدون ارزیابی جنبه‌های
مختلف، برای هر اصطلاح ارائه شده است.

\input{background/test-case}

\input{background/test-v-model}

\input{background/test-pyramid}

\input{background/methodology}

\input{background/tfd}

\input{background/tdd}

\input{background/atdd}

%%% Local Variables:
%%% mode: latex
%%% TeX-master: "../main"
%%% End:


\قسمت{\واژه{تکنیک} \واژه{بی.دی.دی.}}\برچسب{sec:back:bdd}

\واژه{بی.دی.دی.} یکی از تکنیک‌های توسعه‌ی چابک نرم‌افزار است که در سال
۲۰۰۶ توسط \نام{نورث}{Dan North} برای پاسخ‌گویی به ابهام‌ها و مشکل‌هایی که
در \واژه{تی.دی.دی.} بوجود آمده‌بود، معرفی
شد. \مرجع{north2006introducing}.  تاکید \واژه{بی.دی.دی.} بر بهینه‌سازی
ارتباط میان نقش‌های برنامه‌نویس، آزمون‌گر و \واژه{دامین.اکسپرت}‌ می‌باشد و
برای تیمی که تمام اعضای آن برنامه‌نویس باشند، \واژه{بی.دی.دی.} سودی
نسبت به \واژه{تی.دی.دی.} ندارد \مرجع{north2012}.

این تکنیک، که اخیرا هم در عمل و هم در پژوهش بسیار متداول شده است، حاصل
ترکیب و بهبود \واژه{پرکتیس}‌های معرفی شده در تکنیک‌های \واژه{تی.دی.دی.}
و \واژه{ای.تی.دی.دی.} و علاوه‌بر آن در نظر گرفتن اصول زیر است
\مرجع{alience:bdd}:

\شروع{بارهها}

\باره{ لازم است هدف هر \واژه{یوزر.استوری} و سودی که برای مشتری دارد،
  مشخص باشد.}

\باره{ با نگرش «\نام{از بیرون به درون}{From the outside in}»، فقط
  رفتارهایی از \واژه{سیستم} پیاده‌سازی می‌شوند که بیشترین سود را به
  مشتری می‌رسانند. با داشتن این نگرش، اتلاف هزینه کمینه می‌شود. }

\باره{ رفتارهای مورد انتظار از سامانه، بین \واژه{دامین.اکسپرت}،
  توسعه‌دهنده‌ی \واژه{سیستم} و آزمون‌گر به یک زبان مشترک توصیف می‌شوند. در نتیجه
  ارتباط ضروری میان این نقش‌ها بهبود می‌یابد. }

\باره { این اصول، در تمام سطوح انتزاعی توصیف برنامه، تا پایین‌ترین
  سطح که پیاده‌سازی یک واحد است، رعایت می‌شوند. }

\پایان{بارهها}

\زیرقسمت{مشخصات \واژه{بی.دی.دی.}}  در حال حاضر، \واژه{تکنیک}
\واژه{بی.دی.دی.}، هنوز در حال توسعه است و تعریف روشنی از
\واژه{بی.دی.دی.} که بر آن اجماع باشد، وجود ندارد. با توجه به اینکه
فرآیند \واژه{بی.دی.دی.}، از ابتدا به‌صورت انتزاعی معرفی شده و جزییاتی
برای آن ارائه نشده، مشخصات \واژه{بی.دی.دی.} مبهم و پراکنده
هستند. همچنین چارچوب‌ها و ابزارهایی که برای \واژه{بی.دی.دی.} ارائه
شده‌اند، بیشتر به بخش «پیاده‌سازی آزمون» از فرآیند \واژه{بی.دی.دی.}
تمرکز دارند؛ در صورتی که \واژه{بی.دی.دی.} بر حوزه‌ی گسترده‌تری از
\واژه{اس.دی.پی.} تاثیر دارد \مرجع{solis2011study}.

در مقاله‌ی \مرجع{solis2011study}، ۶ مورد از مشخصات اصلی
\واژه{بی.دی.دی.} که بر کل \واژه{اس.دی.پی.}، نه فقط قسمت «پیاده‌سازی
آزمون»، تاثیر گذار اند، ارائه شده است. در اینجا به صورت خلاصه به آن‌ها
اشاره می‌کنیم:

\شروع{شمارش}

\فقره{ \متن‌سیاه{زبان \واژه{یوبی}}:

  مفهوم «زبان \واژه{یوبی}»، هسته‌ی \واژه{بی.دی.دی.} را تشکیل
  می‌دهد. ساختار \واژه{یوبی} برآمده از \واژه{بیزنس.دامین} می‌باشد و
  ابهام را از مکالمه‌ی بین مشتری و تیم توسعه کاهش می‌دهد. همچنین یک
  واژه‌نامه در ابتدای پروژه ایجاد شده، اکثر واژه‌های آن در مرحله‌ی تحلیل
  افزوده می‌شوند و در مراحل بعد امکان گسترش دارد.

  هر \واژه{بیزنس.دامین}، زبان \واژه{یوبی} خاص خود را لازم دارد.
  \واژه{بی.دی.دی.} یک \واژه{قالب} ساده برای مرحله‌ی تحلیل ارائه نموده است که
  مستقل از \واژه{بیزنس.دامین} است و از آن در زبان \واژه{یوبی} بهره‌گیری
  می‌شود. این \واژه{قالب} در بخش \رجوع{gwt} شرح داده شده است.  }

\فقره{ \متن‌سیاه{فرآیند \واژه{تفکیک} \واژه{تکراری}}:

  توقع مشتری از یک پروژه‌ی نرم‌افزاری، ایجاد \واژه{بیزنس.ولیو}
  می‌باشد. معمولاً تشخیص و روشن‌سازی \واژه{بیزنس.ولیو}، دشوار است. به
  همین دلیل، \واژه{بیزنس.ولیو} به اجزای ملموس‌تر تفکیک می‌شود. در شکل
  \رجوع{fig:bdd-conceptual} رابطه‌ی اجزای استفاده شده در این
  \واژه{تفکیک} نسبت به هم نمایش داده شده است.

  \شروع{شکل}[tbp]
  \تنظیم‌ازوسط
  \درج‌تصویر[پهنا=1\پهنای‌سطر]{bdd-conceptual}
  \تر\موقتم{
    رابطه‌ی اجزای تشکیل‌دهنده‌ی \واژه{بیزنس.ولیو}
  }
  \شرح[\موقتم]{\موقتم~\مرجع{solis2011study}}
  \برچسب{fig:bdd-conceptual}
  \پایان{شکل}
 
  مرحله‌ی تحلیل در \واژه{بی.دی.دی.}، با شناسایی \واژه{رفتار}های مورد
  انتظار از سامانه آغاز می‌شود. در مراحل ابتدایی، شناسایی رفتارهای مورد
  انتظار آسان‌تر از شناسایی ارزش‌های تجاری سامانه است. هر رفتار تجاری،
  تعدادی \واژه{نتیجه.تجاری} محسوس را محقق می‌سازد.

  هر \واژه{نتیجه.تجاری} با یک \واژه{فیچر.ست} اعمال می‌شود. یک
  \واژه{فیچر.ست}‌ با گفتگو بین تیم توسعه و مشتری است برای تحقق
  \واژه{نتیجه.تجاری} در سامانه تدوین گشته و اولویت‌بندی و تبیین صریح
  ارتباظ آن با \واژه{نتیجه.تجاری}، ضروری است.

  در نهایت \واژه{اسکوپ} هر \واژه{فیچر.ست}، با استفاده از چند
  \واژه{یوزر.استوری} معین می‌شود. هر واژه{یوزر.استوری}، یک تعامل بین
  کاربر و سامانه را توصیف می‌کند. هر \واژه{یوزر.استوری}، به سه
  سوال زیر پاسخ می‌دهد:

  \شروع{بارهها}

  \باره{نقش کاربر در \واژه{یوزر.استوری} چیست؟}
  
  \باره{کاربر از این \واژه{فیچر} چه می‌خواهد؟}

  \باره{اگر سامانه این \واژه{فیچر} را داشته باشد، چه سودی به کاربر
    می‌رسد؟}

  \پایان{بارهها}

  همچنین برای هر \واژه{یوزر.استوری}، چند \واژه{اکسپتنس.کریتریا} به
  زبان مشتری تعیین می‌گردد که اگر برآورده شوند، آن \واژه{یوزر.استوری}
  به درستی محقق شده است. در \واژه{بی.دی.دی.}، \واژه{اکسپتنس.کریتریا}
  در \واژه{قالب} «سناریو» توصیف می‌شود که در بخش \رجوع{gwt} به تفصیل
  شرح داده شده است.

}

\پایان{شمارش}

\زیرقسمت{\واژه{قالب} توصیف \واژه{یوزر.استوری}‌ها و \واژه{سناریو}‌ها}
\برچسب{gwt}

سلام

\شروع{definition} به گزاره‌ای که در صورت صحّت پیاده‌سازی برقرار باشد یک
\واژه{سناریو} می‌گوییم.  \پایان{definition} \شروع{definition}

در \واژه{بی.دی.دی.} یک \واژه{سناریو} به صورت زیر نمایش
داده می‌شود:

{\نقل
\متن‌سیاه{\واژه{given}} شرط $g$ برقرار باشد، \\
\متن‌سیاه{\واژه{when}} اتفاق $w$ رخ بدهد، \\
\متن‌سیاه{\واژه{then}} شرط $t$ باید برقرار باشد.
}

و یا به طور خلاصه:

{\نقل
\متن‌سیاه{\واژه{given}} $g$ \\
\متن‌سیاه{\واژه{when}} $w$ \\
\متن‌سیاه{\واژه{then}} $t$
}

این شیوه‌ی بیان سناریو \واژه{given-when-then} نام دارد.
\پایان{definition}

%%% Local Variables:
%%% mode: latex
%%% TeX-master: "../main"
%%% End:



%%% Local Variables:
%%% mode: latex
%%% TeX-master: "../main"
%%% End:


\فصل{کارهای ما}
\برچسب{chap:our}

\قسمت{معماری}
آزمودن برنامه راهی برای افزایش کیفیت نرم‌افزار است. پیش از این رویکردهایی در توسعه‌ی \واژه{تست}های نرم‌افزاری را دیدیم. با ایده گرفتن از نقاط قوّت آن‌ها و در تلاش برای یافتن راهکاری برای ضعف‌های آن‌ها، معماری جدیدی را جهت انجام \واژه{تست} ارائه می‌کنیم.

از نظر ما این معماری یک روش جدید نیست، بلکه تکمیل‌کننده‌ی روش رفتاررانه است. بدین معنا که \واژه{بی.دی.دی.} اهداف کلّی و وضعیت مطلوب را توصیف می‌کند؛ حال آن که معماری پیشنهادی ما تلاش می‌کند تا ساختاری مناسب جهت دست‌یابی به این وضعیت را فراهم کند.

در بخش \رجوع{gwt} با نحوه‌ی بیان یک \واژه{سناریو} آشنا شدیم. این بیان به زبان ریاضی به شکل زیر است:

\شروع{equation}
\برچسب{eq:gIwIt}
g \implies (w \implies t)
\پایان{equation}


نکته‌ی قابل تأمّل آن که عبارت \رجوع{eq:gIwIt} و $g \land w \implies t$ معادلند. حال آن که این دو بخش از منظر معنایی تفاوت محسوسی دارند. مادامی که $g$ ؟؟؟؟


یک سناریو، در واقع پاسخ صحیح یک زیرمسأله از مسأله‌ی اصلی را بیان می‌کند. با معماری پیشنهادی ما، 

معماری پیشنهادی، حول ایده‌ی جداسازی \واژه{سناریو}ها از مصادیق صحّت‌سنجی آن صورت گرفته. در این معماری، \واژه{سناریو}هایی توسّط برنامه‌نویس تعبیه می‌شود که مطابق تعریف، انتظار می‌رود همواره برقرار باشند. مجموعه‌ی این \واژه{سناریو}ها یک 

همچنین، 

با توجّه به تعریف، خود \واژه{سناریو}ها تنها مجموعه‌ای از گزاره‌ها هستند. حال آن که سنجش صحّت این گزاره‌ها می‌تواند \واژه{متریک} خوبی برای کیفیّت کد باشد. لایه‌ی سوم معماری، لایه‌ای است که موجبات trigger شدن این \واژه{سناریو}ها را فراهم می‌کند. به عبارتی، این لایه موظّف است شرایط متنوّعی را فراهم آورد که شروط $given$ سناریوهای متفاوت برقرار شود

به این ترتیب، 

تفاوت معنایی: 


در زیر نمونه‌ای از شبه‌کد آزمون 


\قسمت{فواید}

\زیرقسمت{کاهش حجم تست‌ها}
\برچسب{sec:loc}
یکی از دلایل عمده‌ی عدم علاقه‌ی برنامه‌نویس‌ها به تألیف تست، حجم زیاد تست در مقایسه با میزان کد پوشیده شده توسّط آن است. هر چه سطح تست به سطح سیستمی نزدیک‌تر باشد، حجم این کد نیز به دلیل آماده کردن شرایط اوّلیه‌ی تست افزایش می‌یابد.

\زیرقسمت{کاهش هزینه‌ی نگه‌داری}
از آن جا که تغییر سریع در بسیاری از سیستم‌های نرم‌افزاری ضروری است، حجم زیاد تست‌ها (ر.ک. \رجوع{sec:loc}) در هنگام تغییر نیز نیاز به هم‌گام‌سازی دارند. از آنجا که تست‌ها ???

با توجّه به جدا شدن لایه‌ی \واژه{سناریو} و \واژه{اکتور} در معماری پیشنهادی، تغییرات محدود به بخشی از \واژه{تست}ها خواهند بود که واقعاً تغییر کرده و به بخش‌های دیگر انتشار نمی‌یابند.


\زیرقسمت{افزایش احتمال یافتن خطا}
یکی از تفاوت‌های اساسی معماری پیشنهادی و معماری‌های تک‌لایه در 
از آنجا که یک \واژه{سناریو} 
دسته تست به جای تک تست


\قسمت{در سطح سیستمی}
اگر چه معماری پیشنهادی ما مستقل از سطح \واژه{تست} است، با این حال برداشت مؤلفان این است که تأثیر مثبت آن در زمینه‌ی \واژه{سیستم.تست} مشهودتر خواهد بود. در این لایه به دلیل سطح بالا بودن \واژه{تست}‌ها، حجم آماده‌سازی‌هایی که یک \واژه{تست.کیس} می‌بایست انجام دهد تا به \واژه{پری.کاندیشن} دلخواه برسد بسیار بیشتر بوده و با توجه به آن‌چه در \رجوع{sec:loc} آمد، معماری پیشنهادی از طریق جدا کردن \واژه{اکتور} و \واژه{reuse} آن، به کاهش حجم و هزینه‌ی نگه‌داری این‌گونه \واژه{تست}‌ها کمک شایانی می‌کند.


\قسمت{خارج از محیط \واژه{تست}}
انجام \واژه{تست} تنها یکی از روش‌های افزایش کیفیت نرم‌افزار است. ادعا می‌کنیم که معماری پیشنهادی خارج از محیط \واژه{تست} نیز می‌تواند به افزایش کیفیت نرم‌افزار کمک کند.

در هنگام \واژه{تست}، هدف یافتن \واژه{defect} نرم‌افزار، پیش از عملیاتی شدن آن است. اما بسیاری از \واژه{defect}‌های نرم‌افزاری حتی پس از عملیاتی شدن نیز از دیده نهان می‌مانند.

در \واژه{فریم.ورک} ارائه‌شده، می‌توان به جای لایه‌ی اکتور، کاربران حقیقی \واژه{سیستم} را قرار داد تا از نرم‌افزار استفاده کنند. به این ترتیب، \واژه{سیستم} می‌تواند عملکرد خودش را (حتی در حالی که عملیاتی است) ارزیابی کند و برخی از این \واژه{defect}‌ها را یافته، و جهت رسیدگی توسعه‌دهندگان گزارش کند. از جمله \واژه{defect}‌هایی که به خوبی به این روش پیدا می‌شوند، بروز \واژه{inconsistency} در مقادیر محاسبه‌شدنی است.

\قسمت{مقایسه با مفاهیم دیگر}
\زیرقسمت{در مقایسه با contract}

\زیرقسمت{در مقایسه با mutational testing}






\فصل{نتیجه‌گیری و کارهای آتی}
\برچسب{chap:future}

\قسمت{جمع‌بندی}
در طول این پروژه مجموعاً فعالیت‌های زیر صورت گرفت:
\شروع{enumerate}
\فقره بررسی کارهای پیشین در زمینه‌ی توسعه‌ی \واژه{تست}
\فقره ارائه‌ی معماری پیشنهادی با ایده گرفتن از نقاط قوت و ضعف مشهود در روش‌های پیشین
\فقره مقایسه‌ی این معماری با روش‌های قبلی
\فقره پیاده‌سازی یک \واژه{فریم.ورک} \واژه{تست} مبتنی بر این معماری
\فقره پیاده‌سازی یک نرم‌افزار نمونه و تألیف \واژه{تست} برای آن توسط \واژه{فریم.ورک} پیشنهادی
\پایان{enumerate}

نهایتاً برآورد مؤلفان این بود که استفاده از معماری پیشنهادی، همان‌طور که انتظار می‌رفت، کمک شایانی به ساده‌سازی انجام \واژه{تست} نرم‌افزار نمونه کرد. انتظار می‌رود که استفاده از این معماری در \واژه{تست} نرم‌افزارهای بزرگتر، مؤثرتر نیز واقع شود.

\قسمت{کارهای آتی}
در طول انجام پروژه، ایده‌های فراوانی مبتنی بر معماری پیشنهادی به ذهنمان رسید که در راستای خارج نشدن از حوزه‌ی این پروژه وارد آن‌ها نشدیم، اما بررسی آن‌ها خالی از لطف نخواهد بود:

\شروع{itemize}
\فقره حذف و هرس \واژه{سناریو}‌ها پس از بررسی صحت آن‌ها به دفعات کافی
\فقره تعریف \واژه{متریک}‌های جدید سنجش کیفیت نرم‌افزار بر حسب \واژه{سناریو}‌ها
\فقره استفاده از \واژه{فریم.ورک} \واژه{تست} پیاده‌شده در یک پروژه‌ی متن‌باز و بزرگ‌تر، جهت ارزیابی دقیق‌تر این روش
\فقره پیاده‌سازی \واژه{اکتور} به شیوه‌های مختلف و بررسی میزان اثرگذاری آن‌ها
\پایان{itemize}


\فصل{کارهای ما}
\برچسب{chap:our}

\قسمت{معماری}
آزمودن برنامه راهی برای افزایش کیفیت نرم‌افزار است. پیش از این رویکردهایی در توسعه‌ی \واژه{تست}های نرم‌افزاری را دیدیم. با ایده گرفتن از نقاط قوّت آن‌ها و در تلاش برای یافتن راهکاری برای ضعف‌های آن‌ها، معماری جدیدی را جهت انجام \واژه{تست} ارائه می‌کنیم.

از نظر ما این معماری یک روش جدید نیست، بلکه تکمیل‌کننده‌ی روش رفتاررانه است. بدین معنا که \واژه{بی.دی.دی.} اهداف کلّی و وضعیت مطلوب را توصیف می‌کند؛ حال آن که معماری پیشنهادی ما تلاش می‌کند تا ساختاری مناسب جهت دست‌یابی به این وضعیت را فراهم کند.

در بخش \رجوع{gwt} با نحوه‌ی بیان یک \واژه{سناریو} آشنا شدیم. این بیان به زبان ریاضی به شکل زیر است:

\شروع{equation}
\برچسب{eq:gIwIt}
g \implies (w \implies t)
\پایان{equation}


نکته‌ی قابل تأمّل آن که عبارت \رجوع{eq:gIwIt} و $g \land w \implies t$ معادلند. حال آن که این دو بخش از منظر معنایی تفاوت محسوسی دارند. مادامی که $g$ ؟؟؟؟


یک سناریو، در واقع پاسخ صحیح یک زیرمسأله از مسأله‌ی اصلی را بیان می‌کند. با معماری پیشنهادی ما، 

معماری پیشنهادی، حول ایده‌ی جداسازی \واژه{سناریو}ها از مصادیق صحّت‌سنجی آن صورت گرفته. در این معماری، \واژه{سناریو}هایی توسّط برنامه‌نویس تعبیه می‌شود که مطابق تعریف، انتظار می‌رود همواره برقرار باشند. مجموعه‌ی این \واژه{سناریو}ها یک 

همچنین، 

با توجّه به تعریف، خود \واژه{سناریو}ها تنها مجموعه‌ای از گزاره‌ها هستند. حال آن که سنجش صحّت این گزاره‌ها می‌تواند \واژه{متریک} خوبی برای کیفیّت کد باشد. لایه‌ی سوم معماری، لایه‌ای است که موجبات trigger شدن این \واژه{سناریو}ها را فراهم می‌کند. به عبارتی، این لایه موظّف است شرایط متنوّعی را فراهم آورد که شروط $given$ سناریوهای متفاوت برقرار شود

به این ترتیب، 

تفاوت معنایی: 


در زیر نمونه‌ای از شبه‌کد آزمون 


\قسمت{فواید}

\زیرقسمت{کاهش حجم تست‌ها}
\برچسب{sec:loc}
یکی از دلایل عمده‌ی عدم علاقه‌ی برنامه‌نویس‌ها به تألیف تست، حجم زیاد تست در مقایسه با میزان کد پوشیده شده توسّط آن است. هر چه سطح تست به سطح سیستمی نزدیک‌تر باشد، حجم این کد نیز به دلیل آماده کردن شرایط اوّلیه‌ی تست افزایش می‌یابد.

\زیرقسمت{کاهش هزینه‌ی نگه‌داری}
از آن جا که تغییر سریع در بسیاری از سیستم‌های نرم‌افزاری ضروری است، حجم زیاد تست‌ها (ر.ک. \رجوع{sec:loc}) در هنگام تغییر نیز نیاز به هم‌گام‌سازی دارند. از آنجا که تست‌ها ???

با توجّه به جدا شدن لایه‌ی \واژه{سناریو} و \واژه{اکتور} در معماری پیشنهادی، تغییرات محدود به بخشی از \واژه{تست}ها خواهند بود که واقعاً تغییر کرده و به بخش‌های دیگر انتشار نمی‌یابند.


\زیرقسمت{افزایش احتمال یافتن خطا}
یکی از تفاوت‌های اساسی معماری پیشنهادی و معماری‌های تک‌لایه در 
از آنجا که یک \واژه{سناریو} 
دسته تست به جای تک تست


\قسمت{در سطح سیستمی}
اگر چه معماری پیشنهادی ما مستقل از سطح \واژه{تست} است، با این حال برداشت مؤلفان این است که تأثیر مثبت آن در زمینه‌ی \واژه{سیستم.تست} مشهودتر خواهد بود. در این لایه به دلیل سطح بالا بودن \واژه{تست}‌ها، حجم آماده‌سازی‌هایی که یک \واژه{تست.کیس} می‌بایست انجام دهد تا به \واژه{پری.کاندیشن} دلخواه برسد بسیار بیشتر بوده و با توجه به آن‌چه در \رجوع{sec:loc} آمد، معماری پیشنهادی از طریق جدا کردن \واژه{اکتور} و \واژه{reuse} آن، به کاهش حجم و هزینه‌ی نگه‌داری این‌گونه \واژه{تست}‌ها کمک شایانی می‌کند.


\قسمت{خارج از محیط \واژه{تست}}
انجام \واژه{تست} تنها یکی از روش‌های افزایش کیفیت نرم‌افزار است. ادعا می‌کنیم که معماری پیشنهادی خارج از محیط \واژه{تست} نیز می‌تواند به افزایش کیفیت نرم‌افزار کمک کند.

در هنگام \واژه{تست}، هدف یافتن \واژه{defect} نرم‌افزار، پیش از عملیاتی شدن آن است. اما بسیاری از \واژه{defect}‌های نرم‌افزاری حتی پس از عملیاتی شدن نیز از دیده نهان می‌مانند.

در \واژه{فریم.ورک} ارائه‌شده، می‌توان به جای لایه‌ی اکتور، کاربران حقیقی \واژه{سیستم} را قرار داد تا از نرم‌افزار استفاده کنند. به این ترتیب، \واژه{سیستم} می‌تواند عملکرد خودش را (حتی در حالی که عملیاتی است) ارزیابی کند و برخی از این \واژه{defect}‌ها را یافته، و جهت رسیدگی توسعه‌دهندگان گزارش کند. از جمله \واژه{defect}‌هایی که به خوبی به این روش پیدا می‌شوند، بروز \واژه{inconsistency} در مقادیر محاسبه‌شدنی است.

\قسمت{مقایسه با مفاهیم دیگر}
\زیرقسمت{در مقایسه با contract}

\زیرقسمت{در مقایسه با mutational testing}






\singlespace




\PrepareForBibliography

\setlatintextfont[Scale=1]{Linux Libertine}
\setlength{\baselineskip}{0.8cm}
%\setromantextfont[Scale=1.2]{XB Niloofar}

%\bibliographystyle{IEEEtran}
%\bibliographystyle{is-unsrt}
%\bibliographystyle{ieeetr-fa}
%\bibliographystyle{amsplain}

%\bibliography{resources/resources}
\latin
\printbibliography[title=\bibliographytitle,heading=bibintoc]
\persian

% glossaries
{\cleardoublepage\setlength{\baselineskip}{1cm}\printpersianglossary\cleardoublepage\printenglishglossary}


\پایان{نوشتار}

%%% Local Variables:
%%% mode: latex
%%% TeX-master: t
%%% End:
