\documentclass[BScThesis,twoside]{sharifthesis}
%\documentclass[MScThesis,oneside]{sharifthesis}
%\documentclass[PhDThesis,twoside]{sharifthesis}
%\documentclass[PhDThesis,oneside]{sharifthesis}
%\documentclass[PhDProposal,twoside]{sharifthesis}
%\documentclass[PhDProposal,oneside]{sharifthesis}

\def\enSubject{Bachelor of Science Thesis (Information Technology Major), Computer Engineering Department, Sharif University of Technology, Tehran, I. R. Iran}
%\def\enSubject{Master of Science Thesis (Information Technology Major), Computer Engineering Department, Sharif University of Technology, Tehran, I. R. Iran}
%\def\enSubject{Doctor of Philosophy Thesis (Information Technology Major), Computer Engineering Department, Sharif University of Technology, Tehran, I. R. Iran}
%\def\enSubject{Doctor of Philosophy Research Proposal (Information Technology Major), Computer Engineering Department, Sharif University of Technology, Tehran, I. R. Iran}

%do not use newline command (\\) in following definition
\def\enTitle{Decoupling Scenarios from Behavior-Driven Tests}
%if title is long and requires to be splitted in two lines, uncomment following two definitions and split title at appropriate location
%\def\enTitleLineOne{First line of the long title}
%\def\enTitleLineTwo{and its continuation on the second line}
\def\enAuthor{Seyed Mehran Kholdi, Mohammad Hossein Sekhavat}
\def\enKeywords{Behavior-Driven Development, Decoupling, Software Test Quality}


\input{general/preamble}
\addbibresource{resources/resources.bib}


\newcommand{\faKeywords}{تست، رفتار رانه، سناریو، مستقل‌سازی
}
\eqcommand{واژه‌های‌کلیدی}{faKeywords}
\آرم{\درج‌تصویر[scale=.7]{logo}}
\تاریخ{شهریور ۱۳۹۵}
%در دستور زیر، از \\ استفاده نکنید
\عنوان{مستقل‌سازی سناریو از تست رفتار رانه}
%اگر عنوان طولانی بوده (و در عنوان انگلیسی از دو خظ استفاده شده) باید دو خط زیر از کامنت خارج و دو خط عنوان توسط آن‌ها تعریف گردد.
%\عنوانخطیک{خط نخست عنوان طولانی}
%\عنوانخطدو{و ادامه‌ی آن در خط دوم}
\نویسنده{سید مهران خلدی، محمد حسین سخاوت}
\دانشگاه{{\نستعلیق\درشت‌تر دانشگاه صنعتی شریف %
\\[0.6cm]}
دانشکده‌ی مهندسی کامپیوتر}
\دانشگاه‌عادی{دانشگاه صنعتی شریف\\
دانشکده‌ی مهندسی کامپیوتر}
\موضوع{گرایش نرم‌افزار}
\استادراهنما{دکتر سید حسن میریان}
\استادمشاور{مهندس مصطفی مهدیه} 

\newcommand{\efootnote}[1]{\footnote{\lr{#1}}}
\newcommand{\ecfootnote}[1]{}


% ================ Correct hyphenations ================
\hyphenation{test}


\makeglossaries
%\includeonly{related_works/related_works}
%\includeonly{evaluation/evaluation}

% ===== DEPRACATED AREA =====
% Following commands are provided to make older documents compilable.
% These commands are depracated and should not be used in new documents.
\newcommand{\ترجمه‌ج}[2]
{\ترجمه[#1‌ها]{#1}{#2}}
\newcommand{\ترجمه‌جمع}[3]
{\ترجمه[#3]{#1}{#2}}
\newcommand{\برگردان}[3]
{\ترجمه{#1}{#3}\زیرنویس{#2}}
\eqcommand{اسم}
{نام}
% ===== END OF DEPRACATED AREA =====

\شروع{نوشتار}


\newcommand{\StartDocument}{\frontmatter \baselineskip1.2\baselineskip \pagestyle{empty} \null \vfill
\شروع{وسط‌چین}
{\نستعلیق‌درشت بسم اللّه الرّحمن الرّحیم}
\پایان{وسط‌چین}
\vfill}

%the initial title is supposed to be printed on the cover.
%for non final version, you can leave following commands as is to create only one title page (printed on paper)
%for final version you need to swap folowing commented/uncommented makethesistitle commands to achieve this order: title on the cover THEN in the name of god page THEN another title page but this time printed on paper
\makethesistitle
\StartDocument
%\makethesistitle

\setlength{\baselineskip}{0.9cm}
\begin{comment}
\فصل*{پیش‌گفتار}
\thispagestyle{pagenumberonlyPagestyle}
پیشگفتار اختیاری است. در صورت تمایل به نگارش پیش‌گفتار، محیط کامنت که آن را دربرگرفته باید حذف شود.
\end{comment}

\شروع{قدردانی}
از استاد بزرگوار دکتر میریان که ما را در انجام این پروژه یاری کردند،
تشکر و قدردانی می‌کنیم. 

همچنین از جناب آقای مهندس مهدیه به خاطر مشورت‌ها و پیگیری‌های
عالمانه و دلسوزانه‌ی ایشان قدردانی می‌نماییم.

\پایان{قدردانی}
% END OF COMMENT FOR PhD Proposal.

\شروع{چکیده}{\واژه‌های‌کلیدی}
\واژه{تست} نرم‌افزار یکی از روش‌های اصلی افزایش و حصول اطمینان از کیفیت آن است. \واژه{بی.دی.دی.} یکی از \واژه{تکنیک}‌های محبوب و رایج صنعت، در تألیف \واژه{تست} است. در این پایان‌نامه ابتدا به بررسی دقیق \واژه{بی.دی.دی.} و مقایسه‌ی آن با \واژه{تکنیک}‌های مشابه می‌پردازیم.

در ادامه \واژه{فریم.ورک} \واژه{تست}ی جهت استفاده در \واژه{تکنیک} \واژه{بی.دی.دی.} ارائه می‌کنیم، که با اتکا بر \واژه{decoupling} ساختار \واژه{تست}، توسعه‌دهنده را در افزایش کیفیت \واژه{تست}‌های تألیفی و کاهش هزینه‌ی نگهداری آن‌ها یاری می‌کند.

نهایتاً و به عنوان نمونه، با استفاده از این \واژه{فریم.ورک} \واژه{تست} برای یک سامانه‌ی کوچک نرم‌افزاری، \واژه{تست}‌هایی تألیف می‌کنیم.

\پایان{چکیده}


\setlength{\baselineskip}{0.9cm}
\pagenumbering{tartibi}\tableofcontents\listoftables\listoffigures
%list of abbreviations may be added here...


\PrepareForMainContent
%	In the name of GOD

\فصل{مقدمه} \برچسب{chap:intro} \قسمت{تعریف مسأله} \واژه{بی.دی.دی.} یکی
از \واژه{تکنیک}‌های جدید برای \واژه{ای.اس.دی.} است که استفاده از آن در
صنعت نرم‌افزار رواج دارد \مرجع{cucumberusing}. این \واژه{تکنیک} بر
مراحل مختلف \واژه{اس.دی.پی.} از جمله برنامه‌ریزی، تحلیل، طراحی،
پیاده‌سازی و \واژه{تست} تاثیرگذار است (ر.ک. به
\رجوع{sec:bdd-props}). از جمله مسأله‌هایی که در مرحله‌ی \واژه{تست} از
فرآیند \واژه{بی.دی.دی.} وجود دارد، این است که با توجه به اینکه
\واژه{اکسپتنس.تست} ها در این روش، مستقیماً از روی توصیف بدست می‌آیند،
لازم است هنگام توصیف، جزییاتی را تعیین نماییم که برای توصیف نیازمندی
کاربردی ندارند، بلکه در مرحله‌ی تولید \واژه{اکسپتنس.تست} از روی توصیف
به آن‌ها نیاز داریم. این در صورتی است که توصیه می‌شود جزییات غیر ضروری
در توصیف ذکر نشوند \مرجع{adzik2011sbe}.

برای مثال، در یک سامانه‌ی انتخاب واحد، ممکن است برای توصیف قابلیت
ثبت‌نام کاربر در یک ارائه از یک درس، صرف وجود یک دانشجو برای توصیف عمل
انتخاب واحد کافی باشد؛ اما برای تولید \واژه{اکسپتنس.تست} از روی توصیف
این قابلیت، لازم است جزییاتی غیر ضروری مانند «نام» و «نام خانوادگی» از دانشجو
در توصیف این قابلیت ذکر شوند. در مستندات یکی از چارچوب‌های فعلی برای
\واژه{بی.دی.دی.} \مرجع{cucumberusing} نیز به این مسأله اشاره شده است.

\قسمت{اهمیت موضوع}
در عصر اینترنت، سرعت تحول نرم‌افزار موضوعی بسیار مهم است. تا اواخر قرن
بیستم، هر پروژه‌ی نرم‌افزاری برای چندین سال بدون تغییرات عمده استفاده
می‌شد و اجرای هر فاز از اجرای پروژه، چند ماه به‌طول می‌انجامید. اما
امروزه، روش‌های \واژه{ای.اس.دی.} در بسیاری از پروژه‌های نرم‌افزاری به‌کار
گرفته می‌شوند؛ مدت زمان اجرای کل پروژه به چند ماه و مدت زمان اجرای هر
فاز، به چند هفته یا حتی چند روز کاهش پیدا کرده و مقاومت پروژه‌های
نرم‌افزاری در مقابل تغییرات بسیار کاهش یافته است \مرجع{andres2004extreme}.

با این نرخ بالای تغییرات، مستندات پروژه خیلی سریع منسوخ می‌شوند؛ به‌روز
نگه داشتن توصیف نیازمندی‌ها و اجرای \واژه{تست} به شیوه‌های سنتی، بسیار
پرهزینه محسوب شده و عملا بی‌فایده است. در روش‌های \واژه{ای.اس.دی.}،
بیشتر از \نام{تولید درست محصول}{Building the product right}، به
\نام{تولید محصول درست}{Building the right product} توجه می‌شود و لازم
است هزینه‌ی مستندسازی و \واژه{تست} را کاهش داده و بر تولید محصول درست
تمرکز شود \مرجع{adzik2011sbe}.

در تکنیک \واژه{بی.دی.دی.}، فرآیند توصیف نیازمندی‌ها به‌صورت
\واژه{تکراری} است و با توجه به زبان مشترک توصیف نرم‌افزار بین مشتری و
تیم توسعه و همچنین قابل اجرا بودن آن، از توصیف به‌عنوان \واژه{تست} و
مستندات پروژه نیز بهره گرفته می‌شود. این امر باعث کاهش هزینه‌ی نگهداری،
افزایش سرعت توسعه \واژه{تست} و انطباق‌پذیری با تغییرات با وجود حفظ
کیفیت نرم‌افزار می‌شود. همچنین وجود چارچوبی کارا برای پیاده‌سازی این
\واژه{تکنیک}، بر سرعت توسعه و به‌کارگیری صحیح اصول این \واژه{تکنیک}
تاثیر مستقیم دارد.

\قسمت{اهداف پروژه}
ایده‌ی این پروژه از همکاری مؤلفان در یک پروژه‌ی صنعتی شکل گرفت. به عنوان توسعه‌دهندگان \واژه{سیستم} مذکور، همیشه حس می‌کردیم که بخش قابل توجهی از کدهایی که برای \واژه{تست} نرم‌افزار می‌نویسیم شباهت‌های زیادی به یک‌دیگر دارند.
از سوی دیگر، به دلیل تغییر نیازمندی‌های \واژه{سیستم}، همیشه لازم بود تا \واژه{تست}‌ها به‌روزرسانی شوند و این تغییرات معمولاً در جای‌جای پروژه منتشر می‌شد. تمام این مشکلات منجر می‌شد تا کیفیت \واژه{تست}‌ها در طول زمان کاهش یافته و اثرگذاری خود را از دست دهند.

بنابراین سعی کردیم الگوهای مشترکی بین \واژه{تست}‌ها پیدا کنیم و با \واژه{ریفکتورینگ} مکرر آن‌هابه ساختاری رسیدیم که بسیاری از مشکلات فوق را کم‌رنگ کرده بود. تلاش کردیم تا ساختار نهایی را تعمیم دهیم تا قابل استفاده در پروژه‌های دیگر نیز باشد. نتیجه \واژه{فریم.ورک} \واژه{تست}ی شد که در این پروژه ارائه می‌کنیم.

هدف اصلی از ارائه‌ی \واژه{فریم.ورک} \واژه{تست} پیشنهادی، کاهش هزینه‌ی نگهداری \واژه{تست}ها و افزایش اثرگذاری آن‌ها در کیفیت نرم‌افزار است. در بخش \رجوع{sec:adv} فواید مطرح‌شده پس از استفاده از \واژه{فریم.ورک} مورد نظر را با دقت بیشتری بررسی خواهیم کرد.

همچنین، اگر چه ایده‌هایی کم و بیش شبیه معماری پیشنهادی پیش از این نیز ارائه شده‌اند، بخشی از پروژه‌ی ما به انجام پیاده‌سازی ابزارهای مورد نیاز پرداخته است؛ چرا که در حوزه‌هایی مانند \واژه{تست} که در صنعت شدیداً مورد استفاده قرار می‌گیرند، ارائه‌ی ابزار در کنار یک روش و معماری نیز حائز اهمیت است.


\قسمت{ساختار پایان‌نامه}
این پایان‌نامه حاوی پنج فصل است.

در فصل نخست مقدمه‌ای از موضوع پایان‌نامه و اهمیت آن ارائه شده. در فصل دوم پیش‌زمینه‌ای از کارهای قبلی که در این زمینه صورت گرفته آمده، و در ادامه \واژه{بی.دی.دی.} به تفصیل شرح داده شده. فصل سوم \واژه{فریم.ورک} \واژه{تست} پیشنهادی را از جنبه‌ی معماری و پیاده‌سازی بررسی می‌کند. فصل چهارم به ارائه‌ی گزارشی از یک نمونه‌ی استفاده از \واژه{فریم.ورک} پیاده‌سازی‌شده در فصل سوم می‌پردازد. و نهایتاً در فصل پنجم جمع‌بندی و کارهای آتی که انجام آن‌ها در آینده امکان‌پذیر خواهد بود آمده.

%%% Local Variables:
%%% mode: latex
%%% TeX-master: "../main"
%%% End:


\فصل{پیش‌زمینه}
\برچسب{chap:back}


\input{background/history}

\قسمت{ادبیات}\برچسب{sec:back:vocabulary}

اصطلاح‌های مرتبط با توسعه‌ی نرم‌افزار، ابهام‌ها و برداشت‌های گوناگون دارند
\مرجع{gartner2012atdd}. برای پیشگیری از ابهام و آشنایی خواننده با
مفاهیم، در این قسمت اصطلاح‌ها و مفاهیمی که در این پایان‌نامه از آن‌ها
استفاده شده، بدون ارزیابی جنبه‌های مختلف، شرح داده شده اند.

\input{background/methodology}

\input{background/tfd}

%%% Local Variables:
%%% mode: latex
%%% TeX-master: "../main"
%%% End:


\قسمت{\واژه{تکنیک} \واژه{بی.دی.دی.}}\برچسب{sec:back:bdd}

\واژه{بی.دی.دی.} یکی از تکنیک‌های توسعه‌ی چابک نرم‌افزار است که در سال
۲۰۰۶ توسط \نام{نورث}{Dan North} برای پاسخ‌گویی به ابهام‌ها و مشکل‌هایی که
در \واژه{تی.دی.دی.} بوجود آمده‌بود، معرفی
شد. \مرجع{north2006introducing}.  تاکید \واژه{بی.دی.دی.} بر بهینه‌سازی
ارتباط میان نقش‌های برنامه‌نویس، آزمون‌گر و \واژه{دامین.اکسپرت}‌ می‌باشد و
برای تیمی که تمام اعضای آن برنامه‌نویس باشند، \واژه{بی.دی.دی.} سودی
نسبت به \واژه{تی.دی.دی.} ندارد \مرجع{north2012}.

این تکنیک، که اخیرا هم در عمل و هم در پژوهش بسیار متداول شده است، حاصل
ترکیب و بهبود \واژه{پرکتیس}‌های معرفی شده در تکنیک‌های \واژه{تی.دی.دی.}
و \واژه{ای.تی.دی.دی.} و علاوه‌بر آن در نظر گرفتن اصول زیر است
\مرجع{alience:bdd}:

\شروع{بارهها}

\باره{ لازم است هدف هر \واژه{یوزر.استوری} و سودی که برای مشتری دارد،
  مشخص باشد.}

\باره{ با نگرش «\نام{از بیرون به درون}{From the outside in}»، فقط
  رفتارهایی از \واژه{سیستم} پیاده‌سازی می‌شوند که بیشترین سود را به
  مشتری می‌رسانند. با داشتن این نگرش، اتلاف هزینه کمینه می‌شود. }

\باره{ رفتارهای مورد انتظار از سامانه، بین \واژه{دامین.اکسپرت}،
  توسعه‌دهنده‌ی \واژه{سیستم} و آزمون‌گر به یک زبان مشترک توصیف می‌شوند. در نتیجه
  ارتباط ضروری میان این نقش‌ها بهبود می‌یابد. }

\باره { این اصول، در تمام سطوح انتزاعی توصیف برنامه، تا پایین‌ترین
  سطح که پیاده‌سازی یک واحد است، رعایت می‌شوند. }

\پایان{بارهها}

\زیرقسمت{مشخصات \واژه{بی.دی.دی.}}  در حال حاضر، \واژه{تکنیک}
\واژه{بی.دی.دی.}، هنوز در حال توسعه است و تعریف روشنی از
\واژه{بی.دی.دی.} که بر آن اجماع باشد، وجود ندارد. با توجه به اینکه
فرآیند \واژه{بی.دی.دی.}، از ابتدا به‌صورت انتزاعی معرفی شده و جزییاتی
برای آن ارائه نشده، مشخصات \واژه{بی.دی.دی.} مبهم و پراکنده
هستند. همچنین چارچوب‌ها و ابزارهایی که برای \واژه{بی.دی.دی.} ارائه
شده‌اند، بیشتر به بخش «پیاده‌سازی آزمون» از فرآیند \واژه{بی.دی.دی.}
تمرکز دارند؛ در صورتی که \واژه{بی.دی.دی.} بر حوزه‌ی گسترده‌تری از
\واژه{اس.دی.پی.} تاثیر دارد \مرجع{solis2011study}.

در مقاله‌ی \مرجع{solis2011study}، ۶ مورد از مشخصات اصلی
\واژه{بی.دی.دی.} که بر کل \واژه{اس.دی.پی.}، نه فقط قسمت «پیاده‌سازی
آزمون»، تاثیر گذار اند، ارائه شده است. در اینجا به صورت خلاصه به آن‌ها
اشاره می‌کنیم:

\شروع{شمارش}

\فقره{ \متن‌سیاه{زبان \واژه{یوبی}}:

  مفهوم «زبان \واژه{یوبی}»، هسته‌ی \واژه{بی.دی.دی.} را تشکیل
  می‌دهد. ساختار \واژه{یوبی} برآمده از \واژه{بیزنس.دامین} می‌باشد و
  ابهام را از مکالمه‌ی بین مشتری و تیم توسعه کاهش می‌دهد. همچنین یک
  واژه‌نامه در ابتدای پروژه ایجاد شده، اکثر واژه‌های آن در مرحله‌ی تحلیل
  افزوده می‌شوند و در مراحل بعد امکان گسترش دارد.

  هر \واژه{بیزنس.دامین}، زبان \واژه{یوبی} خاص خود را لازم دارد.
  \واژه{بی.دی.دی.} یک \واژه{قالب} ساده برای مرحله‌ی تحلیل ارائه نموده است که
  مستقل از \واژه{بیزنس.دامین} است و از آن در زبان \واژه{یوبی} بهره‌گیری
  می‌شود. این \واژه{قالب} در بخش \رجوع{gwt} شرح داده شده است.  }

\فقره{ \متن‌سیاه{فرآیند \واژه{تفکیک} \واژه{تکراری}}:

  توقع مشتری از یک پروژه‌ی نرم‌افزاری، ایجاد \واژه{بیزنس.ولیو}
  می‌باشد. معمولاً تشخیص و روشن‌سازی \واژه{بیزنس.ولیو}، دشوار است. به
  همین دلیل، \واژه{بیزنس.ولیو} به اجزای ملموس‌تر تفکیک می‌شود. در شکل
  \رجوع{fig:bdd-conceptual} رابطه‌ی اجزای استفاده شده در این
  \واژه{تفکیک} نسبت به هم نمایش داده شده است.

  \شروع{شکل}[tbp]
  \تنظیم‌ازوسط
  \درج‌تصویر[پهنا=1\پهنای‌سطر]{bdd-conceptual}
  \تر\موقتم{
    رابطه‌ی اجزای تشکیل‌دهنده‌ی \واژه{بیزنس.ولیو}
  }
  \شرح[\موقتم]{\موقتم~\مرجع{solis2011study}}
  \برچسب{fig:bdd-conceptual}
  \پایان{شکل}
 
  مرحله‌ی تحلیل در \واژه{بی.دی.دی.}، با شناسایی \واژه{رفتار}های مورد
  انتظار از سامانه آغاز می‌شود. در مراحل ابتدایی، شناسایی رفتارهای مورد
  انتظار آسان‌تر از شناسایی ارزش‌های تجاری سامانه است. هر رفتار تجاری،
  تعدادی \واژه{نتیجه.تجاری} محسوس را محقق می‌سازد.

  هر \واژه{نتیجه.تجاری} با یک \واژه{فیچر.ست} اعمال می‌شود. یک
  \واژه{فیچر.ست}‌ با گفتگو بین تیم توسعه و مشتری است برای تحقق
  \واژه{نتیجه.تجاری} در سامانه تدوین گشته و اولویت‌بندی و تبیین صریح
  ارتباظ آن با \واژه{نتیجه.تجاری}، ضروری است.

  در نهایت \واژه{اسکوپ} هر \واژه{فیچر.ست}، با استفاده از چند
  \واژه{یوزر.استوری} معین می‌شود. هر واژه{یوزر.استوری}، یک تعامل بین
  کاربر و سامانه را توصیف می‌کند. هر \واژه{یوزر.استوری}، به سه
  سوال زیر پاسخ می‌دهد:

  \شروع{بارهها}

  \باره{نقش کاربر در \واژه{یوزر.استوری} چیست؟}
  
  \باره{کاربر از این \واژه{فیچر} چه می‌خواهد؟}

  \باره{اگر سامانه این \واژه{فیچر} را داشته باشد، چه سودی به کاربر
    می‌رسد؟}

  \پایان{بارهها}

  همچنین برای هر \واژه{یوزر.استوری}، چند \واژه{اکسپتنس.کریتریا} به
  زبان مشتری تعیین می‌گردد که اگر برآورده شوند، آن \واژه{یوزر.استوری}
  به درستی محقق شده است. در \واژه{بی.دی.دی.}، \واژه{اکسپتنس.کریتریا}
  در \واژه{قالب} «سناریو» توصیف می‌شود که در بخش \رجوع{gwt} به تفصیل
  شرح داده شده است.

}

\پایان{شمارش}

\زیرقسمت{\واژه{قالب} توصیف \واژه{یوزر.استوری}‌ها و \واژه{سناریو}‌ها}
\برچسب{gwt}

سلام

\شروع{definition} به گزاره‌ای که در صورت صحّت پیاده‌سازی برقرار باشد یک
\واژه{سناریو} می‌گوییم.  \پایان{definition} \شروع{definition}

در \واژه{بی.دی.دی.} یک \واژه{سناریو} به صورت زیر نمایش
داده می‌شود:

{\نقل
\متن‌سیاه{\واژه{given}} شرط $g$ برقرار باشد، \\
\متن‌سیاه{\واژه{when}} اتفاق $w$ رخ بدهد، \\
\متن‌سیاه{\واژه{then}} شرط $t$ باید برقرار باشد.
}

و یا به طور خلاصه:

{\نقل
\متن‌سیاه{\واژه{given}} $g$ \\
\متن‌سیاه{\واژه{when}} $w$ \\
\متن‌سیاه{\واژه{then}} $t$
}

این شیوه‌ی بیان سناریو \واژه{given-when-then} نام دارد.
\پایان{definition}

%%% Local Variables:
%%% mode: latex
%%% TeX-master: "../main"
%%% End:



%%% Local Variables:
%%% mode: latex
%%% TeX-master: "../main"
%%% End:


\فصل{کارهای ما}
\برچسب{chap:our}

\قسمت{معماری \واژه{فریم.ورک}}

در بخش \رجوع{gwt} با نحوه‌ی بیان یک \واژه{سناریو} آشنا شدیم. یک سناریو، در واقع پاسخ صحیح یک زیرمسأله از مسأله‌ی اصلی را بیان می‌کند.

\واژه{فریم.ورک} پیشنهادی ما، حول ایده‌ی جداسازی \واژه{سناریو}‌ها از مصادیق صحت‌سنجی آن شکل گرفته است. در این معماری، \واژه{سناریو}‌هایی توسط برنامه‌نویس تعبیه می‌شود که مطابق تعریف، انتظار می‌رود همواره برقرار باشند. مجموعه‌ی این \واژه{سناریو}‌ها را به عنوان یک لایه از معماری \واژه{فریم.ورک} \واژه{تست} خود در نظر می‌گیریم و آن را «لایه‌ی \واژه{سناریو}» می‌نامیم. در این لایه، نویسنده‌ی \واژه{تست} تنها به توصیف نیازمندی‌های صحت‌سنجی برنامه می‌پردازد و دغدغه‌ی تولید مصادیق \واژه{سناریو}‌ها را نخواهد داشت.

با توجه به تعریف، خود \واژه{سناریو}‌ها تنها مجموعه‌ای از گزاره‌ها هستند که در یک پیاده‌سازی درست، برقرار خواهند بود. حال آن که سنجش صحت این گزاره‌ها می‌تواند \واژه{متریک} خوبی برای کیفیت کد باشد. در \واژه{فریم.ورک} پیشنهادی، لایه‌ی دیگری از معماری \واژه{تست} به این موضوع تخصیص دارد که آن را «لایه‌ی \واژه{اکتور}» می‌نامیم. در واقع، لایه‌ی \واژه{اکتور} موظف است تا با تغییر مداوم وضعیت سامانه، \واژه{پری.کاندیشن}‌های \واژه{سناریو}‌های مختلف را فراهم کند تا صحت آن‌ها سنجیده شود.

دو لایه‌ی \واژه{سناریو} و \واژه{اکتور}، ارکان اصلی معماری پیشنهادی جهت پیاده‌سازی \واژه{فریم.ورک} \واژه{تست} هستند.

\قسمت{فواید}
\برچسب{sec:adv}

\زیرقسمت{کاهش حجم \واژه{تست}‌ها}
\برچسب{sec:loc}
یکی از دلایل عمده‌ی عدم علاقه‌ی برنامه‌نویس‌ها به تألیف \واژه{تست}، حجم زیاد \واژه{تست} در مقایسه با میزان کد پوشیده شده توسط آن است. هر چه سطح \واژه{تست} به سطح سیستمی نزدیک‌تر باشد، حجم این کد نیز به دلیل آماده کردن شرایط اولیه‌ی \واژه{تست} افزایش می‌یابد.

با توجه به این که در معماری پیشنهادی، تعریف \واژه{سناریو}‌ها مستقل از داده‌ها انجام می‌شود، امکان \واژه{reuse} از داده‌های یکسان برای \واژه{تست}‌های متفاوت وجود دارد که این موضوع منجر به کاهش حجم \واژه{تست}‌ها می‌شود.

\زیرقسمت{کاهش هزینه‌ی نگه‌داری}
از آن جا که تغییر سریع در بسیاری از \واژه{سیستم}‌های نرم‌افزاری ضروری است، حجم زیاد \واژه{تست}‌ها (ر.ک. \رجوع{sec:loc}) در هنگام تغییر نیز نیاز به هم‌گام‌سازی دارند. در معماری تک‌لایه، با توجه به اینکه شرایط اجرای \واژه{تست} توسط خود آن فراهم می‌شود، \واژه{coupling} به \واژه{component}‌های خارج از محدوده‌ی \واژه{تست} وجود دارد و هنگام تغییر رفتار هر \واژه{component}، این تغییرات در میان تمام \واژه{تست}‌های درگیر انتشار می‌یابد.

با توجه به جدا شدن لایه‌ی \واژه{سناریو} و \واژه{اکتور} در معماری پیشنهادی، تغییرات محدود به بخشی از \واژه{تست}‌ها خواهند بود که واقعاً تغییر کرده و به بخش‌های دیگر انتشار نمی‌یابند.


\زیرقسمت{افزایش احتمال یافتن خطا}
یکی از تفاوت‌های اساسی معماری پیشنهادی و معماری‌های تک‌لایه در کلی بودن تعریف \واژه{سناریو}‌هاست؛ به این معنا که یک \واژه{سناریو} می‌تواند روی داده‌های مختلفی قابل اعمال باشد. بنابراین، هر \واژه{سناریو} عملاً معادل یک مجموعه‌ی \واژه{تست} عمل می‌کند.

این عمومی بودن تعریف \واژه{سناریو} کمک می‌کند تا صحت آن نه فقط روی یک داده، بلکه برای چندین حالت متفاوت بررسی شود. این موضوع احتمال یافتن خطاهای موجود را افزایش داده و در نتیجه موجب افزایش تأثیرگذاری فرآیند \واژه{تست} می‌گردد.

\قسمت{در سطوح مختلف \واژه{تست}}
اگر چه معماری پیشنهادی ما مستقل از سطح \واژه{تست} است، با این حال تأثیر مثبت آن در تألیف \واژه{سیستم.تست} مشهودتر است. در این لایه به دلیل سطح بالا بودن تست‌ها، حجم

\قسمت{استفاده خارج از محیط \واژه{تست}}
انجام \واژه{تست} تنها یکی از روش‌های افزایش کیفیت نرم‌افزار است. ادعا می‌کنیم که معماری پیشنهادی خارج از محیط \واژه{تست} نیز می‌تواند به افزایش کیفیت نرم‌افزار کمک کند.

در هنگام \واژه{تست}، هدف یافتن \واژه{defect} نرم‌افزار، پیش از عملیاتی شدن آن است. اما بسیاری از \واژه{defect}‌های نرم‌افزاری حتی پس از عملیاتی شدن نیز از دیده نهان می‌مانند.

در \واژه{فریم.ورک} ارائه‌شده، می‌توان به جای لایه‌ی اکتور، کاربران حقیقی \واژه{سیستم} را قرار داد تا از نرم‌افزار استفاده کنند. به این ترتیب، \واژه{سیستم} می‌تواند عملکرد خودش را (حتی در حالی که عملیاتی است) ارزیابی کند و برخی از این \واژه{defect}‌ها را یافته، و جهت رسیدگی توسعه‌دهندگان گزارش کند. از جمله \واژه{defect}‌هایی که به خوبی به این روش پیدا می‌شوند، بروز \واژه{inconsistency} در مقادیر محاسبه‌شدنی است.

\قسمت{مقایسه با مفاهیم دیگر}
\زیرقسمت{در مقایسه با contract}

\زیرقسمت{در مقایسه با mutational testing}






\فصل{نتیجه‌گیری و کارهای آتی}
\برچسب{chap:future}

\قسمت{جمع‌بندی}
در طول این پروژه مجموعاً فعّالیت‌های زیر صورت گرفت:
\شروع{enumerate}
\فقره بررسی کارهای پیشین در زمینه‌ی توسعه‌ی \واژه{تست}
\فقره ارائه‌ی معماری پیشنهادی با ایده گرفتن از نقاط قوّت و ضعف مشهود در روش‌های پیشین
\فقره مقایسه‌ی این معماری با روش‌های قبلی
\فقره پیاده‌سازی یک \واژه{فریم.ورک} \واژه{تست} مبتنی بر این معماری
\فقره پیاده‌سازی یک نرم‌افزار نمونه و تألیف \واژه{تست} برای آن توسّط \واژه{فریم.ورک} پیشنهادی
\پایان{enumerate}

نهایتاً برآورد مؤلّفان این بود که استفاده از معماری پیشنهادی، همان‌طور که انتظار می‌رفت، کمک شایانی به ساده‌سازی انجام \واژه{تست} نرم‌افزار نمونه کرد. انتظار می‌رود که استفاده از این معماری در \واژه{تست} نرم‌افزارهای بزرگتر، مؤثّرتر نیز واقع شود.

\قسمت{کارهای آتی}
در طول انجام پروژه، ایده‌های فراوانی مبتنی بر معماری پیشنهادی به ذهنمان رسید که در راستای خارج نشدن از حوزه‌ی این پروژه وارد آن‌ها نشدیم، امّا بررسی آن‌ها خالی از لطف نخواهد بود:

\شروع{itemize}
\فقره حذف و هرس \واژه{سناریو}ها پس از بررسی صحّت آن‌ها به دفعات کافی
\فقره تعریف \واژه{متریک}های جدید سنجش کیفیّت نرم‌افزار بر حسب \واژه{سناریو}ها
\فقره استفاده از \واژه{فریم.ورک} \واژه{تست} پیاده‌شده در یک پروژه‌ی متن‌باز و بزرگ‌تر، جهت ارزیابی دقیق‌تر این روش
\فقره پیاده‌سازی \واژه{اکتور} به شیوه‌های مختلف و بررسی میزان اثرگذاری آن‌ها
\پایان{itemize}





\PrepareForBibliography

\setlatintextfont[Scale=1]{Linux Libertine}
\setlength{\baselineskip}{0.8cm}
%\setromantextfont[Scale=1.2]{XB Niloofar}

%\bibliographystyle{IEEEtran}
%\bibliographystyle{is-unsrt}
%\bibliographystyle{ieeetr-fa}
%\bibliographystyle{amsplain}

%\bibliography{resources/resources}
\latin
\printbibliography[title=\bibliographytitle,heading=bibintoc]
\persian

% glossaries
{\cleardoublepage\setlength{\baselineskip}{1cm}\printpersianglossary\cleardoublepage\printenglishglossary}


\PrepareForLatinPages
\date{August 2016}
\logo{\includegraphics[scale=.4]{logo-en}}
\title{\sffamily\enTitle}
% uncomment following lines only if you have defined commands for two-lines-title at the beginning of this file
%\titlelineone{\enTitleLineOne}
%\titlelinetwo{\enTitleLineTwo}
\author{\sffamily\enAuthor}
\university{\normalfont\bfseries Sharif University of Technology\\Computer Engineering Department}
\subject{Computer Software Engineering}
\supervisor{\sffamily Dr. Seyed Hassan Mirian Hosseinabadi}
\consult{\sffamily Mostafa Mahdieh}
\begin{abstract}{\enKeywords}
Testing is one of the main means of achieving higher quality of software. Behavior-Driven Development is an industry-wide popular techniques in effectively integrating testing with development process.

In this article, we will first examine BDD and other similar techniques in software testing.

Next, we will propose and implement a BDD testing framework, designed based on the idea of decoupling scenarios from tests. This framework should supposedly help the software developer in writing higher quality tests, as well as decreasing the maintenance cost of such test suites.

And at last, we will implement a sample system, and use our testing framework toward writing tests for our sample application.


\end{abstract}
\makethesistitle
\پایان{نوشتار}

%%% Local Variables:
%%% mode: latex
%%% TeX-master: t
%%% End:
