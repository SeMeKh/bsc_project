\usepackage{amssymb}
\usepackage{mathrsfs}
\usepackage{algorithmicx}
\usepackage{graphicx}
\usepackage{multirow}
\usepackage{comment}
\usepackage[style=ieee,backend=biber]{biblatex}
\newcommand{\bibliographytitle}{\rl{کتاب‌نامه}}
\usepackage{paralist}
\usepackage{textcomp}
\usepackage{ctable}% for tables (it provides table-notes and imports booktabs package too)

\newcounter{tablerow}
\renewcommand{\arraystretch}{1.5}
\usepackage{cleveref}
\newcommand{\crefrangeconjunction}{--}
\crefformat{tablerow}{#2#1#3}
\crefmultiformat{tablerow}{#2#1#3}%
{ و~#2#1#3}%
{, #2#1#3}%
{ و~#2#1#3}
\crefformat{section}{#2#1#3}
\crefmultiformat{section}{#2#1#3}%
{ و~#2#1#3}%
{, #2#1#3}%
{ و~#2#1#3}



\newcommand{\URL}%
{یو.آر.ال.}
\newcommand{\postgresql}%
{پُستگِرِس.کیو.اِل.}
\newcommand{\booktabs}%
{بوک‌تَبز}
\نوواژه[تست.ن.ا.]{آزمون نرم‌افزار}{Software Test}
\نوواژه[تست]{آزمون}{Test}
\نوواژه[یونیت.تست]{آزمون واحد}{Unit Test}
\نوواژه[اکسپتنس.تست]{آزمون پذیرش}{Acceptance Test}
\نوواژه[اکسپتنس.کریتریا]{شرط پذیرش}{Acceptance Criteria}
\نوواژه[هرم.تست]{هرم آزمون}{Test Pyramid}
\نوواژه[تست.سویت]{مجموعه‌ی آزمون‌ها}{Test Suite}
\نوواژه[سیستم.تست]{آزمون سامانه}{System Test}
\نوواژه[سیستم]{سامانه}{System}
\نوواژه[ماژول]{مدول}{Module}
\نوواژه{کلاس}{Class}
\نوواژه{قالب}{Template}
\نوواژه[تست.کیس]{مورد آزمون}{Test Case}
\نوواژه[وی.مدل]{مدل چرخه عمر \چر{V}-شکل}{V-Shaped Software Lifecycle Model}
\نوواژه[مدل.وی]{مدل وی}{V Model}
\نوواژه[یو.آی.]{رابط کاربری}{User Interface}
\نوواژه[اینترفیس]{واسط}{Interface}
\نوواژه[بیزنس.دامین]{حوزه تجاری}{Business Domain}
\نوواژه[بیزنس.ولیو]{ارزش تجاری}{Business Value}
\نوواژه[ریفکتورینگ]{فاکتوربندی مجدد}{Refactoring}
\نوواژه[بیانیه.اجایل]{بیانیه ی توسعه ی چابک نرم افزار}{Manifesto for
Agile Software Development}
\نوواژه[اس.دی.ام.]{متدولوژی توسعه ی نرم افزار}{Software Development Methodology}
\نوواژه{ذینفع}{Stakeholder}
\نوواژه{فعالیت}{Activity}
\نوواژه{قرمز}{Fail}
\نوواژه{سبز}{Success}
\نوواژه[پی.پی.]{برنامه‌نویسی دونفره}{Pair Programming}
\نوواژه[سی.آی.]{یکپارچه‌سازی مداوم}{Continuous Integration}
\نوواژه[اینتگریشن]{یکپارچه‌سازی}{Integration}
\نوواژه[اینتگریشن.تست]{آزمون یکپارچه‌سازی}{Integration Test}
\نوواژه[فانکشنال]{کارکردی}{Functional}
\نوواژه[زبان.مدلینگ]{زبان مدل‌سازی}{Modeling Language}
\نوواژه[سافتویر.اینتنسیو]{نرم‌افزار-متمرکز}{Software-intensive}
\نوواژه[دیباگ]{رفع‌باگ}{Debug}
\نوواژه[آرتیفکت]{ساخته}{Artifact}
\نوواژه[سینتکس]{نحوی}{Syntax}
\نوواژه[سمنتیکس]{معنایی}{Semantics}
\نوواژه[وفق.پذیری]{وفق‌پذیری}{Adaptability}
\نوواژه[باگ]{باگ}{Bug}
\نوواژه[کانونشن]{عرف}{Convention}
\نوواژه[متدولوژی]{متدولوژی}{Methodology}
\نوواژه[تکنیک]{تکنیک}{Technique}
\نوواژه[پرکتیس]{تجربه}{Practice}
\نوواژه[اتکا.پذیری]{اتکاپذیری}{Reliability}
\نوواژه[ایکس.پی.]{اکس.پی.}{Extreme Programming}
\نوواژه[کیفیت.ن.ا.]{کیفیت نرم‌افزار}{Software Quality}
\نوواژه[اعتبار.سنجی]{اعتبار سنجی}{Validation}
\نوواژه{وارسی}{Verification}
\نوواژه{تحلیل}{Analysis}
\نوواژه{تکراری}{Iterative}
\نوواژه{تکرار}{Iteration}
\نوواژه{طراحی}{Design}
\نوواژه{توصیف}{Specification}
\نوواژه[اس.دی.پی.]{فرآیند توسعه‌ی نرم‌افزار}{Software Development
Process}
\نوواژه{تفکیک}{Decomposition}
\نوواژه[اسکوپ]{محدوده}{Scope}
\نوواژه[ای.اس.دی.]{توسعه‌ی چابک نرم‌افزار}{Agile Software Development}
\نوواژه{نگهداری}{Maintenance}
\نوواژه[پیاده.سازی]{پیاده‌سازی}{Implementation}
\نوواژه[مدل.آبشاری]{مدل آبشاری}{Waterfall Model}
\نوواژه[ای.ام.دی.دی.]{توسعه‌ی چابک مدل-رانه}{Agile Model Driven Development}
\نوواژه[بی.دی.دی.]{توسعه‌ی رفتار-رانه}{Behavior Driven Development}
\نوواژه{رفتار}{Behavior}
\نوواژه[فیچر]{ویژگی}{Feature}
\نوواژه[فیچر.ست]{مجموعه‌ی ویژگی}{Feature Set}
\نوواژه[نتیجه.تجاری]{نتیجه تجاری}{Business Outcome}
\نوواژه[تی.دی.دی.]{توسعه‌ی آزمون-رانه}{Test Driven Development}
\نوواژه[تی.اف.دی.]{توسعه‌ی اول-آزمون}{Test First Development}
\نوواژه[ای.تی.دی.دی.]{توسعه‌ی آزمون پذیرش-رانه}{Acceptance Test Driven
Development}
\نوواژه[پری.کاندیشن]{پیش-شرط}{Pre-condition}
\نوواژه[پست.کاندیشن]{پسا-شرط}{Post-condition}
\نوواژه[دان]{انجام‌شده}{Done}
\نوواژه[یوزر.استوری]{داستان کاربر}{User Story}
\نوواژه[یوزر.استوری.ها]{داستان‌های کاربر}{User Stories}
\نوواژه[دامین.اکسپرت]{خبره حوزه}{Domain Expert}
\نوواژه[پلین.تکست]{متن ساده}{Plain Text}

\نوواژه{سناریو}{Scenario}
\نوواژه[given]{اگر}{Given}
\نوواژه[when]{وقتی}{When}
\نوواژه[then]{آن‌گاه}{Then}
\نوواژه[given-when-then]{\واژه{given}-\واژه{when}-\واژه{then}}{Given-When-Then}

\نوواژه{متریک}{Metric}
\نوواژه[کلاسیک]{کلاسیک}{Classic}
\نوواژه[اکتور]{تعامل‌گر}{Actor}
\نوواژه[coupling]{وابستگی}{Coupling}
\نوواژه[component]{مؤلفه}{Component}
\نوواژه[reuse]{استفاده‌ی مجدد از کد}{Code Reuse}
\نوواژه[پایتون]{زبان برنامه‌نویسی پایتون}{Python Programming Language}
\نوواژه[فریم.ورک]{چارچوب}{Framework}
\نوواژه[decorator]{دکوراتور}{Decorator}
\نوواژه[یوبی]{زبان مشترک}{Ubiquitous Language}

\نوواژه[inconsistency]{ناهم‌خوانی}{Inconsistency}
\نوواژه[defect]{نقص}{Defect}
\نوواژه[trigger]{محرک }{Trigger}

\نوواژه[insanity]{اینسنیتی}{Insanity}
\نوواژه[انتیتی]{موجودیت}{Entity}
\نوواژه{جنگو}{Django}
\نوواژه[کراد]{سی.آر.یو.دی.}{Create Read Update Delete}
\نوواژه[سرور]{کارگزار}{Server}
\نوواژه[ادد]{افزودن}{Add}



\newcommand{\myrotate}[3][]{\rotatebox{90}{\parbox[c][#1]{#2}{\centering\arraybackslash\rl{#3}}}}
\newcommand{\multilinescell}[2][c]{\begin{tabular}[#1]{@{}c@{}}#2\end{tabular}}
\newcommand{\twolinescell}[3][c]{\multilinescell[#1]{#2\\#3}}
\newcommand{\itemrl}[1][]{\item[\rl{#1}]}
\eqcommand{چرخش}{myrotate}
\eqcommand{سلولچندخطی}{multilinescell}
\eqcommand{سلول‌دوخطی}{twolinescell}
\eqcommand{فقره‌راست}{itemrl}



\newcommand{\Eqn}[1]%
{فرمول~(#1) }
\newcommand{\Eqns}[1]%
{فرمول‌های~(#1) }

\eqcommand{فرمول}{Eqn}
\eqcommand{فرمولهای}{Eqns}

\فرمان‌نو{\نگا}{ن.بـ.}
\فرمان‌نو{\نگاص}[1]{\نگا{} صفحه‌ی~\رجوع‌صفحه{#1}}
\فرمان‌نو{\آیم}[1][$i$]%
{#1اُم}

\newcolumntype{C}[1]{>{\centering\arraybackslash}m{#1}}

\فرمان‌نو{\نامک}[1]{%
\fcolorbox{red}{yellow}{\begin{minipage}{\textwidth}#1\end{minipage}}}

\newcommand{\tobewritten}[1]{\نامک{%
این جعبه، باید با متن متناظر با \چر{#1} جای‌گزین گردد.

برای دریافتن معنی \چر{#1} به پرونده‌ی \چر{TODO} نگاه کنید.
}}

\graphicspath{{img/}}% tell tex engine address of folder containing your pictures


\hypersetup{
	pdftitle = {\enTitle},
	pdfauthor = {\enAuthor},
	pdfsubject = {\enSubject},
	pdfkeywords = {\enKeywords}
}
