\فصل{مقدمه} \برچسب{chap:intro} 

\قسمت{اهمیت موضوع}
در عصر اینترنت، سرعت تحول نرم‌افزار موضوعی بسیار مهم است. تا اواخر قرن
بیستم، هر پروژه‌ی نرم‌افزاری برای چندین سال بدون تغییرات عمده استفاده
می‌شد و اجرای هر فاز از پروژه، چند ماه به‌طول می‌انجامید. اما
امروزه، روش‌های \واژه{ای.اس.دی.} در بسیاری از پروژه‌های نرم‌افزاری به‌کار
گرفته می‌شوند؛ مدت زمان اجرای کل پروژه به چند ماه و مدت زمان اجرای هر
فاز، به چند هفته یا حتی چند روز کاهش پیدا کرده و مقاومت پروژه‌های
نرم‌افزاری در مقابل تغییرات بسیار کاهش یافته است \مرجع{andres2004extreme}.

با این نرخ بالای تغییرات، مستندات پروژه خیلی سریع منسوخ می‌شوند؛ به‌روز
نگه داشتن توصیف نیازمندی‌ها و اجرای \واژه{تست} به شیوه‌های سنتی، بسیار
پرهزینه محسوب شده و عملا بی‌فایده است. در روش‌های \واژه{ای.اس.دی.}،
بیشتر از \نام{تولید درست محصول}{Building the product right}، به
\نام{تولید محصول درست}{Building the right product} توجه می‌شود و لازم
است هزینه‌ی مستندسازی و \واژه{تست} را کاهش داده و بر تولید محصول درست
تمرکز شود \مرجع{adzik2011sbe}.

در تکنیک \واژه{بی.دی.دی.}، فرآیند توصیف نیازمندی‌ها به‌صورت
\واژه{تکراری} انجام می‌شود و با توجه به زبان مشترک توصیف نرم‌افزار بین مشتری و
تیم توسعه و همچنین قابلیت تبدیل توصیف به \واژه{تست}، از توصیف به
عنوان مواد اولیه‌ی \واژه{تست} و
مستندات پروژه نیز بهره گرفته می‌شود. این امر باعث کاهش هزینه‌ی نگهداری،
افزایش سرعت توسعه \واژه{تست} و انطباق‌پذیری با تغییرات با وجود حفظ
کیفیت نرم‌افزار می‌شود. همچنین وجود چارچوبی کارا برای پیاده‌سازی این
\واژه{تکنیک}، بر سرعت توسعه و به‌کارگیری صحیح اصول این \واژه{تکنیک}
تأثیر مستقیم دارد.

\قسمت{تعریف مسأله} \واژه{بی.دی.دی.} یکی
از \واژه{تکنیک}‌های جدید برای \واژه{ای.اس.دی.} است که استفاده از آن در
صنعت نرم‌افزار رواج دارد \مرجع{cucumberusing}. این \واژه{تکنیک} بر
مراحل مختلف \واژه{اس.دی.پی.} از جمله برنامه‌ریزی، تحلیل، طراحی،
پیاده‌سازی و \واژه{تست} تاثیرگذار است (ر.ک. به
\رجوع{sec:bdd-props}). از جمله مسأله‌هایی که در مرحله‌ی \واژه{تست} از
فرآیند \واژه{بی.دی.دی.} وجود دارد، این است که با توجه به اینکه
\واژه{اکسپتنس.تست} در این روش، مستقیماً از روی توصیف بدست می‌آید،
لازم است هنگام توصیف، جزییاتی را تعیین نماییم که در مرحله‌ی توصیف
کاربردی ندارند، بلکه در مرحله‌ی تولید \واژه{اکسپتنس.تست} از روی توصیف
به آن‌ها نیاز داریم. این در حالی است که توصیه می‌شود جزییات غیر ضروری
در توصیف ذکر نشوند \مرجع{adzik2011sbe}.

برای مثال، در یک سامانه‌ی انتخاب واحد، ممکن است برای توصیف قابلیت
ثبت‌نام کاربر در یک ارائه از یک درس، صرف وجود یک دانشجو برای توصیف عمل
انتخاب واحد کافی باشد؛ اما برای تولید \واژه{اکسپتنس.تست} از روی توصیف
این قابلیت، لازم است جزییاتی غیر ضروری مانند «نام» و «نام خانوادگی» از دانشجو
در توصیف این قابلیت ذکر شوند. در مستندات یکی از چارچوب‌های فعلی 
\واژه{بی.دی.دی.} \مرجع{cucumberusing} نیز به این مسأله اشاره شده است.

\قسمت{اهداف پروژه}
ایده‌ی ابتداییِ این پروژه از همکاری مؤلفان در یک پروژه‌ی صنعتی شکل گرفت. به عنوان توسعه‌دهندگان \واژه{سیستم}‌ی مذکور، همیشه حس می‌کردیم که بخش قابل توجهی از کدهایی که برای \واژه{تست} نرم‌افزار می‌نویسیم شباهت‌های زیادی به یک‌دیگر دارند.
از سوی دیگر، به دلیل تغییر نیازمندی‌های \واژه{سیستم}، همیشه لازم بود تا \واژه{تست}‌ها به‌روزرسانی شوند و این تغییرات معمولاً در جای‌جای پروژه منتشر می‌شد. تمام این مشکلات باعث می‌شد تا کیفیت \واژه{تست}‌ها در طول زمان کاهش یافته و اثر مثبت آن‌ها در فرآیند توسعه‌ی نرم‌افزار کاهش یابد.

بنابراین سعی کردیم الگوهای مشترکی بین \واژه{تست}‌ها پیدا کنیم و با
\واژه{ریفکتورینگ} مکرر آن‌ها به ساختاری رسیدیم که بسیاری از مشکلات فوق
را کم‌رنگ کرده بود. در این ساختار، برای \واژه{تست} هر قسمت، فقط
اطلاعاتی که مرتبط با عملکرد همان قسمت بودند حفظ می‌شد و جزییات مرتبط با
نیازمندی‌های \واژه{تست}، به بخش‌های دیگر منتقل شده بود. تلاش کردیم تا
ساختار نهایی را تعمیم دهیم تا در پروژه‌های دیگر نیز قابل استفاده
باشد. ساختار حاصل، الهام‌بخش \واژه{فریم.ورک} \واژه{تست}ی شد که در این
پروژه ارائه می‌کنیم.

هدف اصلی از ارائه‌ی \واژه{فریم.ورک} \واژه{تست} پیشنهادی، کاهش هزینه‌ی نگهداری \واژه{تست}ها و افزایش اثرگذاری آن‌ها در کیفیت نرم‌افزار است. در بخش \رجوع{sec:adv} فواید مطرح‌شده پس از استفاده از \واژه{فریم.ورک} مورد نظر را با دقت بیشتری بررسی خواهیم کرد.


\قسمت{ساختار پایان‌نامه}
این پایان‌نامه حاوی پنج فصل است.

در فصل نخست مقدمه‌ای از موضوع پایان‌نامه و اهمیت آن ارائه شده. در فصل دوم پیش‌زمینه‌ای از کارهای قبلی که در این زمینه صورت گرفته آمده، و در ادامه \واژه{بی.دی.دی.} به تفصیل شرح داده شده. فصل سوم \واژه{فریم.ورک} \واژه{تست} پیشنهادی را از جنبه‌ی معماری و پیاده‌سازی بررسی می‌کند. فصل چهارم به ارائه‌ی گزارشی از یک نمونه‌ی استفاده از \واژه{فریم.ورک} پیاده‌سازی‌شده در فصل سوم می‌پردازد. و نهایتاً در فصل پنجم جمع‌بندی و کارهای آتی که انجام آن‌ها در آینده امکان‌پذیر خواهد بود آمده.

%%% Local Variables:
%%% mode: latex
%%% TeX-master: "../main"
%%% End:
