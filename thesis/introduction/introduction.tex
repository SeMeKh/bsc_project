\فصل{مقدمه} \برچسب{chap:intro} \قسمت{تعریف مسأله} \واژه{بی.دی.دی.} یکی
از \واژه{تکنیک}‌های جدید \واژه{تست} نرم‌افزار است که استفاده از آن در صنعت
نرم‌افزار رواج زیادی دارد. با توجه به نو بودن این \واژه{متدولوژی}،
توصیف دقیق و مشخصی از اجزاء و حوزه‌ی تأثیرگذاری آن وجود ندارد. علی‌رغم
پیاده‌سازی‌های مختلفی که جهت اعمال این \واژه{متدولوژی} وجود دارد،
راه‌کارهای پیشین در جزئیات تفاوت‌های قابل توجهی با تعاریف
\واژه{بی.دی.دی.} دارند که موجب می‌شود استفاده از آن‌ها توسعه‌دهنده را از
برخی ویژگی‌های مثبت \واژه{بی.دی.دی.} محروم کند.

\قسمت{اهمیت موضوع}
در عصر اینترنت، سرعت تحول نرم‌افزار موضوعی بسیار مهم است. تا اواخر قرن
بیستم، هر پروژه‌ی نرم‌افزاری برای چندین سال بدون تغییرات عمده استفاده
می‌شد و اجرای هر فاز از اجرای پروژه، چند ماه به‌طول می‌انجامید. اما
امروزه، روش‌های \واژه{ای.اس.دی.} در بسیاری از پروژه‌های نرم‌افزاری به‌کار
گرفته می‌شوند؛ مدت زمان اجرای کل پروژه به چند ماه و مدت زمان اجرای هر
فاز، به چند هفته یا حتی چند روز کاهش پیدا کرده و مقاومت پروژه‌های
نرم‌افزاری در مقابل تغییرات بسیار کاهش یافته است \مرجع{andres2004extreme}.

با این نرخ بالای تغییرات، مستندات پروژه خیلی سریع منسوخ می‌شوند؛ به‌روز
نگه داشتن توصیف نیازمندی‌ها و اجرای \واژه{تست} به شیوه‌های سنتی، بسیار
پرهزینه محسوب شده و عملا بی‌فایده است. در روش‌های \واژه{ای.اس.دی.}،
بیشتر از \نام{تولید درست محصول}{Building the product right}، به
\نام{تولید محصول درست}{Building the right product} توجه می‌شود و لازم
است هزینه‌ی مستندسازی و \واژه{تست} را کاهش داده و بر تولید محصول درست
تمرکز شود \مرجع{adzik2011sbe}.

در تکنیک \واژه{بی.دی.دی.}، فرآیند توصیف نیازمندی‌ها به‌صورت
\واژه{تکراری} است و با توجه به زبان مشترک توصیف نرم‌افزار بین مشتری و
تیم توسعه و همچنین قابل اجرا بودن آن، از توصیف به‌عنوان \واژه{تست} و
مستندات پروژه نیز بهره گرفته می‌شود. این امر باعث کاهش هزینه‌ی نگهداری،
افزایش سرعت توسعه \واژه{تست} و انطباق‌پذیری با تغییرات با وجود حفظ
کیفیت نرم‌افزار می‌شود.

\قسمت{اهداف پروژه}
ایده‌ی این پروژه از همکاری مؤلفان در یک پروژه‌ی شرکتی شکل گرفت. به عنوان توسعه‌دهندگان \واژه{سیستم} مذکور، همیشه حس می‌کردیم که بخش قابل توجهی از کدهایی که برای \واژه{تست} نرم‌افزار می‌نویسیم شباهت‌های زیادی به یک‌دیگر دارند.
از سوی دیگر، به دلیل تغییر نیازمندی‌های \واژه{سیستم}، همیشه لازم بود تا \واژه{تست}‌ها به‌روزرسانی شوند و این تغییرات معمولاً در جای‌جای پروژه منتشر می‌شد. تمام این مشکلات منجر می‌شد تا کیفیت \واژه{تست}‌ها در طول زمان کاهش یافته و اثرگذاری خود را از دست دهند.

بنابراین سعی کردیم الگوهای مشترکی بین \واژه{تست}‌ها پیدا کنیم و با \واژه{ریفکتورینگ} مکرر آن‌هابه ساختاری رسیدیم که بسیاری از مشکلات فوق را کم‌رنگ کرده بود. تلاش کردیم تا ساختار نهایی را تعمیم دهیم تا قابل استفاده در پروژه‌های دیگر نیز باشد. نتیجه \واژه{فریم.ورک} \واژه{تست}ی شد که در این پروژه ارائه می‌کنیم.

هدف اصلی از ارائه‌ی \واژه{فریم.ورک} \واژه{تست} پیشنهادی، کاهش هزینه‌ی نگهداری \واژه{تست}ها و افزایش اثرگذاری آن‌ها در کیفیت نرم‌افزار است. در بخش \رجوع{sec:adv} فواید مطرح‌شده پس از استفاده از \واژه{فریم.ورک} مورد نظر را با دقت بیشتری بررسی خواهیم کرد.

همچنین، اگر چه ایده‌هایی کم و بیش شبیه معماری پیشنهادی پیش از این نیز ارائه شده‌اند، بخشی از پروژه‌ی ما به انجام پیاده‌سازی ابزارهای مورد نیاز پرداخته است؛ چرا که در حوزه‌هایی مانند \واژه{تست} که در صنعت شدیداً مورد استفاده قرار می‌گیرند، ارائه‌ی ابزار در کنار یک روش و معماری نیز حائز اهمیت است.


\قسمت{ساختار پایان‌نامه}
این پایان‌نامه حاوی پنج فصل است.

در فصل نخست مقدمه‌ای از موضوع پایان‌نامه و اهمیت آن ارائه شده. در فصل دوم پیش‌زمینه‌ای از کارهای قبلی که در این زمینه صورت گرفته آمده، و در ادامه \واژه{بی.دی.دی.} به تفصیل شرح داده شده. فصل سوم \واژه{فریم.ورک} \واژه{تست} پیشنهادی را از جنبه‌ی معماری و پیاده‌سازی بررسی می‌کند. فصل چهارم به ارائه‌ی گزارشی از یک نمونه‌ی استفاده از \واژه{فریم.ورک} پیاده‌سازی‌شده در فصل سوم می‌پردازد. و نهایتاً در فصل پنجم جمع‌بندی و کارهای آتی که انجام آن‌ها در آینده امکان‌پذیر خواهد بود آمده.

%%% Local Variables:
%%% mode: latex
%%% TeX-master: "../main"
%%% End:
