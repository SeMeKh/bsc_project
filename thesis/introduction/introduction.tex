\فصل{مقدمه}
\برچسب{chap:intro}
	
فرآیند \واژه{تست.ن.ا.} از اول پیدایش تا کنون با رویکرد‌ها و کاربردهای
متنوعی اجرا شده است. ؟؟؟

\قسمت{تعریف موضوع پروژه}
در این پایان‌نامه بر رویکرد بهره‌گیری از توصیف
نرم‌افزار به زبان مشتری در \واژه{تست} تمرکز شده و معماری‌ جدیدی برای
\واژه{تست} با \واژه{تکنیک} \واژه{بی.دی.دی.} بر اساس این رویکرد پیشنهاد می‌شود.
در ادامه، یک \واژه{فریم.ورک} \واژه{تست} بر
حسب این معماری پیاده‌سازی شده، و نهایتاً از این \واژه{فریم.ورک} جهت تألیف
\واژه{تست} برای یک سامانه‌ی کوچک استفاده می‌کنیم.

گرچه ایده‌ی مطرح شده در معماری پیشنهادی منحصر به \واژه{تکنیک}
\واژه{بی.دی.دی.} نیست و با سایر \واژه{تکنیک}‌ها و \واژه{متدولوژی}‌ها نیز
قابل انطباق است، حوزه‌ی بررسی‌این پایان‌نامه به \واژه{تکنیک}
\واژه{بی.دی.دی.} محدود شده است.

\قسمت{اهمیت موضوع پروژه}
انجام \واژه{تست} در حوزه‌ی نرم‌افزار، یکی از عوامل کلیدیِ افزایش کیفیت نرم‌افزار به شمار می‌رود. بنابراین افزایش کیفیت \واژه{تست} در یک نرم‌افزار، به طور غیرمستقیم موجب افزایش کیفیت خود نرم‌افزار خواهد شد. در بخش \رجوع{sec:adv} دلایل افزایش اثرگذاری و کاهش هزینه‌ی انجام فرآیند \واژه{تست} پس از اعمال معماریِ پیشنهادی را با دقت بیشتری بررسی خواهیم کرد.

همچنین، اگر چه ایده‌هایی کم و بیش شبیه معماری پیشنهادی پیش از این نیز ارائه شده‌اند، بخشی از پروژه‌ی ما به انجام پیاده‌سازی ابزارهای مورد نیاز پرداخته است؛ چرا که در حوزه‌هایی مانند \واژه{تست} که در صنعت شدیداً مورد استفاده قرار می‌گیرند، ارائه‌ی ابزار در کنار یک روش و معماری نیز از اهمیت زیادی برخوردار است.



\قسمت{انگیزه‌ی انجام پروژه}
ایده‌ی این پروژه از همکاری مؤلفان در یک پروژه‌ی شرکتی شکل گرفت. به عنوان توسعه‌دهندگان \واژه{سیستم} مذکور، همیشه حس می‌کردیم که بخش قابل توجهی از کدهایی که برای \واژه{تست} نرم‌افزار می‌نویسیم شباهت‌های زیادی به یک‌دیگر دارند.
از سوی دیگر، به دلیل تغییر نیازمندی‌های \واژه{سیستم}، همیشه لازم بود تا \واژه{تست}‌ها به‌روزرسانی شوند و این تغییرات معمولاً در جای‌جای پروژه منتشر می‌شد. تمام این مشکلات منجر می‌شد تا کیفیت \واژه{تست}‌ها در طول زمان کاهش یافته و اثرگذاری خود را از دست دهند.

بنابراین سعی کردیم الگوهای مشترکی بین \واژه{تست}‌ها پیدا کنیم و با \واژه{ریفکتورینگ} مکرر آن‌هابه ساختاری رسیدیم که بسیاری از مشکلات فوق را کم‌رنگ کرده بود. تلاش کردیم تا ساختار نهایی را تعمیم دهیم تا قابل استفاده در پروژه‌های دیگر نیز باشد. نتیجه معماری‌ای شد که در این پروژه ارائه می‌کنیم.


\قسمت{ساختار پایان‌نامه}
این پایان‌نامه حاوی پنج فصل است.

در فصل نخست مقدمه‌ای از موضوع پایان‌نامه و اهمیت آن ارائه شده. در فصل دوم پیش‌زمینه‌ای از کارهای قبلی که در این زمینه صورت گرفته آمده. فصل سوم معماری پیشنهادی را در سطح نظری بررسی می‌کند. فصل چهارم به ارائه‌ی گزارشی از پیاده‌سازی عملی معماری مطرح‌شده در فصل سوم می‌پردازد. و نهایتاً در فصل پنجم نتیجه‌گیری و کارهای آتی که انجام آن‌ها در آینده امکان‌پذیر خواهد بود آمده.

%%% Local Variables:
%%% mode: latex
%%% TeX-master: "../main"
%%% End:
