\فصل{مقدمه}
\برچسب{chap:intro}
	
فرآیند \واژه{تست.ن.ا.} از اولین پیدایش تا کنون با رویکرد‌ها و کاربردهای
متنوعی اجرا شده‌است. در این پایان‌نامه بر رویکرد بهره‌گیری از توصیف
نرم‌افزار به زبان مشتری در \واژه{تست} تمرکز شده و معماری‌ جدیدی برای
\واژه{تست} با \واژه{تکنیک} \واژه{بی.دی.دی.}، که بر اساس این رویکرد
ارائه شده، پیشنهاد می‌شود.

گرچه ایده‌ی مطرح شده در معماری پیشنهادی منحصر به \واژه{تکنیک}
\واژه{بی.دی.دی.} نیست و با سایر \واژه{تکنیک}‌ها و \واژه{متدولوژی}‌ها نیز
قابل انطباق است، حوزه‌ی بررسی‌این پایان‌نامه به \واژه{تکنیک}
\واژه{بی.دی.دی.} محدود شده است.

\قسمت{تعریف موضوع پروژه}


\قسمت{اهمیت موضوع پروژه}
انجام \واژه{تست} در حوزه‌ی نرم‌افزار، یکی از عوامل کلیدیِ افزایش کیفیّت نرم‌افزار به شمار می‌رود. بنابراین افزایش کیفیّت \واژه{تست} در یک نرم‌افزار، به طور غیرمستقیم موجب افزایش کیفیّت خود نرم‌افزار خواهد شد.

در بخش \رجوع{sec:adv} دلایل افزایش اثرگذاری فرآیند \واژه{تست} و کاهش هزینه‌ی انجام آن را با دقّت بیشتری بررسی خواهیم کرد.



\قسمت{انگیزه انجام پروژه}
ایده‌ی این پروژه از همکاری مؤلّفان در یک پروژه‌ی شرکتی شکل گرفت. به عنوان توسعه‌دهندگان سیستم مذکور، همیشه حس می‌کردیم که بخش قابل توجّهی از کدهایی که برای \واژه{تست} نرم‌افزار می‌نویسیم شباهت‌های زیادی به یک‌دیگر دارند.
از سوی دیگر، به دلیل تغییر نیازمندی‌های سیستم، همیشه لازم بود تا \واژه{تست}ها به‌روزرسانی شوند و این تغییرات معمولاً در جای‌جای پروژه منتشر می‌شد. تمام این مشکلات منجر می‌شد تا کیفیّت \واژه{تست}ها در طول زمان کاهش یافته و اثرگذاری خود را از دست دهند.

بنابراین سعی کردیم الگوهای مشترکی بین \واژه{تست}ها پیدا کنیم و با \واژه{ریفکتورینگ} مکرّر آن‌هابه ساختاری رسیدیم که بسیاری از مشکلات فوق را کم‌رنگ کرده بود. تلاش کردیم تا ساختار نهایی را تعمیم دهیم تا قابل استفاده در پروژه‌های دیگر نیز باشد. نتیجه معماری‌ای شد که در این پروژه ارائه می‌کنیم.



\قسمت{ساختار پایان‌نامه}
این پایان‌نامه حاوی پنج فصل است.

در فصل نخست مقدمّه‌ای از موضوع پایان‌نامه و اهمّیّت آن ارائه شده. در فصل دوم پیش‌زمینه‌ای از کارهای قبلی که در این زمینه صورت گرفته آمده. فصل سوم معماری پیشنهادی را در سطح نظری بررسی می‌کند. فصل چهارم به ارائه‌ی گزارشی از پیاده‌سازی عملی معماری مطرح‌شده در فصل سوم می‌پردازد. و نهایتاً در فصل پنجم نتیجه‌گیری و کارهای آتی که انجام آن‌ها در آینده امکان‌پذیر خواهد بود آمده.

%%% Local Variables:
%%% mode: latex
%%% TeX-master: "../main"
%%% End:
