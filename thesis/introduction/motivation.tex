\قسمت{انگیزه انجام پروژه}
ایده‌ی این پروژه از همکاری مؤلّفان در یک پروژه‌ی شرکتی شکل گرفت. به عنوان توسعه‌دهندگان سیستم مذکور، همیشه حس می‌کردیم که بخش قابل توجّهی از کدهایی که برای \واژه{تست} نرم‌افزار می‌نویسیم شباهت‌های زیادی به یک‌دیگر دارند.
از سوی دیگر، به دلیل تغییر نیازمندی‌های سیستم، همیشه لازم بود تا تست‌ها به‌روزرسانی شوند و این تغییرات . تمام این مشکلات منجر می‌شد تا کیفیّت تست‌ها در طول زمان کاهش یافته و اثرگذاری خود را از دست دهند.

بنابراین سعی کردیم الگوهای مشترکی بین تست‌ها پیدا کنیم و با \واژه{ریفکتورینگ} مکرّر تست‌هایی که داشتیم به ساختاری رسیدیم که بسیاری از مشکلات فوق را کم‌رنگ کرده بود. تلاش کردیم تا ساختار نهایی را تعمیم دهیم تا قابل استفاده در پروژه‌های دیگر نیز باشد. نتیجه معماری‌ای شد که در این پروژه ارائه می‌کنیم.

طبق بررسی‌هایی که کردیم اگر چه ایده‌های مشابهی وجود داشت، ...
