\قسمت{تألیف \واژه{سناریو} برای سامانه}
\برچسب{sec:edu-bdd}

جهت تألیف \واژه{تست}، ابتدا \واژه{یوزر.استوری.ها}ی مورد نظر نوشته شده و از آن‌ها \واژه{سناریو}‌هایی استخراج می‌شود. نتیجه‌ی این کار برای دو \واژه{سناریو}ی نمونه در زیر آمده:

\begin{minipage}{\textwidth}
\latin
\textbf{Story}: Enrollment of student in offering\\
\textbf{As a} student\\
\textbf{I want to} enroll in an offering\\
\textbf{So that} I am allowed to participate in an offering's classes\\
\\
\\
\textbf{Scenario 1}: Capacity should decrease by enroll\\
\textbf{Given}
	student s1, offering o1 with available capacity c0\\
\textbf{When}
	s1 commits enrollment in o1\\
\textbf{Then}
	available capacity of o1 should become c0-1\\
\\
\textbf{Scenario 2}: Enrollment should fail for offering with zero capacity\\
\textbf{Given}
	offering o1 with zero capacity\\
\textbf{When}
	someone enrolls in it\\                               
\textbf{Then}
	it should fail with error\\
\end{minipage}

گام بعدی انتقال \واژه{سناریو}‌ها از زبان \واژه{given-when-then} به نرم‌افزار است. باید توجه کرد که نباید در این مسیر از کلیت \واژه{سناریو}‌ها کاسته شود، که این موضوع در \واژه{فریم.ورک} پیشنهادی، با توجه به جدا شدن لایه‌ی \واژه{سناریو} از لایه‌ی \واژه{اکتور} به سادگی محقق می‌شود و تقریباً ترجمه‌ی عبارات به زبان برنامه‌نویسی مقصد کفایت می‌کند. نتیجه‌ی انجام این کار برای دو سناریوی مذکور در شکل‌های \رجوع{fig:scenario-code} آمده.

\شروع{شکل}[tbp]
\تنظیم‌ازوسط
\درج‌تصویر[پهنا=0.9\پهنای‌سطر]{scenario-code}
\تر\موقتم{
کد معادل سناریو
}
\شرح[\موقتم]{\موقتم}
\برچسب{fig:scenario-code}
\پایان{شکل}


%%% Local Variables:
%%% mode: latex
%%% TeX-master: "../main"
%%% End:
