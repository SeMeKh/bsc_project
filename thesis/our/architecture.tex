\قسمت{معماری \واژه{فریم.ورک}}

در بخش \رجوع{gwt} با نحوه‌ی بیان یک \واژه{سناریو} آشنا شدیم. یک سناریو، در واقع پاسخ صحیح یک زیرمسأله از مسأله‌ی اصلی را بیان می‌کند.

\واژه{فریم.ورک} پیشنهادی ما، حول ایده‌ی جداسازی \واژه{سناریو}‌ها از مصادیق صحت‌سنجی آن صورت گرفته است. در این معماری، \واژه{سناریو}‌هایی توسط برنامه‌نویس تعبیه می‌شود که مطابق تعریف، انتظار می‌رود همواره برقرار باشند. مجموعه‌ی این \واژه{سناریو}‌ها را به عنوان یک لایه از معماری \واژه{فریم.ورک} \واژه{تست} خود در نظر می‌گیریم و آن را «لایه‌ی \واژه{سناریو}» می‌نامیم. در این لایه، نویسنده‌ی \واژه{تست} تنها به توصیف نیازمندی‌های صحت‌سنجی برنامه می‌پردازد و به تولید مصادیق \واژه{سناریو}‌ها نمی‌پردازد.

با توجه به تعریف، خود \واژه{سناریو}‌ها تنها مجموعه‌ای از گزاره‌هایی هستند که در پیاده‌سازی درست برقرار هستند. حال آن که سنجش صحت این گزاره‌ها می‌تواند \واژه{متریک} خوبی برای کیفیت کد باشد. در \واژه{فریم.ورک} پیشنهادی، لایه‌ی دیگری از معماری \واژه{تست} به این موضوع تخصیص دارد که آن را «لایه‌ی \واژه{اکتور}» می‌نامیم. در واقع، لایه‌ی \واژه{اکتور} موظف است تا با تغییر مداوم وضعیت سامانه، \واژه{پری.کاندیشن}‌های \واژه{سناریو}‌های مختلف را فراهم کند تا صحت آن‌ها سنجیده شود.

دو لایه‌ی \واژه{سناریو} و \واژه{اکتور}، ارکان اصلی معماری پیشنهادی جهت پیاده‌سازی \واژه{فریم.ورک} \واژه{تست} هستند.
