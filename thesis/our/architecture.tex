\قسمت{معماری \واژه{فریم.ورک}}
\برچسب{sec:arch}
در بخش \رجوع{gwt} با نحوه‌ی بیان یک \واژه{سناریو} آشنا شدیم. یک سناریو، در واقع پاسخ صحیح یک زیرمسأله از مسأله‌ی اصلی را بیان می‌کند.

\واژه{فریم.ورک} پیشنهادی ما، حول ایده‌ی \واژه{decoupling} \واژه{سناریو}‌ها از مصادیق صحت‌سنجی آن شکل گرفته است. در این معماری، \واژه{سناریو}‌هایی توسط برنامه‌نویس تعبیه می‌شود که مطابق تعریف، انتظار می‌رود همواره برقرار باشند. مجموعه‌ی این \واژه{سناریو}‌ها را به عنوان یک لایه از معماری \واژه{فریم.ورک} \واژه{تست} خود در نظر می‌گیریم و آن را «لایه‌ی \واژه{سناریو}» می‌نامیم. در این لایه، نویسنده‌ی \واژه{تست} تنها به توصیف نیازمندی‌های صحت‌سنجی برنامه می‌پردازد و دغدغه‌ی تولید مصادیق \واژه{سناریو}‌ها را نخواهد داشت.

با توجه به تعریف، خود \واژه{سناریو}‌ها تنها مجموعه‌ای از گزاره‌ها هستند که در یک پیاده‌سازی درست، برقرار خواهند بود. حال آن که سنجش صحت این گزاره‌ها می‌تواند \واژه{متریک} خوبی برای کیفیت کد باشد. در \واژه{فریم.ورک} پیشنهادی، لایه‌ی دیگری از معماری \واژه{تست} به این موضوع تخصیص دارد که آن را «لایه‌ی \واژه{اکتور}» می‌نامیم. در واقع، لایه‌ی \واژه{اکتور} موظف است تا با تغییر مداوم وضعیت سامانه، \واژه{پری.کاندیشن}‌های \واژه{سناریو}‌های مختلف را فراهم کند تا صحت آن‌ها سنجیده شود.

\شروع{شکل}[tbp]
\تنظیم‌ازوسط
\درج‌تصویر[پهنا=0.9\پهنای‌سطر]{architecture}
\تر\موقتم{
  مقایسه معماری پیشنهادی با معماری متداول آزمون
}
\شرح[\موقتم]{\موقتم~}
\برچسب{fig:architecture}
\پایان{شکل}

همانطور که در شکل \رجوع{fig:architecture} نشان داده شده، دو لایه‌ی
\واژه{سناریو} و \واژه{اکتور}، ارکان اصلی معماری پیشنهادی (معماری سمت
راست در شکل) جهت پیاده‌سازی \واژه{فریم.ورک} \واژه{تست} هستند این دو
لایه، جایگیزین لایه‌ی «\واژه{تست}» در معماری عمومی (معماری سمت چپ در
شکل) می‌شوند.

در معماری عمومی، لایه‌ی \واژه{تست} عملیات زیر را انجام می‌دهد:

\شروع{بارهها}

\باره{\متن‌سیاه{\نام{آماده‌سازی}{Setup}:} همانطور که در بخش
  \رجوع{sec:testcase} اشاره شد، برای بررسی عملکرد \واژه{sut} در هر
  \واژه{تست.کیس} لازم است ورودی‌ها و پیش‌شرایط آن مهیا شده باشند. مثلاً
  برای بررسی عملکرد قسمت «ثبت‌نام» از یک سامانه‌ی آموزشی، لازم است یک
  دانشجو و یک ارائه از درس در سامانه موجود باشند تا بتوان صحت عملکرد
  ثبت‌نام را بر روی آن سنجید. }

\باره{ \متن‌سیاه{\نام{تحریک کنش در سامانه}{Action Trigger}:} پس از
  آماده‌سازی حالت سامانه، دستوری که باعث اعمال رفتار مورد نظر توسط
  \واژه{sut} می‌شود را فراخوانی می‌نماییم. برای مثال، برای بررسی عملکرد
  ثبت‌نام در سامانه‌ی آموزش، تابعی که مربوط به عمل ثبت‌نام است را
  فراخوانی می‌نماییم. }

\باره{ \متن‌سیاه{\نام{اثبات صحت عملکرد}{Assertion}:} پس از تحریک کنش در
  سامانه، باید حالت نهایی سامانه را بررسی نماییم و برقراری
  \واژه{پست.کاندیشن}‌هایی که توسط مشتری برای این عمل تعریف شده را در
  سامانه بررسی نماییم. در سامانه‌ی آموزش، باید مواردی مانند کم شدن
  ظرفیت ارائه‌ی درس پس از ثبت‌نام و قرار گرفتن دانشجو در لیست دانشجویان
  ارائه‌ی درس را بررسی نماییم. }

\باره{ \متن‌سیاه{\نام{خنثی‌سازی تغییرات}{Tear Down}:} پس از اطمینان از
  بررسی صحت عملکرد \واژه{sut}، تغییراتی که در قسمت‌های آماده‌سازی و
  تحریک، در حالت سامانه ایجاد شده را خنثی‌سازی می‌نماییم تا در عملیات
  سایر \واژه{تست}‌ها تداخل ایجاد نشود. برای مثال، در سامانهه}

\پایان{بارهها}

