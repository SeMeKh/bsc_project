\قسمت{تألیف \واژه{تست} برای سامانه‌ی آموزش}

برای بررسی صحت عملکرد \واژه{insanity}، یک سامانه‌ی بسیار ساده را طراحی
و در چارچوب \واژه{جنگو} پیاده‌سازی نمودیم تا بتوانیم در آن سامانه،
\واژه{تست}‌هایی را در چارچوب \واژه{insanity} تالیف نماییم.

جهت ملموس‌تر بودن \واژه{انتیتی}‌ها برای خواننده، سامانه‌ی انتخاب واحد را
برای \واژه{تست} انتخاب نمودیم و سعی نمودیم جزییاتی که مرتبط با
\واژه{تست} نیستند را از آن حذف نماییم. در این سامانه تنها عملیات
\واژه{کراد} برای موجودیت‌ها و انتخاب واحد توسط دانشجو پیاده‌سازی شده
است.

\زیرقسمت{موجودیت‌های سامانه‌}

\شروع{شکل}[tbp]
\تنظیم‌ازوسط
\درج‌تصویر[پهنا=0.9\پهنای‌سطر]{edu-entities}
\تر\موقتم{
  نمودار موجودیت‌های سامانه‌ی ثبت نام آموزش
}
\شرح[\موقتم]{\موقتم}
\برچسب{fig:erd}
\پایان{شکل}

همانطور که در شکل \رجوع{fig:erd} نشان داده شده، در حالت ساده این
سامانه دارای موجودیت‌های زیر می‌باشد:

\شروع{شمارش}

\فقره{ \نام{کاربر}{User}: این موجودیت اطلاعات کاربری یک نفر را در
  سامانه نگه می‌دارد. نام، نام خانوادگی و مشخصات اهراز هویت از جمله
  صفات این موجودیت می‌باشند. این موجودیت توسط چارچوب \واژه{جنگو} ارائه
  می‌شود. }

\فقره{ \نام{دانشجو}{Student} و \نام{استاد}{Professor}: در این سامانه
به اطلاعاتی جز مشخصات فردی و دسترسی‌ها برای دانشجو و استاد نیاز
نداریم. این موجودیت‌ها ارتباط یک‌به‌یک با موجودیت کاربر دارند که مشخصات
فردی و دسترسی‌ها در موجودیت کاربر ذخیره می‌شوند. }

\فقره{ \نام{نیمسال تحصیلی}{Semester} و \نام{درس}{Course}: برای این دو
  موجودیت، نگه‌داری صفت «نام» برای آن‌ها در سامانه کافیست. }

\فقره{ \نام{ارائه}{Offering}: ارتباط چند به چند بین موجودیت‌های درس و
  نیمسال تحصیلی و استاد ارائه‌دهنده‌ی آن، در این موجودی نگه‌داری
  می‌شود.

  صفت \کد{capacity}، ظرفیت ارائه را نگه می‌دارد. صفت \کد{available\_capacity}،
  یک صفت محاسبه‌پذیر است که برای کارایی بالاتر، در پایگاه داده ذخیره
  می‌شود. این صفت حاصل تفریق ظرفیت درس از تعداد ثبت‌نام‌های آن است. صفت
  \کد{is\_enrollable} فعال یا غیرفعال بودن قابلیت ثبت نام دانشجویان در آن
  ارائه را مشخص می‌نماید. }

\فقره{ \نام{ثبت نام}{Enrollment}: این موجودیت ارتباط چند به چند ثبت‌نام
  میان موجودیت‌های دانشجو و ارائه را نگهداری می‌کند. }

\پایان{شمارش}

\زیرقسمت{\واژه{یو.آی.} سامانه}

برای شروع به کار سامانه، با اجرای دستور \کد{./manage.py~runserver\_plus}
در پوشه‌ی اصلی برنامه، \واژه{سرور} شروع به کار نموده و می‌توان از طریق
نشانی \کد{http://localhost:8000/} به \واژه{یو.آی.}  سامانه دسترسی پیدا
نمود.

ابتدا نام کاربری و کلمه عبور را وارد می‌نماییم (شکل
\رجوع{fig:edu-login}). سپس بسته به نقشی که کاربر در سامانه داشته باشد،
وارد یکی از صفحه‌های کارمند، دانشجو یا استاد خواهیم شد.

\شروع{شکل}[tbp]
\تنظیم‌ازوسط
\درج‌تصویر[پهنا=1\پهنای‌سطر]{edu-login}
\تر\موقتم{
  صفحه‌ی ورود سامانه‌ی ثبت نام
}
\شرح[\موقتم]{\موقتم}
\برچسب{fig:edu-login}
\پایان{شکل}

در صفحه‌ی کارمند (شکل \رجوع{fig:edu-staff})، می‌توان از منوی سمت چپ، هر
کدام از گزینه‌های کاربر، دانشجو، استاد، درس، نیم‌سال تحصیلی و ارائه را
انتخاب نمود؛ که در تصویر \رجوع{fig:edu-staff}، گزینه‌ی ارائه انتخاب شده
است.

\شروع{شکل}[tbp]
\تنظیم‌ازوسط
\درج‌تصویر[پهنا=1\پهنای‌سطر]{edu-staff}
\تر\موقتم{
  صفحه‌ی کاربری کارمند در سامانه‌ی ثبت نام
}
\شرح[\موقتم]{\موقتم}
\برچسب{fig:edu-staff}
\پایان{شکل}

با انتخاب هر گزینه، لیست موارد ثبت‌شده در سامانه برای آن گزینه قابل
نمایش است. دو ستون نام{ویرایش}{Edit} و نام{حذف}{Delete} در انتهای هر
ردیف از لیست هر گزینه قرار دارد که می‌توان آن ردیف را ویرایش یا حذف
نمود. برای مثال، در شکل \رجوع{fig:edu-edit}، صفحه‌ی مربوط به ویرایش
مشخصات یک ارائه نمایش داده شده است.

\شروع{شکل}[tbp]
\تنظیم‌ازوسط
\درج‌تصویر[پهنا=1\پهنای‌سطر]{edu-edit}
\تر\موقتم{
  صفحه‌ی ویرایش مشخصات ارائه توسط کارمند
}
\شرح[\موقتم]{\موقتم}
\برچسب{fig:edu-edit}
\پایان{شکل}

علاوه بر عملیات ویرایش و حذف، ظرفیت ارائه توسط کارمند قابل تغییر
است. برای این کار، با کلیک بر روی کلید تغییر در ستون
\نام{ظرفیت}{Capacity} از جدول، می‌تواند ظرفیت آن درس را تغییر دهد (شکل
\رجوع{fig:edu-capacity})

\شروع{شکل}[tbp]
\تنظیم‌ازوسط
\درج‌تصویر[پهنا=1\پهنای‌سطر]{edu-capacity}
\تر\موقتم{
  تغییر ظرفیت درس توسط کارمند
}
\شرح[\موقتم]{\موقتم}
\برچسب{fig:edu-capacity}
\پایان{شکل}

همچنین اگر کاربر دسترسی \نام{مدیر سیستم}{Superuser} نیز داشته باشد،
می‌تواند با کلیک استفاده از منوی بالا-راست به قسمت \نام{مدیریت سیستم
  جنگو}{Django Administration} نیز دسترسی پیدا کند و به تمام
موجودیت‌های سامانه دسترسی پیدا کند. برای مثال، می‌تواند از طریق تنظیمات
کاربران (شکل \رجوع{fig:edu-hijack}) بدون نیاز به اطلاع از کلمه‌ی عبور
سایر کاربرها، از طرف آن‌ها وارد سامانه بشود و صفحه‌های قابل مشاهده توسط
آن‌ها را ببیند. نوار زرد‌رنگی که در بالای برخی صفحات در شکل‌های بعد
مشاهده می‌شود، به این خاطر است.

\شروع{شکل}[tbp]
\تنظیم‌ازوسط
\درج‌تصویر[پهنا=1\پهنای‌سطر]{edu-hijack}
\تر\موقتم{
  دسترسی مدیر سامانه به صفحه‌های کاربرها
}
\شرح[\موقتم]{\موقتم}
\برچسب{fig:edu-hijack}
\پایان{شکل}

در صورتی که کاربر وارد شده به سامانه نقش استاد داشته باشد، می‌تواند
لیست درس‌هایی که توسط وی ارائه شده است را مشاهده کند. امکان حذف یا
اضافه‌ی دروس ارائه شده‌ی هر استاد، برای نقش کارمند فعال می‌باشد.

\شروع{شکل}[tbp]
\تنظیم‌ازوسط
\درج‌تصویر[پهنا=1\پهنای‌سطر]{edu-professor}
\تر\موقتم{
  صفحه‌ی کاربری استاد در سامانه‌ی ثبت نام
}
\شرح[\موقتم]{\موقتم}
\برچسب{fig:edu-professor}
\پایان{شکل}

همچنین اگر دانشجو وارد سامانه شود، می‌تواند لیست درس‌هایی که در آن‌ها
ثبت‌نام کرده را مشاهده نماید. (شکل \رجوع{fig:edu-student}).
\شروع{شکل}[tbp]
\تنظیم‌ازوسط
\درج‌تصویر[پهنا=1\پهنای‌سطر]{edu-student}
\تر\موقتم{
  صفحه‌ی کاربری دانشجو در سامانه‌ی ثبت نام
}
\شرح[\موقتم]{\موقتم}
\برچسب{fig:edu-student}
\پایان{شکل}

برای ثبت‌نام در یک ارائه از یک درس، دانشجو می‌تواند بر روی کلید در سمت
بالا-راست صفحه‌ی شکل \رجوع{fig:edu-student} کلیک نماید. در این صورت به
صفحه‌ی ثبت‌نام که در شکل \رجوع{fig:edu-enroll} نمایش داده شده، منتقل
می‌شود. در این صفحه، لیستی از ارائه‌هایی که صفت \کد{is\_enrollable} آن‌ها
فعال باشد، در اختیار دانشجو برای انتخاب قرار می‌گیرد.

\شروع{شکل}[tbp]
\تنظیم‌ازوسط
\درج‌تصویر[پهنا=1\پهنای‌سطر]{edu-enroll}
\تر\موقتم{
  صفحه‌ی ثبت‌نام دانشجو در درس‌های ارائه‌شده توسط سامانه
}
\شرح[\موقتم]{\موقتم}
\برچسب{fig:edu-enroll}
\پایان{شکل}

پس از تایید فرم ثبت‌نام، در صورتی که ظرفیت درس تکمیل شده باشد، با خطای
نشان‌داده شده در شکل \رجوع{fig:edu-full} مواجه می‌شود و دانشجو می‌تواند
درسی دیگر را برای ثبت‌نام انتخاب نماید. در غیر این‌صورت، ثبت‌نام انجام
شده و ردیف مربوط به درس جدید در لیست درس‌های دانشجو مشاهده می‌شود.

\شروع{شکل}[tbp]
\تنظیم‌ازوسط
\درج‌تصویر[پهنا=1\پهنای‌سطر]{edu-full}
\تر\موقتم{
  خطای پر بودن ظرفیت هنگام ثبت‌نام دانشجو در سامانه
}
\شرح[\موقتم]{\موقتم}
\برچسب{fig:edu-full}
\پایان{شکل}

                                                                    

%%% Local Variables:
%%% mode: latex
%%% TeX-master: "../main"
%%% End:


