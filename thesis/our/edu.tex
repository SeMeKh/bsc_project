\قسمت{تألیف \واژه{تست} برای یک سامانه‌ی آموزشی}

برای بررسی صحت عملکرد \واژه{insanity}، یک سامانه‌ی بسیار ساده طراحی و
پیاده‌سازی نمودیم تا بتوانیم در آن سامانه، \واژه{تست}‌هایی را در چارچوب
\واژه{insanity} تالیف نماییم.

جهت ملموس‌تر بودن \واژه{انتیتی}‌ها برای خواننده، سامانه‌ی انتخاب واحد را
برای \واژه{تست} انتخاب نمودیم. سعی نمودیم پیچیدگی‌هایی که مرتبط با
\واژه{تست}  همانطور که در شکل \رجوع{fig:erd} نشان
داده شده، این سامانه دارای موجودیت‌های زیر می‌باشد:

\شروع{شمارش}

\فقره{ \نام{کاربر}{User}: این موجودیت اطلاعات کاربری یک نفر را در
  سامانه نگه می‌دارد. نام، نام خانوادگی و مشخصات اهراز هویت از جمله
  صفات این موجودیت می‌باشند. این موجودیت توسط چارچوب \واژه{جنگو} ارائه
  می‌شود. }

\فقره{ \نام{دانشجو}{Student} و \نام{استاد}{Professor} }: در حالت ساده‌ی
سامانه، به اطلاعاتی بجز مشخصات فردی و دسترسی‌ها برای دانشجو و استاد
نیاز نداریم که این اطلاعات در موجودیت کاربر قرار دارند. }

\فقره{ \نام{نیمسال تحصیلی}{Semester} و \نام{درس}{Course} }: 

%%% Local Variables:
%%% mode: latex
%%% TeX-master: "../main"
%%% End:


