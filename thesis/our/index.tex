\فصل{معماری}
\برچسب{chap:arch}

آزمودن برنامه راهی برای افزایش کیفیت نرم‌افزار است. پیش از این رویکردهایی در توسعه‌ی \واژه{تست}های نرم‌افزاری را دیدیم. با ایده گرفتن از نقاط قوّت آن‌ها و در تلاش برای یافتن راهکاری برای ضعف‌های آن‌ها، معماری جدیدی را جهت انجام \واژه{تست} پیشنهاد می‌کنیم، فواید استفاده از آن را بررسی، و با رویکردها و روش‌های مرتبط مقایسه می‌کنیم.


\قسمت{معماری \واژه{فریم.ورک}}

در بخش \رجوع{gwt} با نحوه‌ی بیان یک \واژه{سناریو} آشنا شدیم. یک سناریو، در واقع پاسخ صحیح یک زیرمسأله از مسأله‌ی اصلی را بیان می‌کند.

\واژه{فریم.ورک} پیشنهادی ما، حول ایده‌ی جداسازی \واژه{سناریو}‌ها از مصادیق صحت‌سنجی آن شکل گرفته است. در این معماری، \واژه{سناریو}‌هایی توسط برنامه‌نویس تعبیه می‌شود که مطابق تعریف، انتظار می‌رود همواره برقرار باشند. مجموعه‌ی این \واژه{سناریو}‌ها را به عنوان یک لایه از معماری \واژه{فریم.ورک} \واژه{تست} خود در نظر می‌گیریم و آن را «لایه‌ی \واژه{سناریو}» می‌نامیم. در این لایه، نویسنده‌ی \واژه{تست} تنها به توصیف نیازمندی‌های صحت‌سنجی برنامه می‌پردازد و دغدغه‌ی تولید مصادیق \واژه{سناریو}‌ها را نخواهد داشت.

با توجه به تعریف، خود \واژه{سناریو}‌ها تنها مجموعه‌ای از گزاره‌ها هستند که در یک پیاده‌سازی درست، برقرار خواهند بود. حال آن که سنجش صحت این گزاره‌ها می‌تواند \واژه{متریک} خوبی برای کیفیت کد باشد. در \واژه{فریم.ورک} پیشنهادی، لایه‌ی دیگری از معماری \واژه{تست} به این موضوع تخصیص دارد که آن را «لایه‌ی \واژه{اکتور}» می‌نامیم. در واقع، لایه‌ی \واژه{اکتور} موظف است تا با تغییر مداوم وضعیت سامانه، \واژه{پری.کاندیشن}‌های \واژه{سناریو}‌های مختلف را فراهم کند تا صحت آن‌ها سنجیده شود.

دو لایه‌ی \واژه{سناریو} و \واژه{اکتور}، ارکان اصلی معماری پیشنهادی جهت پیاده‌سازی \واژه{فریم.ورک} \واژه{تست} هستند.

\قسمت{فواید}
\برچسب{sec:adv}

\زیرقسمت{کاهش حجم \واژه{تست}‌ها}
\برچسب{sec:loc}
یکی از دلایل عمده‌ی عدم علاقه‌ی برنامه‌نویس‌ها به تألیف \واژه{تست}، حجم زیاد \واژه{تست} در مقایسه با میزان کد پوشیده شده توسط آن است. هر چه سطح \واژه{تست} به سطح سیستمی نزدیک‌تر باشد، حجم این کد نیز به دلیل آماده کردن شرایط اولیه‌ی \واژه{تست} افزایش می‌یابد.

با توجه به این که در معماری پیشنهادی، تعریف \واژه{سناریو}‌ها مستقل از داده‌ها انجام می‌شود، امکان \واژه{reuse} از داده‌های یکسان برای \واژه{تست}‌های متفاوت وجود دارد که این موضوع منجر به کاهش حجم \واژه{تست}‌ها می‌شود.

\زیرقسمت{کاهش هزینه‌ی نگه‌داری}
از آن جا که تغییر سریع در بسیاری از \واژه{سیستم}‌های نرم‌افزاری ضروری است، حجم زیاد \واژه{تست}‌ها (ر.ک. \رجوع{sec:loc}) در هنگام تغییر نیز نیاز به هم‌گام‌سازی دارند. در معماری تک‌لایه، با توجه به اینکه شرایط اجرای \واژه{تست} توسط خود آن فراهم می‌شود، \واژه{coupling} به \واژه{component}‌های خارج از محدوده‌ی \واژه{تست} وجود دارد و هنگام تغییر رفتار هر \واژه{component}، این تغییرات در میان تمام \واژه{تست}‌های درگیر انتشار می‌یابد.

با توجه به جدا شدن لایه‌ی \واژه{سناریو} و \واژه{اکتور} در معماری پیشنهادی، تغییرات محدود به بخشی از \واژه{تست}‌ها خواهند بود که واقعاً تغییر کرده و به بخش‌های دیگر انتشار نمی‌یابند.


\زیرقسمت{افزایش احتمال یافتن خطا}
یکی از تفاوت‌های اساسی معماری پیشنهادی و معماری‌های تک‌لایه در کلی بودن تعریف \واژه{سناریو}‌هاست؛ به این معنا که یک \واژه{سناریو} می‌تواند روی داده‌های مختلفی قابل اعمال باشد. بنابراین، هر \واژه{سناریو} عملاً معادل یک مجموعه‌ی \واژه{تست} عمل می‌کند.

این عمومی بودن تعریف \واژه{سناریو} کمک می‌کند تا صحت آن نه فقط روی یک داده، بلکه برای چندین حالت متفاوت بررسی شود. این موضوع احتمال یافتن خطاهای موجود را افزایش داده و در نتیجه موجب افزایش تأثیرگذاری فرآیند \واژه{تست} می‌گردد.

\قسمت{در سطوح مختلف \واژه{تست}}
اگر چه معماری پیشنهادی ما مستقل از سطح \واژه{تست} است، با این حال تأثیر مثبت آن در تألیف \واژه{سیستم.تست} مشهودتر است. در این لایه به دلیل سطح بالا بودن تست‌ها، حجم

\قسمت{استفاده خارج از محیط \واژه{تست}}
انجام \واژه{تست} تنها یکی از روش‌های افزایش کیفیت نرم‌افزار است. ادعا می‌کنیم که معماری پیشنهادی خارج از محیط \واژه{تست} نیز می‌تواند به افزایش کیفیت نرم‌افزار کمک کند.

در هنگام \واژه{تست}، هدف یافتن \واژه{defect} نرم‌افزار، پیش از عملیاتی شدن آن است. اما بسیاری از \واژه{defect}‌های نرم‌افزاری حتی پس از عملیاتی شدن نیز از دیده نهان می‌مانند.

در \واژه{فریم.ورک} ارائه‌شده، می‌توان به جای لایه‌ی اکتور، کاربران حقیقی \واژه{سیستم} را قرار داد تا از نرم‌افزار استفاده کنند. به این ترتیب، \واژه{سیستم} می‌تواند عملکرد خودش را (حتی در حالی که عملیاتی است) ارزیابی کند و برخی از این \واژه{defect}‌ها را یافته، و جهت رسیدگی توسعه‌دهندگان گزارش کند. از جمله \واژه{defect}‌هایی که به خوبی به این روش پیدا می‌شوند، بروز \واژه{inconsistency} در مقادیر محاسبه‌شدنی است.

\قسمت{مقایسه با مفاهیم دیگر}
\زیرقسمت{در مقایسه با contract}

\زیرقسمت{در مقایسه با mutational testing}




\فصل{پیاده‌سازی}

در این بخش، ابتدا گزارشی از پیاده‌سازی معماری مطرح‌شده در فصل \رجوع{chap:arch} به \واژه{پایتون} ارائه می‌کنیم. پس از آن نحوه
نهایتاً با استفاده از آن برای یک سامانه‌ی آموزش \واژه{تست}هایی تألیف کردیم.

\قسمت{پیاده‌سازی بر مبنای \واژه{فریم.ورک} \واژه{جنگو}}
\برچسب{sec:insanity}
به عنوان بخشی از پروژه کارشناسی، یک \واژه{فریم.ورک} برای \واژه{تست}
\واژه{بی.دی.دی.} با \واژه{جنگو} و مبتنی بر معماری ارائه‌شده در فصل \رجوع{sec:arch} را پیاده‌سازی نمودیم.


\زیرقسمت{نحوه‌ی استفاده}

این \واژه{فریم.ورک} در قالب یک افزونه برای \واژه{جنگو} نوشته شده. برای
استفاده از این افزونه مراحل زیر را انجام دهید:

\شروع{شکل}[tbp]
\تنظیم‌ازوسط
\درج‌تصویر[پهنا=0.7\پهنای‌سطر]{installed-apps}
\تر\موقتم{
  افزودن چارچوب به جنگو
}
\شرح[\موقتم]{\موقتم}
\برچسب{fig:installed-apps}
\پایان{شکل}

\شروع{شکل}[tbp]
\تنظیم‌ازوسط
\درج‌تصویر[پهنا=0.7\پهنای‌سطر]{sentry-url}
\تر\موقتم{
  ورود تنظیمات سنتری
}
\شرح[\موقتم]{\موقتم}
\برچسب{fig:sentry-url}
\پایان{شکل}


\شروع{enumerate} \فقره {با اجرای دستور \\ {\کد{pip install
      https://github.com/SeMeKh/bsc\_project/archive/master.zip}، \\
    چارچوب \واژه{insanity} را نصب نمایید. }}

\فقره { نام افزونه (\کد{insanity}) را مطابق شکل
  \رجوع{fig:installed-apps} به لیست افزونه‌ها در تنظیمات \واژه{جنگو}
  اضافه نمایید. }

\فقره{ تنظیمات \کد{sentry} را جهت ثبت جزییات خطاها مطابق شکل
  \رجوع{fig:sentry-url} در فایل \کد{settings.py} وارد نمایید.}
  
\فقره { \واژه{سناریو}‌های مطلوب در فایل \کد{scenarios.py} مرتبط با هر
  \واژه{component} \واژه{جنگو} قرار بگیرند. هر سناریو، از با یک
  \واژه{کلاس} نمایش داده می‌شود که  }

\فقره { هر کدام از توابعی که به عنوان \واژه{trigger}
  \واژه{سناریو}‌ها عمل می‌کنند، توسط \واژه{decorator} \کد{action} پوشیده
  شوند. }

\پایان{enumerate}

\زیرقسمت{اجزاء پیاده‌سازی}

\زیرزیرقسمت{\واژه{decorator} \کد{action}}
این \واژه{decorator} به توابع مختلف اعمال می‌شود و آن‌ها را به عنوان یک \واژه{trigger} \واژه{سناریو} ثبت می‌کند. به این ترتیب، در صورت اجرای این توابع، صحت \واژه{سناریو}‌های مرتبط با آن‌ها بررسی می‌شود.

\زیرزیرقسمت{\واژه{کلاس} \کد{Scenario}}
هر \واژه{سناریو} به صورت یک \واژه{کلاس} پیاده‌سازی می‌شود که از \کد{Scenario} ارث می‌برد. این کلاس دارای توابع \کد{given}، \کد{when} و \کد{then} است که معادل مفاهیم متناظر در \واژه{بی.دی.دی.} هستند.

\زیرقسمت{کد منبع}
کد منبع \واژه{فریم.ورک} پیاده‌سازی‌شده از طریق آدرس زیر در دسترس است:

\begin{center}
\url{https://github.com/SeMeKh/bsc_project}
\end{center}

\قسمت{تألیف \واژه{تست} برای سامانه‌ی آموزش}

برای بررسی صحت عملکرد \واژه{insanity}، یک سامانه‌ی بسیار ساده را طراحی
و در چارچوب \واژه{جنگو} پیاده‌سازی نمودیم تا بتوانیم در آن سامانه،
\واژه{تست}‌هایی را در چارچوب \واژه{insanity} تالیف نماییم.

جهت ملموس‌تر بودن \واژه{انتیتی}‌ها برای خواننده، سامانه‌ی انتخاب واحد را
برای \واژه{تست} انتخاب نمودیم و سعی نمودیم جزییاتی که مرتبط با
\واژه{تست} نیستند را از آن حذف نماییم. در این سامانه تنها عملیات
\واژه{کراد} برای موجودیت‌ها و انتخاب واحد توسط دانشجو پیاده‌سازی شده
است.

\زیرقسمت{موجودیت‌های سامانه‌}

\شروع{شکل}[tbp]
\تنظیم‌ازوسط
\درج‌تصویر[پهنا=0.9\پهنای‌سطر]{edu-entities}
\تر\موقتم{
  نمودار موجودیت‌های سامانه‌ی ثبت نام آموزش
}
\شرح[\موقتم]{\موقتم}
\برچسب{fig:erd}
\پایان{شکل}

همانطور که در شکل \رجوع{fig:erd} نشان داده شده، در حالت ساده این
سامانه دارای موجودیت‌های زیر می‌باشد:

\شروع{شمارش}

\فقره{ \نام{کاربر}{User}: این موجودیت اطلاعات کاربری یک نفر را در
  سامانه نگه می‌دارد. نام، نام خانوادگی و مشخصات اهراز هویت از جمله
  صفات این موجودیت می‌باشند. این موجودیت توسط چارچوب \واژه{جنگو} ارائه
  می‌شود. }

\فقره{ \نام{دانشجو}{Student} و \نام{استاد}{Professor}: در این سامانه
به اطلاعاتی جز مشخصات فردی و دسترسی‌ها برای دانشجو و استاد نیاز
نداریم. این موجودیت‌ها ارتباط یک‌به‌یک با موجودیت کاربر دارند که مشخصات
فردی و دسترسی‌ها در موجودیت کاربر ذخیره می‌شوند. }

\فقره{ \نام{نیمسال تحصیلی}{Semester} و \نام{درس}{Course}: برای این دو
  موجودیت، نگه‌داری صفت «نام» برای آن‌ها در سامانه کافیست. }

\فقره{ \نام{ارائه}{Offering}: ارتباط چند به چند بین موجودیت‌های درس و
  نیمسال تحصیلی و استاد ارائه‌دهنده‌ی آن، در این موجودی نگه‌داری
  می‌شود.

  صفت $capacity$، ظرفیت ارائه را نگه می‌دارد. صفت $available\_capacity$،
  یک صفت محاسبه‌پذیر است که برای کارایی بالاتر، در پایگاه داده ذخیره
  می‌شود. این صفت حاصل تفریق ظرفیت درس از تعداد ثبت‌نام‌های آن است. صفت
  $is\_enrollable$ فعال یا غیرفعال بودن قابلیت ثبت نام دانشجویان در آن
  ارائه را مشخص می‌نماید. }

\فقره{ \نام{ثبت نام}{Enrollment}: این موجودیت ارتباط چند به چند ثبت‌نام
  میان موجودیت‌های دانشجو و ارائه را نگهداری می‌کند. }

\پایان{شمارش}

\زیرقسمت{\واژه{یو.آی.} سامانه}

برای شروع به کار سامانه، با اجرای دستور $./manage.py runserver\_plus$
در پوشه‌ی اصلی برنامه، \واژه{سرور} شروع به کار نموده و می‌توان از طریق
نشانی $http://localhost:8000/$ به \واژه{یو.آی.}  سامانه دسترسی پیدا
نمود.

ابتدا نام کاربری و کلمه عبور را وارد می‌نماییم (شکل
\رجوع{fig:edu-login}). سپس بسته به نقشی کاربر در سامانه داشته باشد،
یکی از صفحه‌های کارمند، دانشجو یا استاد خواهیم شد.

\شروع{شکل}[tbp]
\تنظیم‌ازوسط
\درج‌تصویر[پهنا=1\پهنای‌سطر]{edu-login}
\تر\موقتم{
  صفحه‌ی ورود سامانه‌ی ثبت نام
}
\شرح[\موقتم]{\موقتم}
\برچسب{fig:edu-login}
\پایان{شکل}

در صفحه‌ی کارمند (شکل \رجوع{fig:edu-staff})، می‌توان از منوی سمت چپ، هر
کدام از گزینه‌های کاربر، دانشجو، استاد، درس، نیم‌سال تحصیلی و ارائه را
انتخاب نمود. در تصویر \رجوع{fig:edu-staff}، ارائه انتخاب شده است.

\شروع{شکل}[tbp]
\تنظیم‌ازوسط
\درج‌تصویر[پهنا=1\پهنای‌سطر]{edu-staff}
\تر\موقتم{
  صفحه‌ی کاربری کارمند در سامانه‌ی ثبت نام
}
\شرح[\موقتم]{\موقتم}
\برچسب{fig:edu-staff}
\پایان{شکل}

با انتخاب هر گزینه، لیست موارد ثبت‌شده در سامانه برای آن گزینه قابل
نمایش است. دو ستون نام{ویرایش}{Edit} و نام{حذف}{Delete} در انتهای هر
ردیف از لیست هر گزینه قرار دارد که می‌توان آن ردیف را ویرایش یا حذف
نمود. برای مثال، در شکل \رجوع{fig:edu-edit}، صفحه‌ی مربوط به ویرایش
مشخصات یک ارائه نمایش داده شده است.

\شروع{شکل}[tbp]
\تنظیم‌ازوسط
\درج‌تصویر[پهنا=1\پهنای‌سطر]{edu-edit}
\تر\موقتم{
  صفحه‌ی ویرایش مشخصات ارائه توسط کارمند
}
\شرح[\موقتم]{\موقتم}
\برچسب{fig:edu-edit}
\پایان{شکل}

در صورتی که کاربر وارد شده به سامانه استاد باشد، می‌تواند لیست درس‌هایی
که توسط وی ارائه شده است را مشاهده کند. امکان حذف یا اضافه‌ی دروس ارائه
شده‌ی هر استاد، برای نقش کارمند فعال می‌باشد.

\شروع{شکل}[tbp]
\تنظیم‌ازوسط
\درج‌تصویر[پهنا=1\پهنای‌سطر]{edu-professor}
\تر\موقتم{
  صفحه‌ی کاربری استاد در سامانه‌ی ثبت نام
}
\شرح[\موقتم]{\موقتم}
\برچسب{fig:edu-professor}
\پایان{شکل}

\شروع{شکل}[tbp]
\تنظیم‌ازوسط
\درج‌تصویر[پهنا=1\پهنای‌سطر]{edu-student}
\تر\موقتم{
  صفحه‌ی کاربری دانشجو در سامانه‌ی ثبت نام
}
\شرح[\موقتم]{\موقتم}
\برچسب{fig:edu-student}
\پایان{شکل}

\شروع{شکل}[tbp]
\تنظیم‌ازوسط
\درج‌تصویر[پهنا=1\پهنای‌سطر]{edu-staff}
\تر\موقتم{
  صفحه‌ی کاربری کارمند در سامانه‌ی ثبت نام
}
\شرح[\موقتم]{\موقتم}
\برچسب{fig:edu-staff}
\پایان{شکل}


                                                                    

%%% Local Variables:
%%% mode: latex
%%% TeX-master: "../main"
%%% End:




