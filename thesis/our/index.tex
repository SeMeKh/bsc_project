\فصل{معماری}
\برچسب{chap:arch}

آزمودن برنامه راهی برای افزایش کیفیت نرم‌افزار است. پیش از این رویکردهایی در توسعه‌ی \واژه{تست}های نرم‌افزاری را دیدیم. با ایده گرفتن از نقاط قوّت آن‌ها و در تلاش برای یافتن راهکاری برای ضعف‌های آن‌ها، معماری جدیدی را جهت انجام \واژه{تست} پیشنهاد می‌کنیم، فواید استفاده از آن را بررسی، و با رویکردها و روش‌های مرتبط مقایسه می‌کنیم.


\قسمت{معماری}
آزمودن برنامه راهی برای افزایش کیفیت نرم‌افزار است. پیش از این رویکردهایی در توسعه‌ی \واژه{تست}های نرم‌افزاری را دیدیم. با ایده گرفتن از نقاط قوّت آن‌ها و در تلاش برای یافتن راهکاری برای ضعف‌های آن‌ها، معماری جدیدی را جهت انجام \واژه{تست} ارائه می‌کنیم.

از نظر ما این معماری یک روش جدید نیست، بلکه تکمیل‌کننده‌ی روش رفتاررانه است. بدین معنا که \واژه{بی.دی.دی.} اهداف کلّی و وضعیت مطلوب را توصیف می‌کند؛ حال آن که معماری پیشنهادی ما تلاش می‌کند تا ساختاری مناسب جهت دست‌یابی به این وضعیت را فراهم کند.

در بخش \رجوع{gwt} با نحوه‌ی بیان یک \واژه{سناریو} آشنا شدیم. این بیان به زبان ریاضی به شکل زیر است:

\شروع{equation}
\برچسب{eq:gIwIt}
g \implies (w \implies t)
\پایان{equation}


نکته‌ی قابل تأمّل آن که عبارت \رجوع{eq:gIwIt} و $g \land w \implies t$ معادلند. حال آن که این دو بخش از منظر معنایی تفاوت محسوسی دارند. مادامی که $g$ ؟؟؟؟


یک سناریو، در واقع پاسخ صحیح یک زیرمسأله از مسأله‌ی اصلی را بیان می‌کند. با معماری پیشنهادی ما، 

معماری پیشنهادی، حول ایده‌ی جداسازی \واژه{سناریو}ها از مصادیق صحّت‌سنجی آن صورت گرفته. در این معماری، \واژه{سناریو}هایی توسّط برنامه‌نویس تعبیه می‌شود که مطابق تعریف، انتظار می‌رود همواره برقرار باشند. مجموعه‌ی این \واژه{سناریو}ها یک 

همچنین، 

با توجّه به تعریف، خود \واژه{سناریو}ها تنها مجموعه‌ای از گزاره‌ها هستند. حال آن که سنجش صحّت این گزاره‌ها می‌تواند \واژه{متریک} خوبی برای کیفیّت کد باشد. لایه‌ی سوم معماری، لایه‌ای است که موجبات trigger شدن این \واژه{سناریو}ها را فراهم می‌کند. به عبارتی، این لایه موظّف است شرایط متنوّعی را فراهم آورد که شروط $given$ سناریوهای متفاوت برقرار شود

به این ترتیب، 

تفاوت معنایی: 


در زیر نمونه‌ای از شبه‌کد آزمون 


\قسمت{فواید}

\زیرقسمت{کاهش حجم تست‌ها}
\برچسب{sec:loc}
یکی از دلایل عمده‌ی عدم علاقه‌ی برنامه‌نویس‌ها به تألیف تست، حجم زیاد تست در مقایسه با میزان کد پوشیده شده توسّط آن است. هر چه سطح تست به سطح سیستمی نزدیک‌تر باشد، حجم این کد نیز به دلیل آماده کردن شرایط اوّلیه‌ی تست افزایش می‌یابد.

\زیرقسمت{کاهش هزینه‌ی نگه‌داری}
از آن جا که تغییر سریع در بسیاری از سیستم‌های نرم‌افزاری ضروری است، حجم زیاد تست‌ها (ر.ک. \رجوع{sec:loc}) در هنگام تغییر نیز نیاز به هم‌گام‌سازی دارند. از آنجا که تست‌ها ???

با توجّه به جدا شدن لایه‌ی \واژه{سناریو} و \واژه{اکتور} در معماری پیشنهادی، تغییرات محدود به بخشی از \واژه{تست}ها خواهند بود که واقعاً تغییر کرده و به بخش‌های دیگر انتشار نمی‌یابند.


\زیرقسمت{افزایش احتمال یافتن خطا}
یکی از تفاوت‌های اساسی معماری پیشنهادی و معماری‌های تک‌لایه در 
از آنجا که یک \واژه{سناریو} 
دسته تست به جای تک تست


\قسمت{در سطح سیستمی}
اگر چه معماری پیشنهادی ما مستقل از سطح \واژه{تست} است، با این حال برداشت مؤلفان این است که تأثیر مثبت آن در زمینه‌ی \واژه{سیستم.تست} مشهودتر خواهد بود. در این لایه به دلیل سطح بالا بودن \واژه{تست}‌ها، حجم آماده‌سازی‌هایی که یک \واژه{تست.کیس} می‌بایست انجام دهد تا به \واژه{پری.کاندیشن} دلخواه برسد بسیار بیشتر بوده و با توجه به آن‌چه در \رجوع{sec:loc} آمد، معماری پیشنهادی از طریق جدا کردن \واژه{اکتور} و \واژه{reuse} آن، به کاهش حجم و هزینه‌ی نگه‌داری این‌گونه \واژه{تست}‌ها کمک شایانی می‌کند.


\قسمت{خارج از محیط \واژه{تست}}
انجام \واژه{تست} تنها یکی از روش‌های افزایش کیفیت نرم‌افزار است. ادعا می‌کنیم که معماری پیشنهادی خارج از محیط \واژه{تست} نیز می‌تواند به افزایش کیفیت نرم‌افزار کمک کند.

در هنگام \واژه{تست}، هدف یافتن \واژه{defect} نرم‌افزار، پیش از عملیاتی شدن آن است. اما بسیاری از \واژه{defect}‌های نرم‌افزاری حتی پس از عملیاتی شدن نیز از دیده نهان می‌مانند.

در \واژه{فریم.ورک} ارائه‌شده، می‌توان به جای لایه‌ی اکتور، کاربران حقیقی \واژه{سیستم} را قرار داد تا از نرم‌افزار استفاده کنند. به این ترتیب، \واژه{سیستم} می‌تواند عملکرد خودش را (حتی در حالی که عملیاتی است) ارزیابی کند و برخی از این \واژه{defect}‌ها را یافته، و جهت رسیدگی توسعه‌دهندگان گزارش کند. از جمله \واژه{defect}‌هایی که به خوبی به این روش پیدا می‌شوند، بروز \واژه{inconsistency} در مقادیر محاسبه‌شدنی است.

\قسمت{مقایسه با مفاهیم دیگر}
\زیرقسمت{در مقایسه با contract}

\زیرقسمت{در مقایسه با mutational testing}




\فصل{پیاده‌سازی}

در این بخش، ابتدا گزارشی از پیاده‌سازی معماری مطرح‌شده در فصل \رجوع{chap:arch} به \واژه{پایتون} ارائه می‌کنیم. پس از آن نحوه
نهایتاً با استفاده از آن برای یک سامانه‌ی آموزش \واژه{تست}هایی تألیف کردیم.

\قسمت{پیاده‌سازی در \واژه{پایتون}}


\قسمت{تألیف \واژه{تست} برای سامانه‌ی آموزش}

برای بررسی صحت عملکرد \واژه{insanity}، یک سامانه‌ی بسیار ساده را طراحی
و در چارچوب \واژه{جنگو} پیاده‌سازی نمودیم تا بتوانیم در آن سامانه،
\واژه{تست}‌هایی را در چارچوب \واژه{insanity} تالیف نماییم.

جهت ملموس‌تر بودن \واژه{انتیتی}‌ها برای خواننده، سامانه‌ی انتخاب واحد را
برای \واژه{تست} انتخاب نمودیم و سعی نمودیم جزییاتی که مرتبط با
\واژه{تست} نیستند را از آن حذف نماییم. در این سامانه تنها عملیات
\واژه{کراد} برای موجودیت‌ها و انتخاب واحد توسط دانشجو پیاده‌سازی شده
است.

\زیرقسمت{موجودیت‌های سامانه‌}

\شروع{شکل}[tbp]
\تنظیم‌ازوسط
\درج‌تصویر[پهنا=0.9\پهنای‌سطر]{edu-entities}
\تر\موقتم{
  نمودار موجودیت‌های سامانه‌ی ثبت نام آموزش
}
\شرح[\موقتم]{\موقتم}
\برچسب{fig:erd}
\پایان{شکل}

همانطور که در شکل \رجوع{fig:erd} نشان داده شده، در حالت ساده این
سامانه دارای موجودیت‌های زیر می‌باشد:

\شروع{شمارش}

\فقره{ \نام{کاربر}{User}: این موجودیت اطلاعات کاربری یک نفر را در
  سامانه نگه می‌دارد. نام، نام خانوادگی و مشخصات اهراز هویت از جمله
  صفات این موجودیت می‌باشند. این موجودیت توسط چارچوب \واژه{جنگو} ارائه
  می‌شود. }

\فقره{ \نام{دانشجو}{Student} و \نام{استاد}{Professor}: در این سامانه
به اطلاعاتی جز مشخصات فردی و دسترسی‌ها برای دانشجو و استاد نیاز
نداریم. این موجودیت‌ها ارتباط یک‌به‌یک با موجودیت کاربر دارند که مشخصات
فردی و دسترسی‌ها در موجودیت کاربر ذخیره می‌شوند. }

\فقره{ \نام{نیمسال تحصیلی}{Semester} و \نام{درس}{Course}: برای این دو
  موجودیت، نگه‌داری صفت «نام» برای آن‌ها در سامانه کافیست. }

\فقره{ \نام{ارائه}{Offering}: ارتباط چند به چند بین موجودیت‌های درس و
  نیمسال تحصیلی و استاد ارائه‌دهنده‌ی آن، در این موجودی نگه‌داری
  می‌شود.

  صفت \کد{capacity}، ظرفیت ارائه را نگه می‌دارد. صفت \کد{available\_capacity}،
  یک صفت محاسبه‌پذیر است که برای کارایی بالاتر، در پایگاه داده ذخیره
  می‌شود. این صفت حاصل تفریق ظرفیت درس از تعداد ثبت‌نام‌های آن است. صفت
  \کد{is\_enrollable} فعال یا غیرفعال بودن قابلیت ثبت نام دانشجویان در آن
  ارائه را مشخص می‌نماید. }

\فقره{ \نام{ثبت نام}{Enrollment}: این موجودیت ارتباط چند به چند ثبت‌نام
  میان موجودیت‌های دانشجو و ارائه را نگهداری می‌کند. }

\پایان{شمارش}

\زیرقسمت{\واژه{یو.آی.} سامانه}

برای شروع به کار سامانه، با اجرای دستور \کد{./manage.py~runserver\_plus}
در پوشه‌ی اصلی برنامه، \واژه{سرور} شروع به کار نموده و می‌توان از طریق
نشانی \کد{http://localhost:8000/} به \واژه{یو.آی.}  سامانه دسترسی پیدا
نمود.

ابتدا نام کاربری و کلمه عبور را وارد می‌نماییم (شکل
\رجوع{fig:edu-login}). سپس بسته به نقشی که کاربر در سامانه داشته باشد،
وارد یکی از صفحه‌های کارمند، دانشجو یا استاد خواهیم شد.

\شروع{شکل}[tbp]
\تنظیم‌ازوسط
\درج‌تصویر[پهنا=1\پهنای‌سطر]{edu-login}
\تر\موقتم{
  صفحه‌ی ورود سامانه‌ی ثبت نام
}
\شرح[\موقتم]{\موقتم}
\برچسب{fig:edu-login}
\پایان{شکل}

در صفحه‌ی کارمند (شکل \رجوع{fig:edu-staff})، می‌توان از منوی سمت چپ، هر
کدام از گزینه‌های کاربر، دانشجو، استاد، درس، نیم‌سال تحصیلی و ارائه را
انتخاب نمود؛ که در تصویر \رجوع{fig:edu-staff}، گزینه‌ی ارائه انتخاب شده
است.

\شروع{شکل}[tbp]
\تنظیم‌ازوسط
\درج‌تصویر[پهنا=1\پهنای‌سطر]{edu-staff}
\تر\موقتم{
  صفحه‌ی کاربری کارمند در سامانه‌ی ثبت نام
}
\شرح[\موقتم]{\موقتم}
\برچسب{fig:edu-staff}
\پایان{شکل}

با انتخاب هر گزینه، لیست موارد ثبت‌شده در سامانه برای آن گزینه قابل
نمایش است. دو ستون نام{ویرایش}{Edit} و نام{حذف}{Delete} در انتهای هر
ردیف از لیست هر گزینه قرار دارد که می‌توان آن ردیف را ویرایش یا حذف
نمود. برای مثال، در شکل \رجوع{fig:edu-edit}، صفحه‌ی مربوط به ویرایش
مشخصات یک ارائه نمایش داده شده است.

\شروع{شکل}[tbp]
\تنظیم‌ازوسط
\درج‌تصویر[پهنا=1\پهنای‌سطر]{edu-edit}
\تر\موقتم{
  صفحه‌ی ویرایش مشخصات ارائه توسط کارمند
}
\شرح[\موقتم]{\موقتم}
\برچسب{fig:edu-edit}
\پایان{شکل}

علاوه بر عملیات ویرایش و حذف، ظرفیت ارائه توسط کارمند قابل تغییر
است. برای این کار، با کلیک بر روی کلید تغییر در ستون
\نام{ظرفیت}{Capacity} از جدول، می‌تواند ظرفیت آن درس را تغییر دهد (شکل
\رجوع{fig:edu-capacity})

\شروع{شکل}[tbp]
\تنظیم‌ازوسط
\درج‌تصویر[پهنا=1\پهنای‌سطر]{edu-capacity}
\تر\موقتم{
  تغییر ظرفیت درس توسط کارمند
}
\شرح[\موقتم]{\موقتم}
\برچسب{fig:edu-capacity}
\پایان{شکل}

همچنین اگر کاربر دسترسی \نام{مدیر سیستم}{Superuser} نیز داشته باشد،
می‌تواند با کلیک استفاده از منوی بالا-راست به قسمت \نام{مدیریت سیستم
  جنگو}{Django Administration} نیز دسترسی پیدا کند و به تمام
موجودیت‌های سامانه دسترسی پیدا کند. برای مثال، می‌تواند از طریق تنظیمات
کاربران (شکل \رجوع{fig:edu-hijack}) بدون نیاز به اطلاع از کلمه‌ی عبور
سایر کاربرها، از طرف آن‌ها وارد سامانه بشود و صفحه‌های قابل مشاهده توسط
آن‌ها را ببیند. نوار زرد‌رنگی که در بالای برخی صفحات در شکل‌های بعد
مشاهده می‌شود، به این خاطر است.

\شروع{شکل}[tbp]
\تنظیم‌ازوسط
\درج‌تصویر[پهنا=1\پهنای‌سطر]{edu-hijack}
\تر\موقتم{
  دسترسی مدیر سامانه به صفحه‌های کاربرها
}
\شرح[\موقتم]{\موقتم}
\برچسب{fig:edu-hijack}
\پایان{شکل}

در صورتی که کاربر وارد شده به سامانه نقش استاد داشته باشد، می‌تواند
لیست درس‌هایی که توسط وی ارائه شده است را مشاهده کند. امکان حذف یا
اضافه‌ی دروس ارائه شده‌ی هر استاد، برای نقش کارمند فعال می‌باشد.

\شروع{شکل}[tbp]
\تنظیم‌ازوسط
\درج‌تصویر[پهنا=1\پهنای‌سطر]{edu-professor}
\تر\موقتم{
  صفحه‌ی کاربری استاد در سامانه‌ی ثبت نام
}
\شرح[\موقتم]{\موقتم}
\برچسب{fig:edu-professor}
\پایان{شکل}

همچنین اگر دانشجو وارد سامانه شود، می‌تواند لیست درس‌هایی که در آن‌ها
ثبت‌نام کرده را مشاهده نماید. (شکل \رجوع{fig:edu-student}).
\شروع{شکل}[tbp]
\تنظیم‌ازوسط
\درج‌تصویر[پهنا=1\پهنای‌سطر]{edu-student}
\تر\موقتم{
  صفحه‌ی کاربری دانشجو در سامانه‌ی ثبت نام
}
\شرح[\موقتم]{\موقتم}
\برچسب{fig:edu-student}
\پایان{شکل}

برای ثبت‌نام در یک ارائه از یک درس، دانشجو می‌تواند بر روی کلید در سمت
بالا-راست صفحه‌ی شکل \رجوع{fig:edu-student} کلیک نماید. در این صورت به
صفحه‌ی ثبت‌نام که در شکل \رجوع{fig:edu-enroll} نمایش داده شده، منتقل
می‌شود. در این صفحه، لیستی از ارائه‌هایی که صفت \کد{is\_enrollable} آن‌ها
فعال باشد، در اختیار دانشجو برای انتخاب قرار می‌گیرد.

\شروع{شکل}[tbp]
\تنظیم‌ازوسط
\درج‌تصویر[پهنا=1\پهنای‌سطر]{edu-enroll}
\تر\موقتم{
  صفحه‌ی ثبت‌نام دانشجو در درس‌های ارائه‌شده توسط سامانه
}
\شرح[\موقتم]{\موقتم}
\برچسب{fig:edu-enroll}
\پایان{شکل}

پس از تایید فرم ثبت‌نام، در صورتی که ظرفیت درس تکمیل شده باشد، با خطای
نشان‌داده شده در شکل \رجوع{fig:edu-full} مواجه می‌شود و دانشجو می‌تواند
درسی دیگر را برای ثبت‌نام انتخاب نماید. در غیر این‌صورت، ثبت‌نام انجام
شده و ردیف مربوط به درس جدید در لیست درس‌های دانشجو مشاهده می‌شود.

\شروع{شکل}[tbp]
\تنظیم‌ازوسط
\درج‌تصویر[پهنا=1\پهنای‌سطر]{edu-full}
\تر\موقتم{
  خطای پر بودن ظرفیت هنگام ثبت‌نام دانشجو در سامانه
}
\شرح[\موقتم]{\موقتم}
\برچسب{fig:edu-full}
\پایان{شکل}

                                                                    

%%% Local Variables:
%%% mode: latex
%%% TeX-master: "../main"
%%% End:




