
\قسمت{خارج از محیط \واژه{تست}}
انجام \واژه{تست} تنها یکی از روش‌های افزایش کیفیّت نرم‌افزار است. ادّعا می‌کنیم که معماری پیشنهادی خارج از محیط \واژه{تست} نیز می‌تواند به افزایش کیفیت نرم‌افزار کمک کند.

در هنگام \واژه{تست} هدف یافتن \واژه{defect} نرم‌افزار پیش از عملیّاتی شدن آن است. امّا بسیاری از \واژه{defect}های نرم‌افزاری حتّی پس از عملیّاتی شدن نیز از دیده نهان می‌مانند.

در صورتی که در معماری سه لایه، به جای لایه‌ی اکتور، کاربران عادی نرم‌افزار را قرار دهیم تا از نرم‌افزار استفاده کنند، می‌توان برخی از این \واژه{defect}ها را پیدا کرد. از جمله \واژه{defect}هایی که به خوبی به این روش پیدا می‌شوند، بروز \واژه{inconsistency} در مقادیر محاسبه‌شدنی است.
