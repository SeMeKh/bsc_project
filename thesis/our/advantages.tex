\قسمت{فواید}
\برچسب{sec:adv}

\زیرقسمت{کاهش حجم\واژه{تست}‌ها}
\برچسب{sec:loc}
یکی از دلایل عمده‌ی عدم علاقه‌ی برنامه‌نویس‌ها به تألیف\واژه{تست}، حجم زیاد\واژه{تست} در مقایسه با میزان کد پوشیده شده توسط آن است. هر چه سطح\واژه{تست} به سطح سیستمی نزدیک‌تر باشد، حجم این کد نیز به دلیل آماده کردن شرایط اولیه‌ی \واژه{تست} افزایش می‌یابد.

با توجه به این که در معماری پیشنهادی، تعریف \واژه{سناریو}‌ها مستقل از داده‌ها انجام می‌شود، امکان \واژه{reuse} از داده‌های یکسان برای\واژه{تست}‌های متفاوت وجود دارد که این موضوع منجر به کاهش حجم \واژه{تست}‌ها می‌شود.

\زیرقسمت{کاهش هزینه‌ی نگه‌داری}
از آن جا که تغییر سریع در بسیاری از\واژه{سیستم}‌های نرم‌افزاری ضروری است، حجم زیاد\واژه{تست}‌ها (ر.ک. \رجوع{sec:loc}) در هنگام تغییر نیز نیاز به هم‌گام‌سازی دارند. در معماری تک‌لایه، با توجه به اینکه شرایط اجرای \واژه{تست} توسط خود آن فراهم می‌شود، \واژه{coupling} به \واژه{component}‌های خارج از محدوده‌ی\واژه{تست} وجود دارد و هنگام تغییر رفتار هر \واژه{component}، این تغییرات در میان تمام \واژه{تست}‌های درگیر انتشار می‌یابد.

با توجه به جدا شدن لایه‌ی \واژه{سناریو} و \واژه{اکتور} در معماری پیشنهادی، تغییرات محدود به بخشی از \واژه{تست}‌ها خواهند بود که واقعاً تغییر کرده و به بخش‌های دیگر انتشار نمی‌یابند.


\زیرقسمت{افزایش احتمال یافتن خطا}
یکی از تفاوت‌های اساسی معماری پیشنهادی و معماری‌های تک‌لایه در کلی بودن تعریف \واژه{سناریو}‌هاست؛ به این معنا که یک \واژه{سناریو} می‌تواند روی داده‌های مختلفی قابل اعمال باشد. بنابراین، هر \واژه{سناریو} عملاً معادل یک مجموعه‌ی \واژه{تست} عمل می‌کند.

این عمومی بودن تعریف \واژه{سناریو} کمک می‌کند تا صحت آن نه فقط روی یک داده، بلکه برای چندین حالت متفاوت بررسی شود. این موضوع احتمال یافتن خطاهای موجود را افزایش داده و در نتیجه موجب افزایش تأثیرگذاری فرآیند \واژه{تست} می‌گردد.
