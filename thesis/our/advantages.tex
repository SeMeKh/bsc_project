\قسمت{فواید}

\زیرقسمت{کاهش حجم تست‌ها}
\برچسب{sec:loc}
یکی از دلایل عمده‌ی عدم علاقه‌ی برنامه‌نویس‌ها به تألیف تست، حجم زیاد تست در مقایسه با میزان کد پوشیده شده توسّط آن است. هر چه سطح تست به سطح سیستمی نزدیک‌تر باشد، حجم این کد نیز به دلیل آماده کردن شرایط اوّلیه‌ی تست افزایش می‌یابد.

\زیرقسمت{کاهش هزینه‌ی نگه‌داری}
از آن جا که تغییر سریع در بسیاری از سیستم‌های نرم‌افزاری ضروری است، حجم زیاد تست‌ها (ر.ک. \رجوع{sec:loc}) در هنگام تغییر نیز نیاز به هم‌گام‌سازی دارند. از آنجا که تست‌ها ???

با توجّه به جدا شدن لایه‌ی \واژه{سناریو} و \واژه{اکتور} در معماری پیشنهادی، تغییرات محدود به بخشی از \واژه{تست}ها خواهند بود که واقعاً تغییر کرده و به بخش‌های دیگر انتشار نمی‌یابند.


\زیرقسمت{افزایش احتمال یافتن خطا}
یکی از تفاوت‌های اساسی معماری پیشنهادی و معماری‌های تک‌لایه در 
از آنجا که یک \واژه{سناریو} 
دسته تست به جای تک تست

