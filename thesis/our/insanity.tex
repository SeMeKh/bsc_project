\قسمت{پیاده‌سازی بر مبنای \واژه{فریم.ورک} \واژه{جنگو}}
\برچسب{sec:insanity}
به عنوان بخشی از این پروژه، یک \واژه{فریم.ورک} \واژه{تست} با \واژه{جنگو} و مبتنی بر معماری ارائه‌شده در فصل \رجوع{chap:arch} را پیاده‌سازی کردیم.


\زیرقسمت{نحوه‌ی استفاده}
این \واژه{فریم.ورک} در قالب یک افزونه برای \واژه{جنگو} نوشته شده. برای استفاده از این افزونه نیاز است تا:
\شروع{enumerate}
\فقره نام افزونه (\کد{insanity}) به لیست افزونه‌های \واژه{جنگو} اضافه گردد.
\فقره \واژه{سناریو}‌های مطلوب در فایل \کد{scenarios.py} مرتبط با هر \واژه{component} \واژه{جنگو} قرار بگیرند.
\فقره هر کدام از توابعی که به عنوان \واژه{trigger} \واژه{سناریو}‌ها عمل می‌کنند، توسط \واژه{decorator} \کد{action} پوشیده شوند.
\پایان{enumerate}

\زیرقسمت{اجزاء پیاده‌سازی}

\زیرزیرقسمت{\واژه{decorator} \کد{action}}
این \واژه{decorator} به توابع مختلف اعمال می‌شود و آن‌ها را به عنوان یک \واژه{trigger} \واژه{سناریو} ثبت می‌کند. به این ترتیب، در صورت اجرای این توابع، صحت \واژه{سناریو}‌های مرتبط با آن‌ها بررسی می‌شود.

\زیرزیرقسمت{\واژه{کلاس} \کد{Scenario}}
هر \واژه{سناریو} به صورت یک \واژه{کلاس} پیاده‌سازی می‌شود که از \کد{Scenario} ارث می‌برد. این کلاس دارای توابع \کد{given}، \کد{when} و \کد{then} است که معادل مفاهیم متناظر در \واژه{بی.دی.دی.} هستند.

\زیرقسمت{توضیحات بیشتر}
مستندات سطح کد به همراه کد منبع این \واژه{فریم.ورک} از طریق آدرس زیر در دسترس است:

\begin{center}
\url{https://github.com/SeMeKh/bsc_project}
\end{center}
