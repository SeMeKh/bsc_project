\قسمت{پیاده‌سازی بر مبنای \واژه{فریم.ورک} \واژه{جنگو}}
\برچسب{sec:insanity}
به عنوان بخشی از پروژه کارشناسی، یک \واژه{فریم.ورک} برای \واژه{تست}
\واژه{بی.دی.دی.} با \واژه{جنگو} و مبتنی بر معماری ارائه‌شده در فصل \رجوع{sec:arch} را پیاده‌سازی نمودیم.


\زیرقسمت{نحوه‌ی استفاده}

این \واژه{فریم.ورک} در قالب یک افزونه برای \واژه{جنگو} نوشته شده. برای
استفاده از این افزونه مراحل زیر را انجام دهید:

\شروع{شکل}[tbp]
\تنظیم‌ازوسط
\درج‌تصویر[پهنا=0.7\پهنای‌سطر]{installed-apps}
\تر\موقتم{
  افزودن چارچوب به جنگو
}
\شرح[\موقتم]{\موقتم}
\برچسب{fig:installed-apps}
\پایان{شکل}

\شروع{شکل}[tbp]
\تنظیم‌ازوسط
\درج‌تصویر[پهنا=0.7\پهنای‌سطر]{sentry-url}
\تر\موقتم{
  ورود تنظیمات سنتری
}
\شرح[\موقتم]{\موقتم}
\برچسب{fig:sentry-url}
\پایان{شکل}


\شروع{enumerate} \فقره {با اجرای دستور \\ {\کد{pip install
      https://github.com/SeMeKh/bsc\_project/archive/master.zip}، \\
    چارچوب \واژه{insanity} را نصب نمایید. }}

\فقره { نام افزونه (\کد{insanity}) را مطابق شکل
  \رجوع{fig:installed-apps} به لیست افزونه‌ها در تنظیمات \واژه{جنگو}
  اضافه نمایید. }

\فقره{ تنظیمات \کد{sentry} را جهت ثبت جزییات خطاها مطابق شکل
  \رجوع{fig:sentry-url} در فایل \کد{settings.py} وارد نمایید.}
  
\فقره { \واژه{سناریو}‌های مطلوب در فایل \کد{scenarios.py} مرتبط با هر
  \واژه{component} \واژه{جنگو} قرار بگیرند. هر سناریو، از با یک
  \واژه{کلاس} نمایش داده می‌شود که  }

\فقره { هر کدام از توابعی که به عنوان \واژه{trigger}
  \واژه{سناریو}‌ها عمل می‌کنند، توسط \واژه{decorator} \کد{action} پوشیده
  شوند. }

\پایان{enumerate}

\زیرقسمت{اجزاء پیاده‌سازی}

\زیرزیرقسمت{\واژه{decorator} \کد{action}}
این \واژه{decorator} به توابع مختلف اعمال می‌شود و آن‌ها را به عنوان یک \واژه{trigger} \واژه{سناریو} ثبت می‌کند. به این ترتیب، در صورت اجرای این توابع، صحت \واژه{سناریو}‌های مرتبط با آن‌ها بررسی می‌شود.

\زیرزیرقسمت{\واژه{کلاس} \کد{Scenario}}
هر \واژه{سناریو} به صورت یک \واژه{کلاس} پیاده‌سازی می‌شود که از \کد{Scenario} ارث می‌برد. این کلاس دارای توابع \کد{given}، \کد{when} و \کد{then} است که معادل مفاهیم متناظر در \واژه{بی.دی.دی.} هستند.

\زیرقسمت{کد منبع}
کد منبع \واژه{فریم.ورک} پیاده‌سازی‌شده از طریق آدرس زیر در دسترس است:

\begin{center}
\url{https://github.com/SeMeKh/bsc_project}
\end{center}
