\قسمت{پیاده‌سازی بر مبنای \واژه{فریم.ورک} \واژه{جنگو}}
\برچسب{sec:insanity}
به عنوان بخشی از پروژه کارشناسی، یک \واژه{فریم.ورک} برای \واژه{تست}
\واژه{بی.دی.دی.} با \واژه{جنگو} و مبتنی بر معماری ارائه‌شده در فصل \رجوع{sec:arch} را پیاده‌سازی نمودیم.


\زیرقسمت{نحوه‌ی استفاده}

این \واژه{فریم.ورک} در قالب یک افزونه برای \واژه{جنگو} نوشته شده. برای
استفاده از این افزونه مراحل زیر را انجام دهید:

\شروع{شکل}[tbp]
\تنظیم‌ازوسط
\درج‌تصویر[پهنا=0.7\پهنای‌سطر]{installed-apps}
\تر\موقتم{
  افزودن چارچوب به جنگو
}
\شرح[\موقتم]{\موقتم}
\برچسب{fig:installed-apps}
\پایان{شکل}

\شروع{شکل}[tbp]
\تنظیم‌ازوسط
\درج‌تصویر[پهنا=0.7\پهنای‌سطر]{sentry-url}
\تر\موقتم{
  ورود تنظیمات \واژه{سنتری}
}
\شرح[\موقتم]{\موقتم}
\برچسب{fig:sentry-url}
\پایان{شکل}


\شروع{enumerate} \فقره {با اجرای دستور \\ {\کد{pip install
      https://github.com/SeMeKh/bsc\_project/archive/master.zip}، \\
    چارچوب \واژه{insanity} را نصب نمایید. }}

\فقره { نام افزونه (\کد{insanity}) را مطابق شکل
  \رجوع{fig:installed-apps} به لیست افزونه‌ها در تنظیمات \واژه{جنگو}
  اضافه نمایید. }

\فقره{ تنظیمات \واژه{سنتری} را جهت ثبت جزییات خطاها مطابق شکل
  \رجوع{fig:sentry-url} در فایل \کد{settings.py} وارد نمایید.}
  
\فقره { \واژه{سناریو}‌های مطلوب را در فایل \کد{scenarios.py} مرتبط با
  هر \واژه{component} \واژه{جنگو} قرار دهید. هر سناریو، با یک
  \واژه{کلاس} که از کلاس پایه‌ی \کد{Scenario} ارث می‌برد، نمایش داده
  می‌شود که موظف است در \واژه{متد} «\واژه{given}»، پیش‌شرایط سناریو را
  در حالت سامانه بررسی کند و در صورتی که پیش‌شرایط برآورده می‌شوند،
  مقدار \کد{True} را باز گرداند. صفت «\واژه{when}» نام کنشی که این
  سناریو را فعال می‌کند را تعیین می‌کند. در متد «\واژه{then}» نیز
  \واژه{پست.کاندیشن}‌های سناریو را بررسی نمایید.}

\فقره { هر کدام از توابعی که به عنوان \واژه{trigger}
  \واژه{سناریو}‌ها عمل می‌کنند، توسط \واژه{decorator} \کد{action} پوشیده
  شوند. }

\فقره { به طریق دلخواه، با سامانه تعامل ایجاد نمایید به طوریکه
  \واژه{اکشن}‌های تعیین شده توسط سناریوها اعمال شوند. }

\فقره { از نشانی \کد{http://localhost:8000/insanity} نتیجه‌ی بررسی
  سناریو‌ها را مشاهده کنید. همانطور که در شکل \رجوع{fig:report} مشاهده
  می‌نمایید، می‌توان برای هر سناریو تعداد اجراهای موفق و ناموفق نمایش
  داده می‌شود. همچنین مطابق شکل \رجوع{fig:sentry}، با کلیک بر روی علامت
  زنجیر در کنار سناریوهایی که ناموفق بوده‌اند، به صفحه‌ی \واژه{سنتری}
  مرتبط به آن سناریو منتقل می‌شوید و می‌توانید از حالت برنامه هنگام شکست
سناریو مطلع شوید.}

\پایان{enumerate}


\شروع{شکل}[tbp]
\تنظیم‌ازوسط
\درج‌تصویر[پهنا=0.9\پهنای‌سطر]{sentry}
\تر\موقتم{
  نمایش جزییات شکست یک سناریو در \واژه{سنتری}
}
\شرح[\موقتم]{\موقتم}
\برچسب{fig:sentry}
\پایان{شکل}


\شروع{شکل}[tbp]
\تنظیم‌ازوسط
\درج‌تصویر[پهنا=0.7\پهنای‌سطر]{report}
\تر\موقتم{
  گزارش پوشش سناریوهای تعریف شده
}
\شرح[\موقتم]{\موقتم}
\برچسب{fig:report}
\پایان{شکل}

\زیرقسمت{اجزاء پیاده‌سازی}

\زیرزیرقسمت{\واژه{decorator} \کد{action}}
این \واژه{decorator} به توابع مختلف اعمال می‌شود و آن‌ها را به عنوان یک
\واژه{trigger} \واژه{سناریو} ثبت می‌کند. به این ترتیب، در صورت اجرای
این توابع، صحت \واژه{سناریو}‌های مرتبط با آن‌ها بررسی می‌شود.

همانطور که در نمودار توالی شکل \رجوع{fig:sd} نمایش داده شده است، اگر
یک \واژه{متد} (مانند متد \کد{enroll} از کلاس \کد{Offering} در شکل)
توسط \واژه{decorator} \کد{action} نشانه‌گذاری شده باشد، هنگام فراخوانی
آن تابع، پیش از رسیدن کنترل برنامه به تابع \کد{enroll}، کنترل به دست
\واژه{decorator} می‌رسد و این دکوراتور پیش از فراخوانی تابع اصلی،
سناریو‌هایی که \کد{when} آن‌ها برابر \کد{enroll} باشد را انتخاب نموده،
پیش از فراخوانی تابع اصلی \کد{given} آن‌ها را اجرا نموده و اگر مقدار
خروجی قسمت \کد{given} برابر \کد{True} باشد، پس از فراخوانی تابع اصلی
\کد{enroll}، تابع \کد{then} آن سناریوها را اجرا می‌نماید.

\
\شروع{شکل}[tbp]
\تنظیم‌ازوسط
\درج‌تصویر[پهنا=0.9\پهنای‌سطر]{sd}
\تر\موقتم{
  نمودار توالی عملکرد \کد{decorator}
}
\شرح[\موقتم]{\موقتم}
\برچسب{fig:sd}
\پایان{شکل}

\زیرزیرقسمت{\واژه{کلاس} \کد{Scenario}} هر \واژه{سناریو} به صورت یک
\واژه{کلاس} پیاده‌سازی می‌شود که از \کد{Scenario} ارث می‌برد. این کلاس
دارای توابع \کد{given}، \کد{when} و \کد{then} است که معادل مفاهیم
متناظر در \واژه{بی.دی.دی.} هستند.

\زیرقسمت{کد منبع}
کد منبع \واژه{فریم.ورک} پیاده‌سازی‌شده از طریق آدرس زیر در دسترس است:

\begin{center}
\url{https://github.com/SeMeKh/bsc_project}
\end{center}
